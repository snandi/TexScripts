%\documentclass[10pt,dvipsnames,table, handout]{beamer} % To printout the slides without the animations
\documentclass[10pt,dvipsnames,table]{beamer} 
%\usetheme{Luebeck} 
%\usetheme{Madrid} 
%\usetheme{Marburg} 
%\usetheme{Warsaw} 
%\setbeamercolor{structure}{fg=cyan!90!white}
%\setbeamercolor{normal text}{fg=white, bg=black}
\usetheme{CambridgeUS}
%\setbeamertemplate{enumerate items}[default]
\setbeamercolor{item projected}{fg=green,bg=red}
%\setbeamercolor{itemize subitem}{fg=blue}
%\setbeamercolor{structure}{fg=cyan!90!white}
%\setbeamercolor{normal text}{fg=white, bg=black}
\setbeamercolor{block title}{bg=red!80,fg=white}

%%%%%%%%%%%%%%%%%%%%%%%%%%%%%%%%%%%%%%%%%%%%%%%%%%%%%%%%%%%%%%%%%%%%%%
%% Input header file 
%%%%%%%%%%%%%%%%%%%%%%%%%%%%%%%%%%%%%%%%%%%%%%%%%%%%%%%%%%%%%%%%%%%%%%
%%%%%%%%%%%%%%%%%%%%%%%% Packages %%%%%%%%%%%%%%%%%%%%%%%%
\usepackage{amscd}
\usepackage{amsmath}
\usepackage{amssymb}
\usepackage{amsthm}
\usepackage{amsxtra}
\usepackage{animate}
\usepackage{bbold}
%\usepackage{bigints}
\usepackage{color, colortbl}
\usepackage{dsfont}
\usepackage{enumerate}
\usepackage[mathscr]{eucal}
%\usepackage{fancyhdr}
\usepackage{float}
%\usepackage{fullpage} %% Dont use this for beamer presentations
\usepackage{geometry}
\usepackage{graphicx}
\usepackage{hyperref}
\usepackage{indentfirst}
\usepackage{latexsym}
\usepackage{listings}
\usepackage{lscape}
\usepackage{mathtools}
\usepackage{microtype}
\usepackage{multirow}
\usepackage{natbib}
\usepackage{pdfpages}
\usepackage{verbatim}
\usepackage{wrapfig}
\usepackage{xargs}
\usepackage{xcolor}
\DeclareGraphicsExtensions{.pdf,.png,.jpg, .jpeg}
\definecolor{LightCyan}{rgb}{0.88,1,1}

%%%%%%%%%%%%%%%%%%%%%%%% Commands %%%%%%%%%%%%%%%%%%%%%%%%
\newcommand{\Sup}{\textsuperscript}
\newcommand{\Exp}{\mathds{E}}
\newcommand{\Prob}{\mathds{P}}
\newcommand{\Z}{\mathds{Z}}
\newcommand{\Ind}{\mathds{1}}
\newcommand{\A}{\mathcal{A}}
\newcommand{\F}{\mathcal{F}}
\newcommand{\G}{\mathcal{G}}
\newcommand{\I}{\mathcal{I}}
\newcommand{\R}{\mathcal{R}}
\newcommand{\Real}{\mathbb{R}}
\newcommand{\be}{\begin{equation}}
\newcommand{\ee}{\end{equation}}
\newcommand{\bes}{\begin{equation*}}
\newcommand{\ees}{\end{equation*}}
\newcommand{\union}{\bigcup}
\newcommand{\intersect}{\bigcap}
\newcommand{\Ybar}{\overline{Y}}
\newcommand{\ybar}{\bar{y}}
\newcommand{\Xbar}{\overline{X}}
\newcommand{\xbar}{\bar{x}}
\newcommand{\betahat}{\hat{\beta}}
\newcommand{\Yhat}{\widehat{Y}}
\newcommand{\yhat}{\hat{y}}
\newcommand{\Xhat}{\widehat{X}}
\newcommand{\xhat}{\hat{x}}
\newcommand{\E}[1]{\operatorname{E}\left[ #1 \right]}
%\newcommand{\Var}[1]{\operatorname{Var}\left( #1 \right)}
\newcommand{\Var}{\operatorname{Var}}
\newcommand{\Cov}[2]{\operatorname{Cov}\left( #1,#2 \right)}
\newcommand{\N}[2][1=\mu, 2=\sigma^2]{\operatorname{N}\left( #1,#2 \right)}
\newcommand{\bp}[1]{\left( #1 \right)}
\newcommand{\bsb}[1]{\left[ #1 \right]}
\newcommand{\bcb}[1]{\left\{ #1 \right\}}
\newcommand*{\permcomb}[4][0mu]{{{}^{#3}\mkern#1#2_{#4}}}
\newcommand*{\perm}[1][-3mu]{\permcomb[#1]{P}}
\newcommand*{\comb}[1][-1mu]{\permcomb[#1]{C}}


%%%%%%%%%%%%%%%%%%%%%%%%%%%%%%%%%%%%%%%%%%%%%%%%%%%%%%%%%%%%%%%%%%%%%%
%% TITLE PAGE 
%%%%%%%%%%%%%%%%%%%%%%%%%%%%%%%%%%%%%%%%%%%%%%%%%%%%%%%%%%%%%%%%%%%%%%
\DeclarePairedDelimiter\ceil{\lceil}{\rceil}
\title[Fluoroscanning]{Fluoroscanning: {\emph{Next Generation Precision Genomics}}}
\author[S. Nandi]{Subhrangshu Nandi}
\institute[LMCG]{	Laboratory of Molecular and Computational Genomics \\
 			University of Wisconsin - Madison }
\date{January 20, 2017}

\begin{document}
\setlength{\baselineskip}{16truept}
\frame{\maketitle}

%%%%%%%%%%%%%% Slide 1 %%%%%%%%%%%%%%
\begin{frame}
\frametitle{Motivation}
{\Large{
Next generation genomic sciences need to build systems to
\begin{enumerate}
\item Analyze genomes of every individual of the population
\item Analyze the complexities of cancer genomes
\item Develop targeted and precise treatment options for complex diseases
\end{enumerate}
in an {\underline{economical}} and {\underline{parsimonious}} manner.
}}
\end{frame}
%%%%%%%%%%%%%%%%%%%%%%%%%%%%%%%%%%%%%

%%%%%%%%%%%%%% Slide 2 %%%%%%%%%%%%%%
\begin{frame}
\frametitle{Limitations of Sequencing Technologies \footnote{\tiny{Wikipedia}} }
\vspace{-0.8cm}
\begin{columns}[t]
\begin{column}{0.6\textwidth}
\begin{itemize} 

\item {\bf{\small{Pac Bio}}}
{\footnotesize{only 87\% accuracy; moderate throughput }}

\item {\bf{\small{Ion Torrent Sequencing}}}
{\footnotesize{up to 400 bp read length; homo-polymer errors }}

\item {\bf{\small{Illumina sequencing}}}
{\footnotesize{up to 300 bp read length; very expensive equipment}}

\item {\bf{\small{Sanger sequencing}}}
{\footnotesize{impractical for large sequencing projects}}

\end{itemize}
\end{column}

\begin{column}{0.4\textwidth}
\begin{center}
\begin{figure}[t]
\includegraphics[scale=0.28]{Images/sequencingtechnology.pdf} \footnote{\tiny{https://www.broadinstitute.org/ \\blog/celebrating-fruits-human-genome-sequence}}
\end{figure}
\end{center}
\end{column}

\end{columns}

\begin{block}{Common limitations}
\begin{itemize}
{\small{
\item CANNOT completely sequence a human genome because of repetitive structures \footnote{\cite{Lander_etal_2001_Nature}} and other complexities
\item CANNOT detect heterozygotes \footnote{\cite{Wheeler_etal_2008_Nature}}
\item are Inadequate for analyzing complex cancer genomes, or other polygenic diseases
}}
\end{itemize}
\end{block}
\note{}
\end{frame}
%%%%%%%%%%%%%%%%%%%%%%%%%%%%%%%%%%%%%

%%%%%%%%%%%%%% Slide x %%%%%%%%%%%%%%
\begin{frame}
\frametitle{Motivation: mp3 representation of a Genome}
\vspace{-0.5cm}
\begin{columns}[t]
\begin{column}{0.5\textwidth}
\vspace{-0.5cm}
\begin{figure}[t]
\includegraphics[scale=0.24]{Images/wav.png} 
\end{figure}

\vspace{-0.5cm}
\begin{itemize}
{\footnotesize{
\item Wave is an uncompressed or ``lossless'' format
\item High sampling frequency of sound waves; high quality of music
\item Extremely large file size
}}
\end{itemize}

\begin{figure}[H]
\includegraphics[scale=0.18]{Images/200114_Genome_650.jpg} 
\end{figure}

\end{column}

\begin{column}{0.5\textwidth}
\vspace{-0.5cm}
\begin{figure}[t]
\includegraphics[scale=0.18]{Images/mp3.jpg}
\end{figure}

\vspace{-0.5cm}
\begin{itemize}
{\footnotesize{
\item MP3 is a compressed or ``lossy'' format
\item Reduces the signal to its most necessary components
\item Very manageable file size; Easily transferable
}}
\end{itemize}
\vspace{-0.5cm}
\begin{figure}[H]
\hspace{-1cm}
\includegraphics[scale=0.3]{Plots/chr13_frag1000_consensusOnly.pdf} \\
\vspace{-0.5cm} \hspace{0.5cm}
\includegraphics[scale=0.28]{Plots/chr13_frag1025_consensusOnly.pdf} \\
\vspace{-0.5cm} \hspace{-2cm}
\includegraphics[scale=0.3]{Plots/chr13_frag1050_consensusOnly.pdf} \\
\vspace{-0.5cm} \hspace{1cm}
\includegraphics[scale=0.3]{Plots/chr13_frag1052_consensusOnly.pdf} \\
\end{figure}

\end{column}
\end{columns}
{\tiny{Images from \footnote{\tiny{https://nasiirblog.wordpress.com/2014/05/22/wav-vs-mp3/, http://archive.cosmosmagazine.com/}} }}
\end{frame}
%%%%%%%%%%%%%%%%%%%%%%%%%%%%%%%%%%%%%

%%%%%%%%%%%%%% Slide x %%%%%%%%%%%%%%
\begin{frame}
\frametitle{Fluoroscanning}
\begin{block}{Fluoroscanning}
A system to extract usable information about genomic sequences from fluorescence intensity profiles of imaged DNA molecules. 
\end{block}
\end{frame}
%%%%%%%%%%%%%%%%%%%%%%%%%%%%%%%%%%%%%

%%%%%%%%%%%%%% Slide x %%%%%%%%%%%%%%
\begin{frame}
\includegraphics[scale=0.45]{Images/FluoroscanningWorkflow_v3.pdf}
\end{frame}
%%%%%%%%%%%%%%%%%%%%%%%%%%%%%%%%%%%%%

%%%%%%%%%%%%%% Slide x %%%%%%%%%%%%%%
\begin{frame}
\frametitle{Reference}
{\footnotesize{
    \bibliographystyle{apalike}
    \bibliography{bibTex_Reference}
}}
\end{frame}

\end{document}

