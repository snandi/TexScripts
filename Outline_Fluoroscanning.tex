\documentclass[11pt]{extarticle} %extarticle for fontsizes other than 10, 11 And 12
%\documentclass[11p]{article}

%%%%%%%%%%%%%%%%%%%%%%%%%%%%%%%%%%%%%%%%%%%%%%%%%%%%%%%%%%%%%%%%%%%%%%
%% Input header file 
%%%%%%%%%%%%%%%%%%%%%%%%%%%%%%%%%%%%%%%%%%%%%%%%%%%%%%%%%%%%%%%%%%%%%%
%%%%%%%%%%%%%%%%%%%%%%%% Packages %%%%%%%%%%%%%%%%%%%%%%%%
\usepackage{amscd}
\usepackage{amsmath}
\usepackage{amssymb}
\usepackage{amsthm}
\usepackage{amsxtra}
\usepackage{animate}
\usepackage{bbold}
%\usepackage{bigints}
\usepackage{caption}    %% For multiple line captions
\usepackage{color, colortbl}
\usepackage{dsfont}
\usepackage{enumerate}
\usepackage[mathscr]{eucal}
%\usepackage{fancyhdr}
\usepackage{float}
%\usepackage{fullpage}  %% Dont use this for beamer presentations
\usepackage{geometry}
\usepackage{graphicx}
\usepackage{hyperref}
\usepackage{indentfirst}
\usepackage{latexsym}
\usepackage{listings}
\usepackage{longtable}  %% to add pagebreaks in between table
\usepackage{lscape}
\usepackage{mathtools}
\usepackage{microtype}
\usepackage{multirow}
\usepackage{natbib}
\usepackage{pdfpages}
\usepackage{setspace}   %% Allows to set double or single space
\usepackage{tcolorbox}  %% For colored textboxes
\usepackage{verbatim}
\usepackage{wrapfig}
\usepackage{xargs}
\usepackage{xcolor}
\DeclareGraphicsExtensions{.pdf,.png,.jpg, .jpeg}
\definecolor{LightCyan}{rgb}{0.88,1,1}

\usepackage{array}
\newcolumntype{C}[1]{>{\centering\arraybackslash}p{#1}}  %% For wrapping text in table headers

%%%%%%%%%%%%%%%%%%%%%%%% Commands %%%%%%%%%%%%%%%%%%%%%%%%
\newcommand{\Sup}{\textsuperscript}
\newcommand{\Exp}{\mathds{E}}
\newcommand{\Prob}{\mathds{P}}
\newcommand{\Z}{\mathds{Z}}
\newcommand{\Ind}{\mathds{1}}
\newcommand{\A}{\mathcal{A}}
\newcommand{\F}{\mathcal{F}}
%\newcommand{\G}{\mathcal{G}}
\newcommand{\I}{\mathcal{I}}
\newcommand{\R}{\mathcal{R}}
\newcommand{\Y}{\mathcal{Y}}
\newcommand{\Real}{\mathbb{R}}
\newcommand{\be}{\begin{equation}}
\newcommand{\ee}{\end{equation}}
\newcommand{\bes}{\begin{equation*}}
\newcommand{\ees}{\end{equation*}}
\newcommand{\union}{\bigcup}
\newcommand{\intersect}{\bigcap}
\newcommand{\Ybar}{\overline{Y}}
\newcommand{\ybar}{\bar{y}}
\newcommand{\Xbar}{\overline{X}}
\newcommand{\xbar}{\bar{x}}
\newcommand{\betahat}{\hat{\beta}}
\newcommand{\Yhat}{\widehat{Y}}
\newcommand{\yhat}{\hat{y}}
\newcommand{\Xhat}{\widehat{X}}
\newcommand{\xhat}{\hat{x}}
\newcommand{\E}[1]{\operatorname{E}\left[ #1 \right]}
%\newcommand{\Var}[1]{\operatorname{Var}\left( #1 \right)}
\newcommand{\Var}{\operatorname{Var}}
\newcommand{\Cov}[2]{\operatorname{Cov}\left( #1,#2 \right)}
\newcommand{\N}[2][1=\mu, 2=\sigma^2]{\operatorname{N}\left( #1,#2 \right)}
\newcommand{\bp}[1]{\left( #1 \right)}
\newcommand{\bsb}[1]{\left[ #1 \right]}
\newcommand{\bcb}[1]{\left\{ #1 \right\}}
\newcommand*{\permcomb}[4][0mu]{{{}^{#3}\mkern#1#2_{#4}}}
\newcommand*{\perm}[1][-3mu]{\permcomb[#1]{P}}
\newcommand*{\comb}[1][-1mu]{\permcomb[#1]{C}}
\newcommand{\indep}{\rotatebox[origin=c]{90}{$\models$}}

\DeclareMathOperator*{\argmin}{arg\,min}


\begin{document}
%\SweaveOpts{concordance=TRUE}
\bibliographystyle{plain}  %Choose a bibliograhpic style

\title{Fluoroscanning: Next generation precision genomics}
\author{Subhrangshu Nandi \\
%  Statistics PhD Student, \\
%  Research Assistant,
%  Laboratory of Molecular and Computational Genomics, \\
%  University of Wisconsin - Madison}
%\date{February 16, 2015}
\date{}
}

\maketitle


\section*{Abstract}
The Human Genome Project (HGP), completed in 2003, is considered one of the greatest accomplishments of exploration in history of science. Since then thousands of genomes have been sequenced. However, no individual human genome has been annotated to completion. Nanocoding \cite{Jo_etal_2007_PNAS} (PNAS, 2007), developed by Laboratory of Molecular and Computational Genomics (LMCG), UW Madison , is a novel system for physically mapping genomes, using measurements of single DNA molecules to construct a high-resolution genome-wide restriction map, whose representation of genome structure complements genome sequences to yield biological insight. Nanocoding is scalable, robust, reliable system that

\newpage

\section*{Motivation}
The Human Genome Project (HGP), completed in 2003, is considered one of the greatest accomplishments of exploration in history of science. Since then thousands of genomes have been sequenced. However, no individual human genome has been annotated to completion. DNA sequencing-based genomic analysis continue to evolve, but their abilities to detect large scale structural variations, or heterozygosity in diploid genomes, remain limited. Next generation sequencing (NGS) is considerable more cost effective, with longer reads, but still have difficulty inferring repetitive structures and duplications \cite{Lander_etal_2001_Nature}, further complicated by gaps and errors in the reference genome. NGS also has inferior sensitivity of detecting heterozygotes \cite{Wheeler_etal_2008_Nature}. In addition, the sheer size of the diploid human genome (6 gigabases) presents multiple challenges of just using NGS technologies to analyze its complexities. The analysis of cancer genomes is made even more complex by the accumulation of large and small-scale structural variations (SVs), and genotype heterogeneity fostered by on-going mutagenesis processes, especially apparent in in solid tumor. The sequencing technologies currently being used were developed  primarily for characterization of single genes, not entire genomes and, as such, are not ideal to analyze polygenic diseases, complex trait inheritance, and population-based molecular genetics \cite{Samad_etal_1995_GenomeResearch}. Given the current need for comprehensively analyzed human and cancer genomes that are readily created, contemporary sequencing and mapping approaches are insufficient to meet these challenges. In order to achieve our dreams of improving healthcare by precision genomics we need techniques that can overcome the shortcomings of NGS, yet, maintaining economic viability. The answer lies in the latest developments of single molecule genome mapping techniques like optical mapping and nanocoding. 

\section*{Background} 
\noindent
{\bf{Optical Mapping}} \\
Pioneered by LMCG, single molecule genome mapping techniques like optical mapping \cite{Schwartz_etal_1993_Science}, \cite{Dimalanta_etal_2004_AnalChem}, and nanocoding \cite{Jo_etal_2007_PNAS} have changed the landscape of whole genome analysis. Optical Mapping is a novel platform for analyzing genomes: it uses measurements of single DNA molecules to infer a high-resolution genome-wide restriction map, whose representation of genome structure complements genome sequence to yield biological insight. Briefly, DNA from thousands of cells in solution is randomly sheared to produce pieces that are around 500 Kb long. The solution is then passed through a micro-channel, where the DNA molecules are stretched and then attached to a positively charged glass support. A restriction enzyme is then applied, cleaving the DNA at corresponding restriction sites. The DNA molecules remain attached to the surface. The surface is photographed under a microscope after being stained with a fluorochrome. The cleavage sites show up in the image as tiny gaps in the fluorescent line of the molecule, giving an snapshot of the full restriction map. Even though these molecules are large by many standards, they may still represent only a small fraction of the chromosome they come from. Naturally, the amount of information in an optical map data set is related to the size of the underlying genome. 

Information about genomic variation can thus be obtained from these restriction maps, that do not record the full nucleotide sequence. A physical map is a listing of the locations along the genome where certain markers occur. A restriction map is a physical map induced by restriction enzymes. The ordered sequence of distances in base pairs between successive marker positions summarizes the genome sequence and can be viewed as a sort of bar code of the genome. Genomic differences can affect the presence or absence of markers, the distances between them and their orientation, inducing analogous changes in the bar code. Having a reference copy of the human genome allows us to perform such {\emph{in silico}} experiments. The availability of in silico reference maps can be extremely helpful. \\

\noindent
{\bf{Nanocoding}} \\
Nanocoding (1), also invented in LMCG, is also a novel DNA barcoding technique, to obtain genome-wide restriction maps.  \\

\noindent
{\bf{Dye - DNA interaction}} \\
Stuff

\section*{Discovery of a new effect - Fluoroscanning}
Sequence/dye interactions produce reproducible fluoresce intensity profiles when optically measured along the axis of individual DNA molecules. Variables include DNA stretch, specific dye, dye/DNA ratio (stoichiometry)

\section*{Fluorescence Intensity Profiles}
Show examples of fluorescence intensity profiles—human (MM52), M. florum,  BACs
\begin{itemize}
\item Noise considerations (select a set of noise metrics—explain them, justify them and use them consistently throughout the manuscript) that stem from: stretching (show analysis that links stretching with fluorescence signal information content (this curve may look parabolic…..), amount of dye, staing techniques….How fluorescence signals are measured and pre-processed—METHODS SECTION (INCA, etc.)
\item Dealing with noise (Introduction)—Systematic vs. random effects. How a signal consensus making approach deals must deal with types of noise through preprocessing (automated ranking of fluorescence signal QUAILTY (mathematical, morphological analysis): uneven stretching, dispersed stretching within an interval dataset, extraneous noise (“fluorescent garbage,” nearby molecules , crossing molecules, etc.). 
\item The Consensus algorithm- Outline algorithm, describe in detail, pseudocode???……..examples of synthetic data, data comparisons using previously. Explain how variants of this algorithm/analysis have been used in other applications…
\item Preprocessing to improve consensus signal making— Describe both mathematical and computational algorithms and workflow. Comparison and analysis of examples drawn from human, synthetic, M. florum, and BAC data, showing how more confident fluoroscans emerge from considered filtering. Use noise metrics to prove this point.
\end{itemize}

\section*{Reproducible Fluorescence Intensity signals}
Discuss mathematics of curve registration, and estimation of consensus signals
Discuss estimation of consensus profiles for any region on the genome

\section*{Uniqueness of Fluorescence Intensity Profiles}
Discuss the statistics of establishing uniqueness of consensus signals

\section*{Application: Detecting Insertion}

\section*{Application: Detecting Heterozygosity}

\section*{Conclusion}

\newpage
\bibliography{bibTex_Reference}
%\bibliography{research}

\end{document}
