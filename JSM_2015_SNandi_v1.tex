\documentclass[final]{beamer}
\mode<presentation>
{
% you can chose your theme here:
\usetheme{confposter}
% further beamerposter themes are available at
% http://www-i6.informatik.rwth-aachen.de/~dreuw/latexbeamerposter.php
}

%%%%%%%%%%%%%%%%%%%%%%%%%%%%%%%%%%%%%%%%%%%%%%%%%%%%%%%%%%%%%%%%%%%%%%
%% Input header file 
%%%%%%%%%%%%%%%%%%%%%%%%%%%%%%%%%%%%%%%%%%%%%%%%%%%%%%%%%%%%%%%%%%%%%%
%%%%%%%%%%%%%%%%%%%%%%%%%%%%%%%%%%%%%%%%%%%%%%%%%%%%%%%%%%
%% These packages are from Sunyoun, Rene and Nathaniel's posters
%% Will organize them after my JSM poster is finalized

\usepackage[orientation=landscape,scale=1.2,debug]{beamerposter}

%\usetheme{LLT-poster}
%\usecolortheme{ComingClean}
%\usecolortheme{ConspiciousCreep}
\usecolortheme{Entrepreneur}
\usepackage{xkeyval}

\usepackage{calc}
\usepackage{epstopdf}
\usepackage{times}
\usepackage{array, color, xcolor}
\usepackage[english]{babel}
\usepackage[latin1]{inputenc}
\usepackage{graphics, graphicx, tikz}
\usepackage{subfigure}
\usepackage[labelformat=empty]{caption} %no figure in caption
\usepackage[round]{natbib}
\usepackage{fancyvrb, listings, verbatim} %verbatim color, verbatim alternatives
\usepackage{type1cm}
\usepackage{exscale}

\newcommand{\dis}{\displaystyle}

\newcommand{\bx}{\mathbbm{x}}
%\newcommand{\R}{\mathbb{R}}

%%%%%%%%%%%%%%%%%%%%%%%% Packages %%%%%%%%%%%%%%%%%%%%%%%%
%\usepackage[export]{adjustbox}
\usepackage{amscd}
\usepackage{amsmath}
\usepackage{amssymb}
\usepackage{amsthm}
\usepackage{amsxtra}
\usepackage{animate}
\usepackage{bbm}
\usepackage{bbold}
%\usepackage{bigints}
\usepackage{color, colortbl}
\usepackage{dsfont}
\usepackage{enumerate}
\usepackage[mathscr]{eucal}
%\usepackage{fancyhdr}
\usepackage{float}
%\usepackage{fullpage} %% Dont use this for beamer presentations
\usepackage{geometry}
\usepackage{graphicx}
\usepackage{hyperref}
\usepackage{indentfirst}
\usepackage{latexsym}
\usepackage{listings}
\usepackage{lscape}
\usepackage{mathtools}
\usepackage{microtype}
\usepackage{multirow}
\usepackage{natbib}
\usepackage{pdfpages}
\usepackage{verbatim}
\usepackage{wrapfig}
\usepackage{xargs}
\usepackage{xcolor}
\DeclareGraphicsExtensions{.pdf,.png,.jpg, .jpeg}
\definecolor{LightCyan}{rgb}{0.88,1,1}

%%%%%%%%%%%%%%%%%%%%%%%% Commands %%%%%%%%%%%%%%%%%%%%%%%%
\newcommand{\Sup}{\textsuperscript}
\newcommand{\Exp}{\mathds{E}}
\newcommand{\Prob}{\mathds{P}}
\newcommand{\Z}{\mathds{Z}}
\newcommand{\Ind}{\mathds{1}}
\newcommand{\A}{\mathcal{A}}
\newcommand{\F}{\mathcal{F}}
\newcommand{\G}{\mathcal{G}}
\newcommand{\I}{\mathcal{I}}
\newcommand{\R}{\mathcal{R}}
\newcommand{\Real}{\mathbb{R}}
\newcommand{\be}{\begin{equation}}
\newcommand{\ee}{\end{equation}}
\newcommand{\bes}{\begin{equation*}}
\newcommand{\ees}{\end{equation*}}
\newcommand{\beq}{\begin{eqnarray*}}
\newcommand{\eeq}{\end{eqnarray*}}
\newcommand{\bi}{\begin{itemize}}
\newcommand{\ei}{\end{itemize}}
\newcommand{\union}{\bigcup}
\newcommand{\intersect}{\bigcap}
\newcommand{\Ybar}{\overline{Y}}
\newcommand{\ybar}{\bar{y}}
\newcommand{\Xbar}{\overline{X}}
\newcommand{\xbar}{\bar{x}}
\newcommand{\betahat}{\hat{\beta}}
\newcommand{\Yhat}{\widehat{Y}}
\newcommand{\yhat}{\hat{y}}
\newcommand{\Xhat}{\widehat{X}}
\newcommand{\xhat}{\hat{x}}
\newcommand{\E}[1]{\operatorname{E}\left[ #1 \right]}
%\newcommand{\Var}[1]{\operatorname{Var}\left( #1 \right)}
\newcommand{\Var}{\operatorname{Var}}
\newcommand{\Cov}[2]{\operatorname{Cov}\left( #1,#2 \right)}
\newcommand{\N}[2][1=\mu, 2=\sigma^2]{\operatorname{N}\left( #1,#2 \right)}
\newcommand{\bp}[1]{\left( #1 \right)}
\newcommand{\bsb}[1]{\left[ #1 \right]}
\newcommand{\bcb}[1]{\left\{ #1 \right\}}
\newcommand*{\permcomb}[4][0mu]{{{}^{#3}\mkern#1#2_{#4}}}
\newcommand*{\perm}[1][-3mu]{\permcomb[#1]{P}}
\newcommand*{\comb}[1][-1mu]{\permcomb[#1]{C}}



%-----------------------------------------------------------
% Define the column width and poster size
% To set effective sepwid, onecolwid and twocolwid values, first choose how many columns you want and 
% how much separation you want between columns
% The separation I chose is 0.024 and I want 4 columns
% Then set onecolwid to be (1-(3+1)*0.024)/3 = 0.3
% Set twocolwid to be 2*onecolwid + sepwid = 0.624
%-----------------------------------------------------------

\newlength{\sepwid}
\newlength{\onecolwid}
\newlength{\twocolwid}
\setlength{\paperwidth}{48in}
\setlength{\paperheight}{36in}
\setlength{\sepwid}{0.024\paperwidth}
\setlength{\onecolwid}{0.3\paperwidth}
\setlength{\twocolwid}{0.624\paperwidth}
\setlength{\topmargin}{-0.5in}

%-----------------------------------------------------------
% Name and authors of poster/paper/research ------------\usetheme{confposter}
%-----------------------------------------------------------
\titlegraphic{\includegraphics[width=\textwidth,height=.5\textheight]{UWlogo_ctr_4c.jpg}}
%% \title{\parbox[l]{40in}{Iterated curve registration for noisy functional data with phase variability} 
%% \parbox[r]{4in}{\includegraphics[width=.12\textwidth]{UWlogo_ctr_4c.jpg}}}
\title{Iterated, penalized registration for noisy curves with phase variability}
\author{Subhrangshu Nandi, Michael Newton}
\institute{Department of Statistics, Department of Biostatistics and Medical Informatics, University of Wisconsin, Madison \\snandi@wisc.edu }

\begin{document}
%\bibliographystyle{plain}  %Choose a bibliograhpic style
\begin{frame}{}
  \begin{columns}[t]
    %%%%%%%%%%%%%%%%%%%%%%%%%%%%%%%%%%%%%%%%%%%%%%%%%%%%%%%%%%
    %% First Column
    %%%%%%%%%%%%%%%%%%%%%%%%%%%%%%%%%%%%%%%%%%%%%%%%%%%%%%%%%%
    \begin{column}{\onecolwid}\vspace{-2in}
      \begin{alertblock}{Goal}
        To align features of noisy functional data with phase variability, and discern multiple clusters from a sample of curves
      \end{alertblock}
      \begin{block}{Some real data examples}
        In functional data analysis, where both amplitude and phase variability affect individual trajectories, it is often desirable to eliminate (or reduce) the phase variability in order to study a cross-sectional mean function. %\cite{Ramsay_Silverman_2002_Applied} 
        An example of growth data with phase variability is shown below:
        \begin{figure}
          \includegraphics[scale=0.6, page=40]{Regist_Male_fda_ggplot_2015-08-06.pdf}
          \raisebox{1.8in}{\includegraphics[scale=0.2]{Arrow.jpg}}
          \includegraphics[scale=0.6, page=41]{Regist_Male_fda_ggplot_2015-08-06.pdf}
        \end{figure}
        \begin{columns}
          \begin{column}{0.35\textwidth}
            \begin{figure}
              \centering
              \includegraphics[scale=0.6, page=3]{Set1andSet2.pdf}
            \end{figure}
          \end{column}
          \begin{column}{0.65\textwidth}
            But, what if the data has more noise, almost of similar magnitude of that of the features of the curves? \\
            Or, what if the sample curves are noisy realizations of multiple templates? \\
            Could we {\emph{simultaneously}} identify the templates and cluster the curves?
          \end{column}
        \end{columns}
    \end{block}
      \begin{block}{Existing registration methods}
        \begin{itemize}
          \item {\bf{Minimum Second Eigen value method}} - Proposed by Ramsay \& Li \\
            Let $x(t)$ be the true curve; $y(t)$ be the curve to register; Let $h(t)$ be the warping function; $y[h(t)]$ be the registered curve; \\
          If $y[h(t)]$ and and $x(t)$ differ only in terms of amplitude, then, PCA of the following order two matrix should reveal essentially one component (smallest eigen value $\approx 0$). Choose $h(t)$ that minimizes the eigen value of $C(h)$, where
          \begin{columns}
            \begin{column}{\sepwid}\end{column}			% empty spacer column
            \begin{column}{0.55\textwidth}
              $C(h) = 
              \begin{bmatrix}
                \int \{x(t) \}^2dt & \int x(t) y[h(t)]dt \vspace{0.5cm} \\ 
                \int x(t) y[h(t)]dt & \int \{y[h(t)]\}^2dt
              \end{bmatrix}$
            \end{column}
            \begin{column}{0.45\textwidth}
              The warping function is $h(t) = C_0 + C_1\int\limits_0^t w(t)dt$, the \\solution to $w(t) = \frac{h''(t)}{h'(t)}$
            \end{column}
          \end{columns}
          \item {\bf{k-means clustering}} - Proposed by Sangalli, et. al\\
            Jointly clusters and aligns curves; Only enables affine transformations of the abscissa; Uses distance metric: $ \rho(f_i, f_j) = \frac{\int _{S_{ij}}f'_i(s)f'_j(s) ds}{\int _{S_{ij}}f'_i(s)^2 ds \int _{S_{ij}}f'_j(s)^2 ds} $, where, $S_{ij} = S_i \cap S_j$, the intersection of Sobolev spaces formed by the functions $f_i, f_j$. Properties of $\rho(f_i, f_j)$: (1) $|\rho(f_i, f_j)| \leq 1$; (2) $|\rho(f_i, f_j)| = 1 \Leftrightarrow \exists \ \ A \in \Real ^+, B \in \Real, \ni f_i = Af_j + B$; (3) $\rho(f_i, f_j) = \text{sign}(A_i\cdot A_j) \rho(A_if_i + B_i, A_jf_j + B_j)$
          \item {\bf{Using Fisher-Rao Metric}} - Proposed by Srivastava, et. al. \\
            For any $f \in \F$, where $\F$ is the space of absolutely continuous functions on $[0,1]$, and $v_1, v_2 \in T_f(\F)$, where $T_f(\F)$ is the tangent space to $\F$ at $f$, the F.R. metric is defined as $$\langle \langle v_1, v_2 \rangle \rangle_f = \frac{1}{4}\int_0^1\dot{v_1}(t) \dot{v_2}(t)\frac{1}{|\dot{f}(t)|}dt$$
            Robust, but warping function difficult to interpret; Not suitable for samples that are mixtures of realizations from multiple parent templates.
          \item {\bf{Bayesian hierarchical joint model for registration and clustering}} - Proposed by Zhang \& Telesca - Integrates reproducing representations of functions in the framework on Dirichlet process mixtures.          
        \end{itemize}
      \end{block}
    \end{column}

    %%%%%%%%%%%%%%%%%%%%%%%%%%%%%%%%%%%%%%%%%%%%%%%%%%%%%%%%%%
    %% Second Column
    %%%%%%%%%%%%%%%%%%%%%%%%%%%%%%%%%%%%%%%%%%%%%%%%%%%%%%%%%%
    \begin{column}{\onecolwid}\vspace{-2in}
      \begin{block}{Iterated, penalized registration scheme}
        \begin{itemize}
        \item Warping function, $h(t)$ is a strictly {\emph{monotone increasing}} function; Hence {\emph{invertible}}.
        \item $w(t) = \frac{h''(t)}{h'(t)}$ is the relative acceleration of the warping function. 
        \item The objective function is:
          \[ F_{\lambda}(y,x|h) = \int \|y(t) - x\{h(t)\} \|^2 dt + \lambda \int w^2(t) dt \]
          Penalizing $w(t)$ ensures smoothness and monotonicity
        \item Express $w(t)$ in terms of B-Spline expansion: 
          $w(t) = \sum \limits_{k=0}^{K} c_k B_k(t)$
        \end{itemize}
      \end{block}
    \end{column}

    %%%%%%%%%%%%%%%%%%%%%%%%%%%%%%%%%%%%%%%%%%%%%%%%%%%%%%%%%%
    %% Third Column
    %%%%%%%%%%%%%%%%%%%%%%%%%%%%%%%%%%%%%%%%%%%%%%%%%%%%%%%%%%
    \begin{column}{\onecolwid}\vspace{-2in}
      \begin{block}{Ongoing and future work}
        \vspace{12in}
      \end{block}
      \begin{block}{References}
        %% {\footnotesize
        %%   \bibliographystyle{plain} 
        %%   \bibliography{bibTex_Reference}
        %% }
      \end{block}
      \begin{block}{Acknowledgements}
      \end{block}
      \begin{figure}[!b]
        \includegraphics[scale=0.3]{UWlogo_ctr_4c.jpg}
      \end{figure}
    \end{column}
  \end{columns}
\end{frame}
\end{document}
