\documentclass[final]{beamer}
\mode<presentation>
{
% you can chose your theme here:
\usetheme{confposter}
% further beamerposter themes are available at
% http://www-i6.informatik.rwth-aachen.de/~dreuw/latexbeamerposter.php
}

%%%%%%%%%%%%%%%%%%%%%%%%%%%%%%%%%%%%%%%%%%%%%%%%%%%%%%%%%%%%%%%%%%%%%%
%% Input header file 
%%%%%%%%%%%%%%%%%%%%%%%%%%%%%%%%%%%%%%%%%%%%%%%%%%%%%%%%%%%%%%%%%%%%%%
%%%%%%%%%%%%%%%%%%%%%%%%%%%%%%%%%%%%%%%%%%%%%%%%%%%%%%%%%%
%% These packages are from Sunyoun, Rene and Nathaniel's posters
%% Will organize them after my JSM poster is finalized

\usepackage[orientation=landscape,scale=1.2,debug]{beamerposter}

%\usetheme{LLT-poster}
%\usecolortheme{ComingClean}
%\usecolortheme{ConspiciousCreep}
\usecolortheme{Entrepreneur}


\usepackage{calc}
\usepackage{epstopdf}
\usepackage{times}
\usepackage{array, color, xcolor}
\usepackage[english]{babel}
\usepackage[latin1]{inputenc}
\usepackage{graphics, graphicx, tikz}
\usepackage{subfigure}
\usepackage[labelformat=empty]{caption} %no figure in caption
\usepackage[round]{natbib}
\usepackage{fancyvrb, listings, verbatim} %verbatim color, verbatim alternatives
\usepackage{type1cm}
\usepackage{exscale}

\newcommand{\dis}{\displaystyle}

\newcommand{\bx}{\mathbbm{x}}
%\newcommand{\R}{\mathbb{R}}

%%%%%%%%%%%%%%%%%%%%%%%% Packages %%%%%%%%%%%%%%%%%%%%%%%%
%\usepackage[export]{adjustbox}
\usepackage{amscd}
\usepackage{amsmath}
\usepackage{amssymb}
\usepackage{amsthm}
\usepackage{amsxtra}
\usepackage{animate}
\usepackage{bbm}
\usepackage{bbold}
%\usepackage{bigints}
\usepackage{color, colortbl}
\usepackage{dsfont}
\usepackage{enumerate}
\usepackage[mathscr]{eucal}
%\usepackage{fancyhdr}
\usepackage{float}
%\usepackage{fullpage} %% Dont use this for beamer presentations
\usepackage{geometry}
\usepackage{graphicx}
\usepackage{hyperref}
\usepackage{indentfirst}
\usepackage{latexsym}
\usepackage{listings}
\usepackage{lscape}
\usepackage{mathtools}
\usepackage{microtype}
\usepackage{multirow}
\usepackage{natbib}
\usepackage{pdfpages}
\usepackage{verbatim}
\usepackage{wrapfig}
\usepackage{xargs}
\usepackage{xcolor}
\DeclareGraphicsExtensions{.pdf,.png,.jpg, .jpeg}
\definecolor{LightCyan}{rgb}{0.88,1,1}

%%%%%%%%%%%%%%%%%%%%%%%% Commands %%%%%%%%%%%%%%%%%%%%%%%%
\newcommand{\Sup}{\textsuperscript}
\newcommand{\Exp}{\mathds{E}}
\newcommand{\Prob}{\mathds{P}}
\newcommand{\Z}{\mathds{Z}}
\newcommand{\Ind}{\mathds{1}}
\newcommand{\A}{\mathcal{A}}
\newcommand{\F}{\mathcal{F}}
\newcommand{\G}{\mathcal{G}}
\newcommand{\I}{\mathcal{I}}
\newcommand{\R}{\mathcal{R}}
\newcommand{\Real}{\mathbb{R}}
\newcommand{\be}{\begin{equation}}
\newcommand{\ee}{\end{equation}}
\newcommand{\bes}{\begin{equation*}}
\newcommand{\ees}{\end{equation*}}
\newcommand{\beq}{\begin{eqnarray*}}
\newcommand{\eeq}{\end{eqnarray*}}
\newcommand{\bi}{\begin{itemize}}
\newcommand{\ei}{\end{itemize}}
\newcommand{\union}{\bigcup}
\newcommand{\intersect}{\bigcap}
\newcommand{\Ybar}{\overline{Y}}
\newcommand{\ybar}{\bar{y}}
\newcommand{\Xbar}{\overline{X}}
\newcommand{\xbar}{\bar{x}}
\newcommand{\betahat}{\hat{\beta}}
\newcommand{\Yhat}{\widehat{Y}}
\newcommand{\yhat}{\hat{y}}
\newcommand{\Xhat}{\widehat{X}}
\newcommand{\xhat}{\hat{x}}
\newcommand{\E}[1]{\operatorname{E}\left[ #1 \right]}
%\newcommand{\Var}[1]{\operatorname{Var}\left( #1 \right)}
\newcommand{\Var}{\operatorname{Var}}
\newcommand{\Cov}[2]{\operatorname{Cov}\left( #1,#2 \right)}
\newcommand{\N}[2][1=\mu, 2=\sigma^2]{\operatorname{N}\left( #1,#2 \right)}
\newcommand{\bp}[1]{\left( #1 \right)}
\newcommand{\bsb}[1]{\left[ #1 \right]}
\newcommand{\bcb}[1]{\left\{ #1 \right\}}
\newcommand*{\permcomb}[4][0mu]{{{}^{#3}\mkern#1#2_{#4}}}
\newcommand*{\perm}[1][-3mu]{\permcomb[#1]{P}}
\newcommand*{\comb}[1][-1mu]{\permcomb[#1]{C}}


\lstdefinestyle{base}{
  language=C,
  emptylines=1,
  breaklines=true,
  basicstyle=\ttfamily\color{black},
  moredelim=**[is][\color{red}]{@}{@},
}
%-----------------------------------------------------------
% Define the column width and poster size
% To set effective sepwid, onecolwid and twocolwid values, first choose how many columns you want and 
% how much separation you want between columns
% The separation I chose is 0.024 and I want 4 columns
% Then set onecolwid to be (1-(3+1)*0.024)/3 = 0.3
% Set twocolwid to be 2*onecolwid + sepwid = 0.624
%-----------------------------------------------------------

\newlength{\sepwid}
\newlength{\onecolwid}
\newlength{\twocolwid}
\setlength{\paperwidth}{48in}
\setlength{\paperheight}{36in}
\setlength{\sepwid}{0.024\paperwidth}
\setlength{\onecolwid}{0.3\paperwidth}
\setlength{\twocolwid}{0.624\paperwidth}
\setlength{\topmargin}{-0.5in}

%-----------------------------------------------------------
% Define colours (see beamerthemeconfposter.sty to change these colour definitions)
%-----------------------------------------------------------
\setbeamercolor{block title}{fg=white,bg=red}
\setbeamercolor{block body}{fg=black,bg=white}
\setbeamercolor{block alerted title}{fg=white,bg=red}
\setbeamercolor{block alerted body}{fg=black,bg=red}

\definecolor{darkgreen}{rgb}{0.05,0.45,0.15}
\lstset{escapeinside={<@}{@>}} %for coloring lstlisting

%-----------------------------------------------------------
% Name and authors of poster/paper/research ------------\usetheme{confposter}
%-----------------------------------------------------------
%\titlegraphic{\includegraphics[width=\textwidth,height=.5\textheight]{OldWell0}}
\title{\parbox[c]{40in}{Application of iterated curve registration to single molecule genomics} \parbox[c]{4in}{\includegraphics[width=.08\textwidth]{UWlogo_ctr_4c.jpg}}}
\author{Subhrangshu Nandi, Michael Newton, David Schwartz}
\institute{Department of Statistics, Department of Biostatistics and Medical Informatics, University of Wisconsin Madison \vspace{0.5cm} \\snandi@wisc.edu }
\begin{document}
\begin{frame}{}
%\vfill
\begin{columns}[t]
%\hfill
\begin{column}{1\linewidth}
\begin{columns}     

%\hfill
%\begin{column}{\linewidth} %.35
\begin{column}{.38\textwidth}

\begin{block}{ARoG: Integration of GWAS and Functional Genomic Data} %p.3
\begin{columns}
\begin{column}{.6\textwidth}    
\begin{center}
\begin{figure}
\includegraphics[width=0.3\textwidth]{RibosomalDNA_2015-03-19.pdf}
\caption{{\footnotesize \copyright Paltra et al., Front. Genet., 2012}}
\end{figure}
\end{center}
\end{column}
\begin{column}{.36\textwidth}
\bi
\item SNPs can affect disease risk through transcription factors (TF) regulation.
\ei
\end{column}
\end{columns}

\begin{columns}
 \begin{column}{.5\textwidth}                            
%\begin{center}
\begin{figure}
\includegraphics[width=0.4\textwidth]{RibosomalDNA_2015-03-19.pdf}
\caption{{\footnotesize GWAS\\ \copyright Pasieka, Science Photo Library}}
\end{figure}
%\end{center}
\end{column}
\begin{column}{.4\textwidth}
%\begin{center}
\begin{figure}
\includegraphics[width=0.4\textwidth]{RibosomalDNA_2015-03-19.pdf}
\caption{{\footnotesize SNPs can disrupt TF binding. \\ \textit{atSNP} (Zuo et al., 2015)}}
\end{figure}
%\end{center}
\end{column}
\end{columns}

\bi
\item GWAS provide associations measures of SNPs and atSNP calculates TF binding change scores for SNPs.
\ei

\end{block}

\begin{block}{Psychiatric Genomics Consortium (PGC) Data} %p.3
{\color{orange!100} \textbf{Mega-analyses of genome-wide genetic data for psychiatric disorders}} %p.4
\centering
\bi
\item Attention deficit disorder, autism, bipolar disorder, major depressive disorder, schizophrenia (Smoller et al, 2013)
\begin{table}
{\small
 \begin{tabular}{l|rr|rr}
 \hline
 &  \multicolumn{2}{c|}{No. of samples} & \multicolumn{2}{c}{No. of SNPs}\\
 \hline
 &Cases &  Controls & Total & {\color{red} Genome-wide significance*}\\
 \hline
SCZ & 9379 & 7736& 1,237,958&9\\
 \hline
\end{tabular}
}
\end{table}
\ei
 
 {\color{orange!100} \textbf{ {\color{orange!100} \textbf{Input data for ARoG}}}}
\centering
\bi
\item Preliminary SNPs: 1,219,805 SNPs from intersection of PGC data of the five disorders
\item SNPs of interest: 1447 SNPs with Benjamini-Hochberg (BH) adjusted p-values $<$ 0.1 for at least one disorder
\begin{table}
\centering
{\small
 \begin{tabular}{rrrrrr}
 \hline
 ADD &  AUT & BIP & MDD &SCZ & Total \\
 \hline
 0 & 10& 77* & 0 &1367* & 1447\\
 \hline
\end{tabular}
\caption{No. of SNPs for each disease. Seven SNPs are duplicated in both BIP and SCZ.}
}
\end{table}
\item JASPAR motifs: 205 motifs from JASPAR core libraries (Mathelier et al., 2014)
\ei

%\begin{figure}[p]
%  \centering
%  \includegraphics[width=0.5\linewidth]{BoxplotsAdjustedPvalsAll.png}
% %\caption{\textcopyright Pasieka, Science Photo Library}
% \end{figure}
 

\begin{columns}
\begin{column}{.34\textwidth}
{\footnotesize
\begin{table}
  \caption{{\small $Y_{n \times 1}$: GWAS-association measure based response}}
 \begin{tabular}{l|r}
      SNPs & log ORs \\ \hline
       \textit{rs4948418} &\textcolor{red}{0.3365}\\
       $\vdots$ & \vdots\\
  \end{tabular}
\end{table}
}
\end{column}

                  \begin{column}{.6\textwidth}   
                  {\footnotesize
                   \begin{table}
                     \caption{{\small $X_{n \times p}$: predictors based on binding affinity score changes.}}
 \begin{tabular}{l|rr}
              & TFs &    \\ \hline
      SNPs & MYBL2 &  $\cdots$  \\ \hline
      \textit{rs4948418} & $\log{\textcolor{red}{0.004}}-\log{\textcolor{red}{0.607}}$ & $\cdots$ \\
       $\vdots$ & $\vdots$ & $\ddots$ \\
  \end{tabular}
\end{table}
}
\end{column}

\end{columns}
%\begin{verbatim}
%motif                snpid     log_lik_ref log_lik_snp log_lik_ratio  
%MYBL2_jolma_DBD_M183 rs4948418 -9.967      -19.722     9.754
%ref_str snp_str pval_ref pval_snp pval_diff pval_rank
%-       -      \textcolor{red}{ 0.004    0.607}    0.004     0.001
%\end{verbatim}
%      \bibliographystyle{plainnat}
%\def\newblock{}
%\item Sklar, P., et al. "Large-scale genome-wide association analysis of bipolar disorder identifies a new susceptibility locus near ODZ4." \textit{Nature genetics} 43.10 (2011): 977.

{\textbf{References}}
\centering
\bi
{\footnotesize
%\item Mailman, M.D., et al. The NCBI dbGaP database of genotypes and phenotypes. \textit{Nature genetics} 39.10 (2007): 1181-1186.
\item Mathelier, A., et al. JASPAR 2014: an extensively expanded and updated open-access database of transcription factor binding profiles. \textit{Nucleic Acids Research}, 42(D1) (2014): D142-D147.
\item Smoller, J.W., et al. Identification of risk loci with shared effects on five major psychiatric disorders: a genome-wide analysis. \textit{The Lancet} 381.9875 (2013): 1371-1379.}
\ei



\end{block}
 \end{column}

            \begin{column}{.26\textwidth}
        \begin{block}{Sparse Mixture ARoG}
                                  {\color{orange!100} \textbf{Heterogeneity and sparsity of gene regulation}}
			\centering
%\vspace{-0.65cm}
\begin{center}
      \begin{columns}
            \begin{column}{.47\textwidth}
  \end{column}
            \begin{column}{.41\textwidth}
    \end{column}
\end{columns}
\end{center}

{\color{orange!100} \textbf{Aim of ARoG}}
\begin{center}

\bi
\item[(i)] improvement of detection power of significant SNPs by utilizing both the GWAS association measures and information through functional data;
\item[(ii)] elucidation of the functional information correlated with the association measures.
\ei
\end{center}

{\color{orange!100} \textbf{Finite mixture of Gaussian regression models}}
\centering
\bi
\item $n$ SNPs are partitioned into $K$ clusters.
\ei
{\small
\begin{align*}
&Y_i|X_i=x \sim f_{\xi}(y|x) \text{ independent for } i=1,\cdots, n\\
&f_{\xi}(y|\bx)=\dis\sum_{k=1}^K\pi_k\dis\frac{1}{\sqrt{2\pi}\sigma_k}\mathrm{exp}\Bigg(-\dis\frac{(y-\bx^T\beta_k)^2}{2\sigma^2_k}\Bigg),   ~y \in \R, \bx\in \R^p\\%, \beta_k\in\R^p\\
&\xi=(\beta_1,\cdots, \beta_K, \sigma_1,\cdots, \sigma_K, \pi_1,\cdots,\pi_{K-1})\in \R^{Kp}\times \R^K_{>0}\times \Pi,\\
&\Pi=\{\pi: \pi_k>0 \text{ for } k=1,\cdots, K-1, \dis\sum_{k=1}^{K-1}\pi_k<1\}
\end{align*}
}

{\color{orange!100} \textbf{Lasso to mixture of regressions (St\"{a}dler et al., 2010)}}
\bi
\item Reparameterization: {\small
$\phi_k=\beta_k/\sigma_k,~\rho_k=\sigma^{-1}_k,~k=1, \cdots, K$
}
\item Parameters: {\small
$\theta=(\pi_1, \cdots, \pi_K, \rho_1, \cdots, \rho_K, \phi_1, \cdots, \phi_{K-1})$
}
\item FMRLasso: penalized negative log-likelihood
{\small
\begin{align*}
\hat{\theta}_{\lambda}&=\underset{\theta \in \Theta}{\mathrm{argmin}}-n^{-1}l_{pen,\lambda}(\theta),~\Theta=\R^{Kp}\times \R^K_{>0}\times \Pi\\
-l_{pen,\lambda}(\theta)&=-\dis\sum_{i=1}^n\mathrm{log}(\dis\sum_{k=1}^K\phi_k\dis\frac{\rho_k}{\sqrt{2\pi}}\mathrm{exp}(-\dis\frac{1}{2}(\rho_kY_i-X_i^T\phi_k)^2\\
&+n\lambda\dis\sum_{k=1}^{K}\pi_k||\phi_k||_1\\
\end{align*}
}
\ei

{\color{orange!100} \textbf{Another clusterwise Lasso}}
\bi
\item Issue of FMRLasso: there are still too many false positive TFs.
\item Proposal: add hard clustering-based Lasso.
\bi
\item Consider a model for each cluster with corresponding SNPs and selected TFs.
\item Perform another level of Lasso to each model if available.
%or refitting after Lasso model selection
%or just refitting only if $p<n$ and refitting works, i.e. no singularity nor ill-posed problems.
\ei
\ei
%{\color{orange!100} \textbf{Optimal number of components and tuning parameter selection}}
%\bi
%\item Motified BIC criterion: {\small
%$-2l(\hat{\theta}_{\lambda,k})+\mathrm{log}(n)d_{\mathrm{e}},$
%$d_{\mathrm{e}}$: effective no. of components
%}
%%d_{\mathrm{e}}&=k+(k-1)+\dis\sum_{j=1,\cdots, p; r=1, \cdots, k}1_{\{\hat{\pi}_{r, j}\neq 0\}}
%%\end{align*}
%\item Cross-validation: minimizing cross-validated negative log-likelihood
%\ei
{\textbf{Acknowledgements}}\\
\centering       
{\footnotesize
\bi
\item[] This work was supported by National Institutes of Health Grants (HG003747 and HG007019) to S. K.
\ei
}
\end{block}
\end{column}

\begin{column}{.33\textwidth}

\begin{block}{PGC Schizophrenia (SCZ) Data Driven Simulation Studies} %p.5
{\color{orange!100} \textbf{Setting}}                                                 
\centering
\bi
\item Four SNP clusters with a truncated binding affinity score matrix
 \begin{table}
{\small
 \begin{tabular}{l|rrrr}
 \hline
 No. of SNPs & 326 & 74 & 677 & 370\\
 \hline
 Prior prob. & 0.2351 & 0.0963 & 0.4381 & 0.2305\\
 \hline
SD & 0.0222 & 0.1139 & 0.0325 &0.0083\\
 \hline
\end{tabular}
}
\end{table}

\item Regression parameters: 10 times increased from real data analysis results
\ei
{\color{orange!100} \textbf{Results}}  
\begin{figure}[p]
  \centering
  \includegraphics[width=0.2\linewidth]{RibosomalDNA_2015-03-19.pdf}
  \caption{Comparison of our method with ordinary least squares, refit Lasso, FMRLasso}
\end{figure}
\end{block}

\begin{block}{ARoG Analysis for PGC SCZ Data} %p.5
{\color{orange!100} \textbf{Observed vs. fitted log ORs}}                                                 
\centering


{\color{orange!100} \textbf{Effect sizes of selected TFs for each cluster}}
 \begin{table}
{\small
 \begin{tabular}{l|rr|rrrr|rr}
 \hline
Cluster &  \multicolumn{2}{c|}{1} & \multicolumn{4}{c|}{2}& \multicolumn{2}{c}{4}\\
 \hline
TFs &FOXL1 &  Hoxc9 & MAX & BATF::JUN & FLl1 & USF2 & Arnt & ELK4\\
 \hline
Coefficients & -0.0044 & -0.006 & -0.0631 & 0.0621 & 0.036 & -0.0103 & 0.0011 & 0.0014\\
 \hline
\end{tabular}
}
\end{table}

 {\color{orange!100} \textbf{TFs' binding affinity changes in each cluster}}
\begin{columns}
\begin{column}{.32\textwidth}                                                      
\begin{figure}[p]
\centering
 \includegraphics[width=0.7\linewidth]{RibosomalDNA_2015-03-19.pdf}
 \caption{{\small Cluster 1}}
\end{figure}
\end{column}

\begin{column}{.32\textwidth}                                                      
\begin{figure}[p]
  \centering
  \includegraphics[width=0.7\linewidth]{RibosomalDNA_2015-03-19.pdf}
 \caption{{\small Cluster 2}}
\end{figure}
\end{column}

\begin{column}{.32\textwidth}                                                      
\begin{figure}[p]
  \centering
  \includegraphics[width=0.7\linewidth]{RibosomalDNA_2015-03-19.pdf}
 \caption{{\small Cluster 4}}
\end{figure}
\end{column}
\end{columns}
\bi
\item[] Hierarchical clustering of selected TFs and SNPs in each cluster was performed based on pairwise distances of truncated binding affinity scores.
\ei

{\textbf{References}}
\centering
\bi
{\footnotesize
\item St\"{a}dler, N., B\"{u}hlmannl, P., and Van De Geer, S. $l$1-penalization for mixture regression models. \textit{Test} 19.2 (2010): 209-256.
\item Zuo, C., Shin, S., and Kele\c{s}, S. atSNP: Transcription factor binding affinity testing for regulatory SNP detection. \textit{Bioinformatics} (2015): to appear.}
\ei

\end{block}


\end{column}
\end{columns}


         \end{column}

%\end{column}

     \hfill
     \end{columns}
  \end{frame}
\end{document}
