%%%%%%%%%%%%%%%%%%%%%%%%%%%%%%%%%%%%%%%%%%%%%%%%%%%%%%%%%%%%%%%%%%%%%%
%% This is for Stat 992, Fall 2015 presentation on 11/20/2015
%% Parts of this talk are from Documentation/Schwartz_Grp_Mtg/forMeeting_2015-09-28/slides_2015-09-28.tex
%%%%%%%%%%%%%%%%%%%%%%%%%%%%%%%%%%%%%%%%%%%%%%%%%%%%%%%%%%%%%%%%%%%%%%

%\documentclass[10pt,dvipsnames,table, handout]{beamer} % To printout the slides without the animations
\documentclass[10pt,dvipsnames,table, notes]{beamer} 

%\documentclass[notes]{beamer}       % print frame + notes
%\documentclass[notes=only]{beamer}   % only notes
%\documentclass{beamer}              % only frames

%\usetheme{Luebeck} 
\usetheme{Madrid} 
%\usetheme{Marburg} 
%\usetheme{Warsaw} 
%\setbeamercolor{structure}{fg=cyan!90!white}
\setbeamercolor{normal text}{fg=white, bg=black}

%%%%%%%%%%%%%%%%%%%%%%%%%%%%%%%%%%%%%%%%%%%%%%%%%%%%%%%%%%%%%%%%%%%%%%
%% Input header file 
%%%%%%%%%%%%%%%%%%%%%%%%%%%%%%%%%%%%%%%%%%%%%%%%%%%%%%%%%%%%%%%%%%%%%%
%%%%%%%%%%%%%%%%%%%%%%%% Packages %%%%%%%%%%%%%%%%%%%%%%%%
\usepackage{amscd}
\usepackage{amsmath}
\usepackage{amssymb}
\usepackage{amsthm}
\usepackage{amsxtra}
\usepackage{animate}
\usepackage{bbold}
%\usepackage{bigints}
\usepackage{color, colortbl}
\usepackage{dsfont}
\usepackage{enumerate}
\usepackage[mathscr]{eucal}
%\usepackage{fancyhdr}
\usepackage{float}
%\usepackage{fullpage} %% Dont use this for beamer presentations
\usepackage{geometry}
\usepackage{graphicx}
\usepackage{hyperref}
\usepackage{indentfirst}
\usepackage{latexsym}
\usepackage{listings}
\usepackage{lscape}
\usepackage{mathtools}
\usepackage{microtype}
\usepackage{multirow}
\usepackage{natbib}
\usepackage{pdfpages}
\usepackage{verbatim}
\usepackage{wrapfig}
\usepackage{xargs}
\usepackage{xcolor}
\DeclareGraphicsExtensions{.pdf,.png,.jpg, .jpeg}
\definecolor{LightCyan}{rgb}{0.88,1,1}

%%%%%%%%%%%%%%%%%%%%%%%% Commands %%%%%%%%%%%%%%%%%%%%%%%%
\newcommand{\Sup}{\textsuperscript}
\newcommand{\Exp}{\mathds{E}}
\newcommand{\Prob}{\mathds{P}}
\newcommand{\Z}{\mathds{Z}}
\newcommand{\Ind}{\mathds{1}}
\newcommand{\A}{\mathcal{A}}
\newcommand{\F}{\mathcal{F}}
\newcommand{\G}{\mathcal{G}}
\newcommand{\I}{\mathcal{I}}
\newcommand{\R}{\mathcal{R}}
\newcommand{\Real}{\mathbb{R}}
\newcommand{\be}{\begin{equation}}
\newcommand{\ee}{\end{equation}}
\newcommand{\bes}{\begin{equation*}}
\newcommand{\ees}{\end{equation*}}
\newcommand{\union}{\bigcup}
\newcommand{\intersect}{\bigcap}
\newcommand{\Ybar}{\overline{Y}}
\newcommand{\ybar}{\bar{y}}
\newcommand{\Xbar}{\overline{X}}
\newcommand{\xbar}{\bar{x}}
\newcommand{\betahat}{\hat{\beta}}
\newcommand{\Yhat}{\widehat{Y}}
\newcommand{\yhat}{\hat{y}}
\newcommand{\Xhat}{\widehat{X}}
\newcommand{\xhat}{\hat{x}}
\newcommand{\E}[1]{\operatorname{E}\left[ #1 \right]}
%\newcommand{\Var}[1]{\operatorname{Var}\left( #1 \right)}
\newcommand{\Var}{\operatorname{Var}}
\newcommand{\Cov}[2]{\operatorname{Cov}\left( #1,#2 \right)}
\newcommand{\N}[2][1=\mu, 2=\sigma^2]{\operatorname{N}\left( #1,#2 \right)}
\newcommand{\bp}[1]{\left( #1 \right)}
\newcommand{\bsb}[1]{\left[ #1 \right]}
\newcommand{\bcb}[1]{\left\{ #1 \right\}}
\newcommand*{\permcomb}[4][0mu]{{{}^{#3}\mkern#1#2_{#4}}}
\newcommand*{\perm}[1][-3mu]{\permcomb[#1]{P}}
\newcommand*{\comb}[1][-1mu]{\permcomb[#1]{C}}


%%%%%%%%%%%%%%%%%%%%%%%%%%%%%%%%%%%%%%%%%%%%%%%%%%%%%%%%%%%%%%%%%%%%%%
%% TITLE PAGE 
%%%%%%%%%%%%%%%%%%%%%%%%%%%%%%%%%%%%%%%%%%%%%%%%%%%%%%%%%%%%%%%%%%%%%%
\DeclarePairedDelimiter\ceil{\lceil}{\rceil}
\title[Curve Registration in Fluoroscanning]{Applications of Curve Registration to Fluoroscanning\\ {\emph{essential tool for precision genomics}}}
\author[S. Nandi]{Subhrangshu Nandi \\
%\institute[Stat 741]{Stat 741, Spring 2015 \\
Statistics PhD Student \\
\vspace{0.5cm}
\small{in collaboration with:} \\
\textcolor{yellow}{Laboratory of Molecular and Computational Genomics}, \\
University of Wisconsin-Madison \\
Advisors: Prof. Michael Newton, Prof. David Schwartz}
\date{November 20, 2015}

\begin{document}
\setlength{\baselineskip}{16truept}
\frame{\maketitle}

%%%%%%%%%%%%%%%%%%%%%%%%%%%%%%%%%%%%%%%%%%%%%%%%%%%%%%%%%%%%%%%%%%%%%%
%% This is for 2015-09-28 Schwartz group meeting:
%% 1. Introduction to fluoroscanning
%% 2. Some statistical challenges of fluoroscanning
%% 3. Quality score & improvement of consensus
%%%%%%%%%%%%%%%%%%%%%%%%%%%%%%%%%%%%%%%%%%%%%%%%%%%%%%%%%%%%%%%%%%%%%%

%% Outline for this presentation:
%% Motivation
%% Background - Sequencing
%% Background - Nanocoding
%% 


%%%%%%%%%%%% Slide 1 %%%%%%%%%%%%
\begin{frame}
\frametitle{Motivation - road to precision genomics}
\begin{figure}[T]
\includegraphics[scale=0.38]{Image_Motivation.pdf}
\end{figure}

\note{Problem was motivated by a cancer genome (Multiple Myeloma); \\ 
LMCG biophysics, chemistry; \\ Data analysis; \\ with the goal to complement current sequencing efforts, \\ and enhance precision genomics}

%Images from the following websites:
%www.medimoon.com
%www.medcitynews.com
%jamia.oxfordjournals.org
%www.geuvadis.org
%www.sciencenutshell.com
%www.imaging-git.com
\end{frame}

%%%%%%%%%%%% Slide 2 %%%%%%%%%%%%
\begin{frame}
\frametitle{Background: Genome sequencing}
Genome sequencing \footnote{http://www.genomenewsnetwork.org/} is figuring out the order of DNA nucleotides, or bases that make up an organism's DNA. The human genome is made up of over 3 billion of these nucleic acids. A DNA sequence that has been translated from life's {\emph{chemical}} alphabet into our alphabet of written letters might look like this:
\begin{figure}[H]
\includegraphics[scale=0.6]{Image_Sequence_1.jpg}
\end{figure}
\pause
\begin{figure}[H]
\includegraphics[scale=0.4]{Image_DNA.jpg}
\end{figure}

\note{}
\end{frame}

%%%%%%%%%%%% Slide 3 %%%%%%%%%%%%
\begin{frame}
\frametitle{Background: Nanocoding}
\begin{figure}[T]
\includegraphics[scale=0.45]{Image_Nanocoding.jpg}
\end{figure}

\note{
I will try to explain the work that spanned over 10 years, in one slide, so please bear with me.\\

}
\end{frame}

%%%%%%%%%%%% Slide 1 %%%%%%%%%%%%
\begin{frame}
\frametitle{Background}
\begin{figure}[T]
\includegraphics[scale=0.28]{molecule37_frame65.png}
\end{figure}
\end{frame}

%%%%%%%%%%%% Slide 2 %%%%%%%%%%%%
\begin{frame}
\frametitle{Nanocoding - brief background}
\begin{figure}[T]
\includegraphics[scale=0.25]{BarcodeImage.jpg}
\end{figure}

{\emph{M.florum}} reference map: \\
81.6 18.7 59.4 13.9 9.0 5.0 12.3 10.2 15.0 25.4 3.9 20.9 15.6 10.2 9.5 11.1 4.5 
13.7 26.3 38.3 2.1 31.0 19.1 3.6 32.2 41.3 9.8 \textcolor<3>{SpringGreen}{16.4 15.7 6.3 36.5 18.3} 32.1 21.0 
3.3 14.3 51.3 16.0 17.9

\pause
Molecule $2064486\_0\_11$: NtBspQI N  31.24  \textcolor<3>{SpringGreen}{16.80  15.77  6.25  35.08  18.12}

Alignment successful!!
\end{frame}

%%%%%%%%%%%% Slide 3 %%%%%%%%%%%%
\begin{frame}
\frametitle{Nanocoding - brief background}
{\emph{M.florum}} reference map: \\
81.6 18.7 59.4 13.9 9.0 5.0 12.3 10.2 15.0 25.4 3.9 20.9 15.6 10.2 9.5 11.1 4.5 
13.7 26.3 38.3 2.1 31.0 19.1 3.6 32.2 41.3 9.8 \textcolor{SpringGreen}{16.4 15.7 6.3 36.5 18.3} 32.1 21.0 
3.3 14.3 51.3 16.0 17.9

Molecule $2064486\_0\_11$: NtBspQI N  31.24  \textcolor{SpringGreen}{16.80  15.77  6.25  35.08  18.12}

\begin{figure}[T]
\includegraphics[scale=1]{NMaps_Aligned.png}
\end{figure}

\end{frame}

%%%%%%%%%%%% Slide 17 %%%%%%%%%%%%
\begin{frame}
\frametitle{Acknowledgements}
Images in my slides were borrowed from: \\
www.medimoon.com \\
www.medcitynews.com \\
jamia.oxfordjournals.org \\
www.geuvadis.org \\
www.sciencenutshell.com \\
www.imaging-git.com \\

\end{frame}

%%%%%%%%%%%% Slide 17 %%%%%%%%%%%%
\begin{frame}
\frametitle{Questions? Comments?}

\end{frame}

\end{document}

