%\documentclass[10pt,dvipsnames,table, handout]{beamer} % To printout the slides without the animations
\documentclass[10pt,dvipsnames,table]{beamer} 
%\usetheme{Luebeck} 
%\usetheme{Madrid} 
%\usetheme{Marburg} 
%\usetheme{Warsaw} 
%\setbeamercolor{structure}{fg=cyan!90!white}
%\setbeamercolor{normal text}{fg=white, bg=black}
\usetheme{CambridgeUS}
%\setbeamercolor{structure}{fg=cyan!90!white}
%\setbeamercolor{normal text}{fg=white, bg=black}
\setbeamercolor{block title}{bg=red!80,fg=white}

%%%%%%%%%%%%%%%%%%%%%%%%%%%%%%%%%%%%%%%%%%%%%%%%%%%%%%%%%%%%%%%%%%%%%%
%% Input header file 
%%%%%%%%%%%%%%%%%%%%%%%%%%%%%%%%%%%%%%%%%%%%%%%%%%%%%%%%%%%%%%%%%%%%%%
%%%%%%%%%%%%%%%%%%%%%%%% Packages %%%%%%%%%%%%%%%%%%%%%%%%
\usepackage{amscd}
\usepackage{amsmath}
\usepackage{amssymb}
\usepackage{amsthm}
\usepackage{amsxtra}
\usepackage{animate}
\usepackage{bbold}
%\usepackage{bigints}
\usepackage{color, colortbl}
\usepackage{dsfont}
\usepackage{enumerate}
\usepackage[mathscr]{eucal}
%\usepackage{fancyhdr}
\usepackage{float}
%\usepackage{fullpage} %% Dont use this for beamer presentations
\usepackage{geometry}
\usepackage{graphicx}
\usepackage{hyperref}
\usepackage{indentfirst}
\usepackage{latexsym}
\usepackage{listings}
\usepackage{lscape}
\usepackage{mathtools}
\usepackage{microtype}
\usepackage{multirow}
\usepackage{natbib}
\usepackage{pdfpages}
\usepackage{verbatim}
\usepackage{wrapfig}
\usepackage{xargs}
\usepackage{xcolor}
\DeclareGraphicsExtensions{.pdf,.png,.jpg, .jpeg}
\definecolor{LightCyan}{rgb}{0.88,1,1}

%%%%%%%%%%%%%%%%%%%%%%%% Commands %%%%%%%%%%%%%%%%%%%%%%%%
\newcommand{\Sup}{\textsuperscript}
\newcommand{\Exp}{\mathds{E}}
\newcommand{\Prob}{\mathds{P}}
\newcommand{\Z}{\mathds{Z}}
\newcommand{\Ind}{\mathds{1}}
\newcommand{\A}{\mathcal{A}}
\newcommand{\F}{\mathcal{F}}
\newcommand{\G}{\mathcal{G}}
\newcommand{\I}{\mathcal{I}}
\newcommand{\R}{\mathcal{R}}
\newcommand{\Real}{\mathbb{R}}
\newcommand{\be}{\begin{equation}}
\newcommand{\ee}{\end{equation}}
\newcommand{\bes}{\begin{equation*}}
\newcommand{\ees}{\end{equation*}}
\newcommand{\union}{\bigcup}
\newcommand{\intersect}{\bigcap}
\newcommand{\Ybar}{\overline{Y}}
\newcommand{\ybar}{\bar{y}}
\newcommand{\Xbar}{\overline{X}}
\newcommand{\xbar}{\bar{x}}
\newcommand{\betahat}{\hat{\beta}}
\newcommand{\Yhat}{\widehat{Y}}
\newcommand{\yhat}{\hat{y}}
\newcommand{\Xhat}{\widehat{X}}
\newcommand{\xhat}{\hat{x}}
\newcommand{\E}[1]{\operatorname{E}\left[ #1 \right]}
%\newcommand{\Var}[1]{\operatorname{Var}\left( #1 \right)}
\newcommand{\Var}{\operatorname{Var}}
\newcommand{\Cov}[2]{\operatorname{Cov}\left( #1,#2 \right)}
\newcommand{\N}[2][1=\mu, 2=\sigma^2]{\operatorname{N}\left( #1,#2 \right)}
\newcommand{\bp}[1]{\left( #1 \right)}
\newcommand{\bsb}[1]{\left[ #1 \right]}
\newcommand{\bcb}[1]{\left\{ #1 \right\}}
\newcommand*{\permcomb}[4][0mu]{{{}^{#3}\mkern#1#2_{#4}}}
\newcommand*{\perm}[1][-3mu]{\permcomb[#1]{P}}
\newcommand*{\comb}[1][-1mu]{\permcomb[#1]{C}}


%%%%%%%%%%%%%%%%%%%%%%%%%%%%%%%%%%%%%%%%%%%%%%%%%%%%%%%%%%%%%%%%%%%%%%
%% TITLE PAGE 
%%%%%%%%%%%%%%%%%%%%%%%%%%%%%%%%%%%%%%%%%%%%%%%%%%%%%%%%%%%%%%%%%%%%%%
\DeclarePairedDelimiter\ceil{\lceil}{\rceil}
\title[Fluoroscanning]{Fluoroscanning: {\emph{Manuscript outline}}}
\author[S. Nandi]{Subhrangshu Nandi}
%\institute[Stat 741]{Stat 741, Spring 2015 \\
% Department of Statistics \\


% University of Wisconsin-Madison}
\date{December 09, 2016}

\begin{document}
\setlength{\baselineskip}{16truept}
\frame{\maketitle}

\begin{frame}
\frametitle{Stretch vs Signal}
\begin{enumerate}
\item Use {\emph{M. florum}} nanocoded dataset
\item Divide Nmaps into 5 groups, by lengths (number of molecules) of their images
\item Smooth, register and estimate consensus of each group
\item Compare the consensus with GC-profile, using ``similarity index''
\end{enumerate}

\vspace{1cm}

Example: Fragment 22, 30

Example (reproducibility): Fragment 21, 25, 28 

Example (spurious): Fragment 12
\end{frame}

\begin{frame}
\frametitle{Predicting Fscan from sequence features}
\begin{enumerate}
\item Used counts of g, c, a, t, 2-mers, 3-mers and 4-mers to fit consensus Fscans (FCscans)
\item Used Gaussian kernel, to account for effect of 400bp on either side of a pixel in addition to the 200bp in the middle
\item Fit a random forest model
\end{enumerate}
\vspace{2cm}
Some examples of prediction
\end{frame}

\begin{frame}
\frametitle{Prediction example, Interval 970, Chr 19}
\begin{figure}
\includegraphics[scale=0.35, page=2]{SeqVsConsensus_SelectedFew_Chr19.pdf}
\pause
\includegraphics[scale=0.35]{chr19_frag970_0_predicted.pdf}
\end{figure}
\end{frame}

\begin{frame}
\frametitle{Prediction example, Interval 5615, Chr 19}
\begin{figure}
\includegraphics[scale=0.35, page=5]{SeqVsConsensus_SelectedFew_Chr19.pdf}
\pause
\includegraphics[scale=0.35]{chr19_frag5615_2_predicted.pdf}
\end{figure}
\end{frame}

\begin{frame}
\frametitle{Prediction example, Interval 5518, Chr 19}
\begin{figure}
\includegraphics[scale=0.35, page=8]{SeqVsConsensus_SelectedFew_Chr19.pdf}
\pause
\includegraphics[scale=0.35]{chr19_frag5518_0_predicted.pdf}
\end{figure}
\end{frame}

\begin{frame}
\frametitle{Prediction example, Interval 1374, Chr 19}
\begin{figure}
\includegraphics[scale=0.35, page=27]{SeqVsConsensus_SelectedFew_Chr19.pdf}
\pause
\includegraphics[scale=0.35]{chr19_frag1374_1_predicted.pdf}
\end{figure}
\end{frame}

\begin{frame}
\frametitle{Manuscript outline}

\end{frame}


\end{document}

