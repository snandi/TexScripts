%\documentclass[10pt,dvipsnames,table, handout]{beamer} % To printout the slides without the animations
\documentclass[10pt,dvipsnames,table]{beamer} 
%\usetheme{Luebeck} 
%\usetheme{Madrid} 
%\usetheme{Marburg} 
%\usetheme{Warsaw} 
%\setbeamercolor{structure}{fg=cyan!90!white}
%\setbeamercolor{normal text}{fg=white, bg=black}
\usetheme{CambridgeUS}
%\setbeamercolor{structure}{fg=cyan!90!white}
%\setbeamercolor{normal text}{fg=white, bg=black}
\setbeamercolor{block title}{bg=red!80,fg=white}

%%%%%%%%%%%%%%%%%%%%%%%%%%%%%%%%%%%%%%%%%%%%%%%%%%%%%%%%%%%%%%%%%%%%%%
%% Input header file 
%%%%%%%%%%%%%%%%%%%%%%%%%%%%%%%%%%%%%%%%%%%%%%%%%%%%%%%%%%%%%%%%%%%%%%
\input{HeaderfileTexSlides}

%%%%%%%%%%%%%%%%%%%%%%%%%%%%%%%%%%%%%%%%%%%%%%%%%%%%%%%%%%%%%%%%%%%%%%
%% TITLE PAGE 
%%%%%%%%%%%%%%%%%%%%%%%%%%%%%%%%%%%%%%%%%%%%%%%%%%%%%%%%%%%%%%%%%%%%%%
\DeclarePairedDelimiter\ceil{\lceil}{\rceil}
\title[Fluoroscanning]{Fluoroscanning: {\emph{Comparison of FScan Consensus and Genomic Sequences}}}
\author{Subhrangshu Nandi}
%\institute[Stat 741]{Stat 741, Spring 2015 \\
% Department of Statistics \\
% University of Wisconsin-Madison}
\date{October 21, 2016}

\begin{document}
\setlength{\baselineskip}{16truept}
\frame{\maketitle}

%%%%%%%%%%%%%% Slide 1 %%%%%%%%%%%%%%
\begin{frame}
\frametitle{Distance between consensus signals}
Similarity index $\rho(f_i, f_j)$ between two consensus FScan signals $f_i$ and $f_j$:
\[ \rho(f_i, f_j) = \frac{\int _{S_{ij}}f'_i(s)f'_j(s) ds}{\int _{S_{ij}}f'_i(s)^2 ds \int _{S_{ij}}f'_j(s)^2 ds} \]
where, $S_{ij} = S_i \cap S_j$, the intersection of Sobolev spaces formed by the functions $f_i, f_j$.\\
$-1 \leq \rho(f_i, f_j) \leq 1$ \\
Higher values of $\rho$ implies more {\emph{similarity}} between two consensus

\end{frame}

%%%%%%%%%%%%%% Slide 2 %%%%%%%%%%%%%%
\begin{frame}
\frametitle{Distance between sequences of two intervals}
Needleman–Wunsch algorithm for global alignment between two intervals:
\begin{itemize}
\item Substitution matrix: \\
\[ \bordermatrix{~  & A & C & G & T \cr
                  A & 1 & 0 & 0 & 1 \cr
                  C & 0 & 1 & 1 & 0 \cr
                  G & 0 & 1 & 1 & 0 \cr
                  T & 1 & 0 & 0 & 1 \cr
}
\]
\item Gap penalty = -10
\item Slide $\text{Sequence}_1$ 200 bp either side of $\text{Sequence}_2$ and obtain global alignment score
\item Final sequence similarity score: Average of alignment scores
\end{itemize}

\end{frame}

%%%%%%%%%%%%%% Slide 3 %%%%%%%%%%%%%%
\begin{frame}
\frametitle{Example}
\begin{figure}
\begin{centering}
\includegraphics[page=5, scale=0.3]{SeqVsConsensus_SelectedFew_Chr22.pdf}
\includegraphics[page=17, scale=0.3]{SeqVsConsensus_SelectedFew_Chr22.pdf}
\end{centering}
\end{figure}
Consensus similarity: $\rho = 0.5984$ \\
Sequence alignment score: $F = 0.5061$

\end{frame}

%%%%%%%%%%%%%% Slide 4 %%%%%%%%%%%%%%
\begin{frame}
\frametitle{Pairwise comparison}
\begin{itemize}
\item In chromosome 22, there are 403 sub-intervals that are 10.3 kb long, with at least 15 Nmaps depth
\item $ \binom {403} {2} = 81,003 $ pairs to compare. 
\end{itemize}

\end{frame}

\end{document}

