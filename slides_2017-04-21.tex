%\documentclass[10pt,dvipsnames,table, handout]{beamer} % To printout the slides without the animations
\documentclass[10pt,dvipsnames,table]{beamer} 
%\usetheme{Luebeck} 
%\usetheme{Madrid} 
%\usetheme{Marburg} 
%\usetheme{Warsaw} 
%\setbeamercolor{structure}{fg=cyan!90!white}
%\setbeamercolor{normal text}{fg=white, bg=black}
\usetheme{CambridgeUS}
%\setbeamertemplate{enumerate items}[default]
\setbeamercolor{item projected}{fg=green,bg=red}
%\setbeamercolor{itemize subitem}{fg=blue}
%\setbeamercolor{structure}{fg=cyan!90!white}
%\setbeamercolor{normal text}{fg=white, bg=black}
\setbeamercolor{block title}{bg=red!80,fg=white}

%%%%%%%%%%%%%%%%%%%%%%%%%%%%%%%%%%%%%%%%%%%%%%%%%%%%%%%%%%%%%%%%%%%%%%
%% Input header file 
%%%%%%%%%%%%%%%%%%%%%%%%%%%%%%%%%%%%%%%%%%%%%%%%%%%%%%%%%%%%%%%%%%%%%%
%%%%%%%%%%%%%%%%%%%%%%%% Packages %%%%%%%%%%%%%%%%%%%%%%%%
\usepackage{amscd}
\usepackage{amsmath}
\usepackage{amssymb}
\usepackage{amsthm}
\usepackage{amsxtra}
\usepackage{animate}
\usepackage{bbold}
%\usepackage{bigints}
\usepackage{color, colortbl}
\usepackage{dsfont}
\usepackage{enumerate}
\usepackage[mathscr]{eucal}
%\usepackage{fancyhdr}
\usepackage{float}
%\usepackage{fullpage} %% Dont use this for beamer presentations
\usepackage{geometry}
\usepackage{graphicx}
\usepackage{hyperref}
\usepackage{indentfirst}
\usepackage{latexsym}
\usepackage{listings}
\usepackage{lscape}
\usepackage{mathtools}
\usepackage{microtype}
\usepackage{multirow}
\usepackage{natbib}
\usepackage{pdfpages}
\usepackage{verbatim}
\usepackage{wrapfig}
\usepackage{xargs}
\usepackage{xcolor}
\DeclareGraphicsExtensions{.pdf,.png,.jpg, .jpeg}
\definecolor{LightCyan}{rgb}{0.88,1,1}

%%%%%%%%%%%%%%%%%%%%%%%% Commands %%%%%%%%%%%%%%%%%%%%%%%%
\newcommand{\Sup}{\textsuperscript}
\newcommand{\Exp}{\mathds{E}}
\newcommand{\Prob}{\mathds{P}}
\newcommand{\Z}{\mathds{Z}}
\newcommand{\Ind}{\mathds{1}}
\newcommand{\A}{\mathcal{A}}
\newcommand{\F}{\mathcal{F}}
\newcommand{\G}{\mathcal{G}}
\newcommand{\I}{\mathcal{I}}
\newcommand{\R}{\mathcal{R}}
\newcommand{\Real}{\mathbb{R}}
\newcommand{\be}{\begin{equation}}
\newcommand{\ee}{\end{equation}}
\newcommand{\bes}{\begin{equation*}}
\newcommand{\ees}{\end{equation*}}
\newcommand{\union}{\bigcup}
\newcommand{\intersect}{\bigcap}
\newcommand{\Ybar}{\overline{Y}}
\newcommand{\ybar}{\bar{y}}
\newcommand{\Xbar}{\overline{X}}
\newcommand{\xbar}{\bar{x}}
\newcommand{\betahat}{\hat{\beta}}
\newcommand{\Yhat}{\widehat{Y}}
\newcommand{\yhat}{\hat{y}}
\newcommand{\Xhat}{\widehat{X}}
\newcommand{\xhat}{\hat{x}}
\newcommand{\E}[1]{\operatorname{E}\left[ #1 \right]}
%\newcommand{\Var}[1]{\operatorname{Var}\left( #1 \right)}
\newcommand{\Var}{\operatorname{Var}}
\newcommand{\Cov}[2]{\operatorname{Cov}\left( #1,#2 \right)}
\newcommand{\N}[2][1=\mu, 2=\sigma^2]{\operatorname{N}\left( #1,#2 \right)}
\newcommand{\bp}[1]{\left( #1 \right)}
\newcommand{\bsb}[1]{\left[ #1 \right]}
\newcommand{\bcb}[1]{\left\{ #1 \right\}}
\newcommand*{\permcomb}[4][0mu]{{{}^{#3}\mkern#1#2_{#4}}}
\newcommand*{\perm}[1][-3mu]{\permcomb[#1]{P}}
\newcommand*{\comb}[1][-1mu]{\permcomb[#1]{C}}


%%%%%%%%%%%%%%%%%%%%%%%%%%%%%%%%%%%%%%%%%%%%%%%%%%%%%%%%%%%%%%%%%%%%%%
%% TITLE PAGE 
%%%%%%%%%%%%%%%%%%%%%%%%%%%%%%%%%%%%%%%%%%%%%%%%%%%%%%%%%%%%%%%%%%%%%%
\DeclarePairedDelimiter\ceil{\lceil}{\rceil}
\title[Fluoroscanning]{Fluoroscanning: {\emph{Next Generation Precision Genomics}}}
\author[S. Nandi]{Subhrangshu Nandi}
\institute[LMCG]{	Laboratory of Molecular and Computational Genomics \\
 			University of Wisconsin - Madison }
\date{April 21, 2017}

\begin{document}
\setlength{\baselineskip}{16truept}
\frame{\maketitle}

%%%%%%%%%%%%%% Slide x %%%%%%%%%%%%%%
\begin{frame}
\LARGE
\begin{block}{Fluoroscanning}
A system to extract the most necessary and informative components of genomic sequences from fluorescence intensity profiles (Fscans) of imaged DNA molecules. 
\end{block}
\end{frame}
%%%%%%%%%%%%%%%%%%%%%%%%%%%%%%%%%%%%%

%%%%%%%%%%%%%% Slide x %%%%%%%%%%%%%%
\begin{frame}
\includegraphics[scale=0.45]{Images/FluoroscanningWorkflow_v3.pdf}
\end{frame}
%%%%%%%%%%%%%%%%%%%%%%%%%%%%%%%%%%%%%

%%%%%%%%%%%%%% Slide x %%%%%%%%%%%%%%
\begin{frame}
\frametitle{Fluoroscanning: Specific Aims}
\begin{block}{Aim 1}
Fluoroscanning signals (Fscans) of genomic intervals are unique and reproducible
\end{block}
\vspace{0.5in}
\begin{block}{Aim 2}
Predict Fscans from underlying sequence compositions of genomic intervals
\end{block}

\end{frame}
%%%%%%%%%%%%%%%%%%%%%%%%%%%%%%%%%%%%%

%%%%%%%%%%%%%% Slide x %%%%%%%%%%%%%%
\begin{frame}
\frametitle{Uniqueness of cFscans}

\begin{block}{Uniqueness}
How unique are short (4kb, 5kb, etc) cFscan segments across the whole genome?
\end{block}

\end{frame}
%%%%%%%%%%%%%%%%%%%%%%%%%%%%%%%%%%%%%

%%%%%%%%%%%%%% Slide x %%%%%%%%%%%%%%
\begin{frame}
\frametitle{Discretizing cFscans}
\footnotesize
\vspace{-0.5cm}
\begin{figure}
\centering
\includegraphics[width=0.5\textwidth, page=7]{Plots/chr19_Interval2481_registered.pdf}
\end{figure}
\vspace{-0.5cm}
Consider a cFscan $f(x)$, where $x$ denotes pixels.
\begin{itemize}
\item At each pixel $x$, a discretized representation of $f(x)$ is $f(x) > 0, f'(x) > 0, f''(x) > 0$
\item The discretized representation consisted of eight possible states $A: (1,1,1), B:(1,0,1), C:(1,1,0), D:(1,0,0), E:(0,1,1), F:(0,0,1), G:(0,1,0), H:(0,0,0)$
\item 144-letter ``Sequence'': GHHFFFEEGGGGGEEEACCCCDDDBBFFEEGHFEEACCCCCCCDDDDBFF ...EECCCDDDBFEEACCDDBBFHHHHHFFFFEEEEGGCCDDBBHHFFEEEA
\end{itemize}

\end{frame}
%%%%%%%%%%%%%%%%%%%%%%%%%%%%%%%%%%%%%

%%%%%%%%%%%%%% Slide x %%%%%%%%%%%%%%
\begin{frame}
\frametitle{Uniqueness of discretized cFscans}
METHOD
\begin{itemize}
\item Randomly select 1,000 intervals from 21,972 intervals in the dataset
\item Randomly choose 4kb (20 states), 5kb (25 states), ..., 8kb (40 states) and count the empirical frequency in the rest of the interval
\end{itemize}

RESULTS
\begin{table}[H]
\centering
\caption{Uniqueness of cFscans segments} 
\begin{tabular}{c|c|c}
\hline
\hline
Length (states) & Length (kb) 	& Empirical frequency 	\\
\hline
20 		& 4.120		& 5.077 		\\
25		& 5.150		& 0.194			\\
30		& 6.180		& 0.005			\\
35		& 7.210		& 0.000			\\
40		& 8.240		& 0.000			\\
\hline
\hline
\end{tabular}
\end{table}
\end{frame}
%%%%%%%%%%%%%%%%%%%%%%%%%%%%%%%%%%%%%

%%%%%%%%%%%%%%%%%%%%%% SLIDE 9 %%%%%%%%%%%%%%%%%%%%%%
\begin{frame}
\frametitle{Ongoing work}
\begin{itemize}
\item Writing up manuscript
\item Using markov property, derive theoretical bounds for uniqueness lengths
\item Derive error bound for Fscan uniqueness lengths
\end{itemize}
\end{frame}

\end{document}

