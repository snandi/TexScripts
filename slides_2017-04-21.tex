%\documentclass[10pt,dvipsnames,table, handout]{beamer} % To printout the slides without the animations
\documentclass[10pt,dvipsnames,table]{beamer} 
%\usetheme{Luebeck} 
%\usetheme{Madrid} 
%\usetheme{Marburg} 
%\usetheme{Warsaw} 
%\setbeamercolor{structure}{fg=cyan!90!white}
%\setbeamercolor{normal text}{fg=white, bg=black}
\usetheme{CambridgeUS}
%\setbeamertemplate{enumerate items}[default]
\setbeamercolor{item projected}{fg=green,bg=red}
%\setbeamercolor{itemize subitem}{fg=blue}
%\setbeamercolor{structure}{fg=cyan!90!white}
%\setbeamercolor{normal text}{fg=white, bg=black}
\setbeamercolor{block title}{bg=red!80,fg=white}

%%%%%%%%%%%%%%%%%%%%%%%%%%%%%%%%%%%%%%%%%%%%%%%%%%%%%%%%%%%%%%%%%%%%%%
%% Input header file 
%%%%%%%%%%%%%%%%%%%%%%%%%%%%%%%%%%%%%%%%%%%%%%%%%%%%%%%%%%%%%%%%%%%%%%
\input{HeaderfileTexSlides}

%%%%%%%%%%%%%%%%%%%%%%%%%%%%%%%%%%%%%%%%%%%%%%%%%%%%%%%%%%%%%%%%%%%%%%
%% TITLE PAGE 
%%%%%%%%%%%%%%%%%%%%%%%%%%%%%%%%%%%%%%%%%%%%%%%%%%%%%%%%%%%%%%%%%%%%%%
\DeclarePairedDelimiter\ceil{\lceil}{\rceil}
\title[Fluoroscanning]{Fluoroscanning: {\emph{Next Generation Precision Genomics}}}
\author[S. Nandi]{Subhrangshu Nandi}
\institute[LMCG]{	Laboratory of Molecular and Computational Genomics \\
 			University of Wisconsin - Madison }
\date{April 21, 2017}

\begin{document}
\setlength{\baselineskip}{16truept}
\frame{\maketitle}

%%%%%%%%%%%%%% Slide x %%%%%%%%%%%%%%
\begin{frame}
\LARGE
\begin{block}{Fluoroscanning}
A system to extract the most necessary and informative components of genomic sequences from fluorescence intensity profiles (Fscans) of imaged DNA molecules. 
\end{block}
\end{frame}
%%%%%%%%%%%%%%%%%%%%%%%%%%%%%%%%%%%%%

%%%%%%%%%%%%%% Slide x %%%%%%%%%%%%%%
\begin{frame}
\includegraphics[scale=0.45]{Images/FluoroscanningWorkflow_v3.pdf}
\end{frame}
%%%%%%%%%%%%%%%%%%%%%%%%%%%%%%%%%%%%%

%%%%%%%%%%%%%% Slide x %%%%%%%%%%%%%%
\begin{frame}
\frametitle{Fluoroscanning: Specific Aims}
\begin{block}{Aim 1}
Fluoroscanning signals (Fscans) of genomic intervals are unique and reproducible
\end{block}
\vspace{0.5in}
\begin{block}{Aim 2}
Predict Fscans from underlying sequence compositions of genomic intervals
\end{block}

\end{frame}
%%%%%%%%%%%%%%%%%%%%%%%%%%%%%%%%%%%%%

%%%%%%%%%%%%%% Slide x %%%%%%%%%%%%%%
\begin{frame}
\frametitle{Uniqueness of cFscans}

\begin{block}{Uniqueness}
How unique are short (4kb, 5kb, etc) cFscan segments across the whole genome?
\end{block}

\end{frame}
%%%%%%%%%%%%%%%%%%%%%%%%%%%%%%%%%%%%%

%%%%%%%%%%%%%% Slide x %%%%%%%%%%%%%%
\begin{frame}
\frametitle{Discretizing cFscans}


\end{frame}
%%%%%%%%%%%%%%%%%%%%%%%%%%%%%%%%%%%%%

%%%%%%%%%%%%%% Slide x %%%%%%%%%%%%%%
\begin{frame}
\frametitle{Reference}
{\footnotesize{
    \bibliographystyle{apalike}
    \bibliography{bibTex_Reference}
}}
\end{frame}

\end{document}

