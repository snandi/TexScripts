\documentclass[11pt,a4paper]{article}

%%%%%%%%%%%%%%%%%%%%%%%%%%%%%%%%%%%%%%%%%%%%%%%%%%%%%%%%%%%%%%%%%%%%%%
%% Input header file 
%%%%%%%%%%%%%%%%%%%%%%%%%%%%%%%%%%%%%%%%%%%%%%%%%%%%%%%%%%%%%%%%%%%%%%
%%%%%%%%%%%%%%%%%%%%%%%% Packages %%%%%%%%%%%%%%%%%%%%%%%%
\usepackage{amscd}
\usepackage{amsmath}
\usepackage{amssymb}
\usepackage{amsthm}
\usepackage{amsxtra}
\usepackage{animate}
\usepackage{bbold}
%\usepackage{bigints}
\usepackage{color, colortbl}
\usepackage{dsfont}
\usepackage{enumerate}
\usepackage[mathscr]{eucal}
%\usepackage{fancyhdr}
\usepackage{float}
%\usepackage{fullpage} %% Dont use this for beamer presentations
\usepackage{geometry}
\usepackage{graphicx}
\usepackage{hyperref}
\usepackage{indentfirst}
\usepackage{latexsym}
\usepackage{listings}
\usepackage{lscape}
\usepackage{mathtools}
\usepackage{microtype}
\usepackage{multirow}
\usepackage{natbib}
\usepackage{pdfpages}
\usepackage{verbatim}
\usepackage{wrapfig}
\usepackage{xargs}
\usepackage{xcolor}
\DeclareGraphicsExtensions{.pdf,.png,.jpg, .jpeg}
\definecolor{LightCyan}{rgb}{0.88,1,1}

%%%%%%%%%%%%%%%%%%%%%%%% Commands %%%%%%%%%%%%%%%%%%%%%%%%
\newcommand{\Sup}{\textsuperscript}
\newcommand{\Exp}{\mathds{E}}
\newcommand{\Prob}{\mathds{P}}
\newcommand{\Z}{\mathds{Z}}
\newcommand{\Ind}{\mathds{1}}
\newcommand{\A}{\mathcal{A}}
\newcommand{\F}{\mathcal{F}}
\newcommand{\G}{\mathcal{G}}
\newcommand{\I}{\mathcal{I}}
\newcommand{\R}{\mathcal{R}}
\newcommand{\Real}{\mathbb{R}}
\newcommand{\be}{\begin{equation}}
\newcommand{\ee}{\end{equation}}
\newcommand{\bes}{\begin{equation*}}
\newcommand{\ees}{\end{equation*}}
\newcommand{\union}{\bigcup}
\newcommand{\intersect}{\bigcap}
\newcommand{\Ybar}{\overline{Y}}
\newcommand{\ybar}{\bar{y}}
\newcommand{\Xbar}{\overline{X}}
\newcommand{\xbar}{\bar{x}}
\newcommand{\betahat}{\hat{\beta}}
\newcommand{\Yhat}{\widehat{Y}}
\newcommand{\yhat}{\hat{y}}
\newcommand{\Xhat}{\widehat{X}}
\newcommand{\xhat}{\hat{x}}
\newcommand{\E}[1]{\operatorname{E}\left[ #1 \right]}
%\newcommand{\Var}[1]{\operatorname{Var}\left( #1 \right)}
\newcommand{\Var}{\operatorname{Var}}
\newcommand{\Cov}[2]{\operatorname{Cov}\left( #1,#2 \right)}
\newcommand{\N}[2][1=\mu, 2=\sigma^2]{\operatorname{N}\left( #1,#2 \right)}
\newcommand{\bp}[1]{\left( #1 \right)}
\newcommand{\bsb}[1]{\left[ #1 \right]}
\newcommand{\bcb}[1]{\left\{ #1 \right\}}
\newcommand*{\permcomb}[4][0mu]{{{}^{#3}\mkern#1#2_{#4}}}
\newcommand*{\perm}[1][-3mu]{\permcomb[#1]{P}}
\newcommand*{\comb}[1][-1mu]{\permcomb[#1]{C}}


%%%%%%%%%%%%%%%%%%% To change the margins and stuff %%%%%%%%%%%%%%%%%%%
\geometry{left=0.9in, right=0.9in, top=0.9in, bottom=0.8in}
%\setlength{\voffset}{0.5in}
%\setlength{\hoffset}{-0.4in}
%\setlength{\textwidth}{7.6in}
%\setlength{\textheight}{10in}
%%%%%%%%%%%%%%%%%%%%%%%%%%%%%%%%%%%%%%%%%%%%%%%%%%%%%%%%%%%%%%%%%%%%%%%
\begin{document}

\title{Final Exam}
\author{Subhrangshu Nandi\\
  Stat 641; Fall 2015}
\date{December 17, 2015}
%\date{}

\maketitle

\begin{enumerate}
%%%%%%%%%%%%%%%%%%%%%%%%%%%%%%%%%%%%%%%%%%%%%%%%%%%%%%%%%%%%%%%%%%%%%%%
%% Problem 1
%%%%%%%%%%%%%%%%%%%%%%%%%%%%%%%%%%%%%%%%%%%%%%%%%%%%%%%%%%%%%%%%%%%%%%%
\item 
\begin{enumerate}
\item[(a)] 
\item[(b)] 
\item[(c)] 
\item[(d)] 
\end{enumerate}

%%%%%%%%%%%%%%%%%%%%%%%%%%%%%%%%%%%%%%%%%%%%%%%%%%%%%%%%%%%%%%%%%%%%%%%
%% Problem 2
%%%%%%%%%%%%%%%%%%%%%%%%%%%%%%%%%%%%%%%%%%%%%%%%%%%%%%%%%%%%%%%%%%%%%%%
\item
\begin{enumerate}
\item[(a)] 
\item[(b)] 
\item[(c)] 
\end{enumerate}

%%%%%%%%%%%%%%%%%%%%%%%%%%%%%%%%%%%%%%%%%%%%%%%%%%%%%%%%%%%%%%%%%%%%%%%
%% Problem 3
%%%%%%%%%%%%%%%%%%%%%%%%%%%%%%%%%%%%%%%%%%%%%%%%%%%%%%%%%%%%%%%%%%%%%%%
\item
Let $\hat{\beta}_u$ be the unadjusted estimate of $\beta$, and $\hat{\beta}_a$ be the adjusted estimate of $\beta$. Let $\bar{w}_o$ and $\bar{w}_1$ be the means of $w$ in the groups $z = 0$ and $z = 1$ respectively. Then,
\begin{eqnarray*}
\hat{\beta}_a & = & \hat{\beta}_u - \hat{\gamma}(\bar{w}_1 - \bar{w}_0) \\
              & = & 0.7545 - 0.4250 \cdot 0.1574 \\
              & = & 0.6876
\end{eqnarray*}
$w$ cannot be a confounder because this is a randomized trial.

%%%%%%%%%%%%%%%%%%%%%%%%%%%%%%%%%%%%%%%%%%%%%%%%%%%%%%%%%%%%%%%%%%%%%%%
%% Problem 4
%%%%%%%%%%%%%%%%%%%%%%%%%%%%%%%%%%%%%%%%%%%%%%%%%%%%%%%%%%%%%%%%%%%%%%%
\item
\begin{enumerate}
\item[(a)] 
\item[(b)] 
\end{enumerate}

%%%%%%%%%%%%%%%%%%%%%%%%%%%%%%%%%%%%%%%%%%%%%%%%%%%%%%%%%%%%%%%%%%%%%%%
%% Problem 5
%%%%%%%%%%%%%%%%%%%%%%%%%%%%%%%%%%%%%%%%%%%%%%%%%%%%%%%%%%%%%%%%%%%%%%%
\item
\begin{enumerate}
\item[(a)] 
Assuming power = 90\%, $\beta = 0.1; \alpha = 0.025$ \\
Assuming equal number of patients in each treatment group, $\xi_0=\xi_1=\frac{1}{2}$ \\
Hazard ratio $= r = 0.8$ \\
By Schoenfeld’s formula we have:
\[ \dfrac{(Z_{1-\alpha}+Z_{1-\beta})^ 2}{\xi_0\xi_1(\log r)^2} = \dfrac{3.24^ 2}{(1/2)^2(\log 0.8)^2} = 844.09 \]

%Assume no censoring. \\
The hazard rate $\lambda$ for the control group is constant at .10/person-year. The active group will have a hazard of 0.8/person-year. Hence, the average of the control and active group hazards is 0.09/person-year; The cumulative hazard is $\Lambda(t) = 0.09t, \ 0 \leq t \leq 4$. \\
The probability that a subject experiences an event before time $t$ is $1 - e^{-\Lambda(t)}$. If the total length of follow-up is $F$ and the length of the recruitment period is $R$, then the probability of an event is
\begin{eqnarray*}
\bar{\rho} &=& \frac{1}{R} \int_{F - R}^{F} 1 - e^{-\Lambda(s)}ds \\
           &=& 1 - \frac{1}{R} \int_{F - R}^{F} e^{-0.09s} ds \\
           &=& 1 + \frac{1}{0.09R}\left(e^{-0.09F} - e^{-0.09(F - R)} \right) \\
           &=& 0.2356
\end{eqnarray*}
Hence, the total number of events required is 844 and the expected sample size is $\frac{844}{0.2356} = 3583$. 

%If we assume Censoring times, $C_{ij}$ are uniformly distributed on interval $[F - R, F]$, then the expected sample size is given by:
%\[ n = \frac{(Z_{1-\alpha}+Z_{1-\beta})^ 2 R \bar{\lambda}}{\left(R\bar{\lambda} - e^{-\bar{\lambda}(F - R)} + e^{-\bar{\lambda}F} \right) \xi_0\xi_1(\log r)^2} = \]

\item[(b)] 
\item[(c)] The critical values $b_k$, for $Z_k$ for rejecting $H_0: \mu \geq 0$ are:
% latex table generated in R 3.2.2 by xtable 1.7-4 package
% Sat Dec 12 16:16:00 2015
\begin{table}[H]
\centering
\begin{tabular}{rrrrr}
  \hline
  Time &  Upper & Exit pr. & Diff. pr. & Nominal Alpha \\ 
  \hline
  0.2000 & 3.2905 & 0.0010 & 0.0010 & 0.0010 \\ 
  0.4000 & 2.9404 & 0.0040 & 0.0030 & 0.0033 \\ 
  0.6000 & 2.7211 & 0.0090 & 0.0050 & 0.0065 \\ 
  0.8000 & 2.5481 & 0.0160 & 0.0070 & 0.0108 \\ 
  1.0000 & 2.4011 & 0.0250 & 0.0090 & 0.0163 \\ 
  \hline
\end{tabular}
\end{table}
The critical values at $t_k = (0.2, 0.4, 0.6, 0.8, 1.0)$ are 3.2905, 2.9404, 2.7211, 2.5481, 2.4011.

\item[(d)] The non-centrality parameter is
\begin{eqnarray*}
 \theta & = & \sqrt{\I}\log r \\
        & = & \sqrt{D \xi_1 \xi_2} \log r \\
        & = & \sqrt{844 \cdot 0.5 \cdot 0.5 } \log 0.8 \\
        & = & -3.2413
\end{eqnarray*}
For any interim period $k$, $\Exp(Z_k) = \sqrt{t_k}\theta$. Hence,\\
$\Exp(Z_1) = -1.4496$; 
$\Exp(Z_2) = -2.05$; 
$\Exp(Z_3) = -2.5107$; 
$\Exp(Z_4) = -2.8991$; 
$\Exp(Z_5) = -3.2413$ \\
Hence, the critical values $a_k$, for rejecting $H_1: \mu \leq \log 0.8$ are:\\
$a_1 = -4.74, \ a_2 = -4.99,\ a_3 = -5.23,\ a_4 = -5.45,\ a_5 = -5.64$

\item[(e)] Power = $0.8160 + \sum_{k = 1}^5 \Prob(\text{Rejecting } H_0) = 0.8160 + 0.0380 = 0.8540$

\item[(f)] 
\item[(g)] 
\item[(h)] 
\end{enumerate}

%%%%%%%%%%%%%%%%%%%%%%%%%%%%%%%%%%%%%%%%%%%%%%%%%%%%%%%%%%%%%%%%%%%%%%%
%% Problem 6
%%%%%%%%%%%%%%%%%%%%%%%%%%%%%%%%%%%%%%%%%%%%%%%%%%%%%%%%%%%%%%%%%%%%%%%
\item
\begin{enumerate}
\item[(a)] 
\item[(b)] 
\item[(c)] 
\item[(d)] 
\item[(e)] 
\item[(f)] 
\end{enumerate}

\end{enumerate}

\end{document}


