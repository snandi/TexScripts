\documentclass[10pt,dvipsnames,table, handout]{beamer} % To printout the slides without the animations
%\documentclass[10pt,dvipsnames,table]{beamer} 
%\usetheme{Luebeck} 
\usetheme{Madrid} 
%\usetheme{Marburg} 
\setbeamercolor{structure}{fg=cyan!90!white}
%\setbeamercolor{normal text}{fg=white, bg=black}

%%%%%%%%%%%%%%%%%%%%%%%% Packages %%%%%%%%%%%%%%%%%%%%%%%%
\usepackage{amscd}
\usepackage{amsmath}
\usepackage{amssymb}
\usepackage{amsthm}
\usepackage{amsxtra}
\usepackage{animate}
\usepackage{bbold}
%\usepackage{bigints}
\usepackage{color, colortbl}
\usepackage{dsfont}
\usepackage{enumerate}
\usepackage[mathscr]{eucal}
%\usepackage{fancyhdr}
\usepackage{float}
%\usepackage{fullpage} %% Dont use this for beamer presentations
\usepackage{geometry}
\usepackage{graphicx}
\usepackage{hyperref}
\usepackage{indentfirst}
\usepackage{latexsym}
\usepackage{listings}
\usepackage{lscape}
\usepackage{mathtools}
\usepackage{microtype}
\usepackage{multirow}
\usepackage{natbib}
\usepackage{pdfpages}
\usepackage{verbatim}
\usepackage{wrapfig}
\usepackage{xargs}
\usepackage{xcolor}
\DeclareGraphicsExtensions{.pdf,.png,.jpg, .jpeg}
\definecolor{LightCyan}{rgb}{0.88,1,1}

%%%%%%%%%%%%%%%%%%%%%%%% Commands %%%%%%%%%%%%%%%%%%%%%%%%
\newcommand{\Sup}{\textsuperscript}
\newcommand{\Exp}{\mathds{E}}
\newcommand{\Prob}{\mathds{P}}
\newcommand{\Z}{\mathds{Z}}
\newcommand{\Ind}{\mathds{1}}
\newcommand{\A}{\mathcal{A}}
\newcommand{\F}{\mathcal{F}}
\newcommand{\G}{\mathcal{G}}
\newcommand{\I}{\mathcal{I}}
\newcommand{\R}{\mathcal{R}}
\newcommand{\Real}{\mathbb{R}}
\newcommand{\be}{\begin{equation}}
\newcommand{\ee}{\end{equation}}
\newcommand{\bes}{\begin{equation*}}
\newcommand{\ees}{\end{equation*}}
\newcommand{\union}{\bigcup}
\newcommand{\intersect}{\bigcap}
\newcommand{\Ybar}{\overline{Y}}
\newcommand{\ybar}{\bar{y}}
\newcommand{\Xbar}{\overline{X}}
\newcommand{\xbar}{\bar{x}}
\newcommand{\betahat}{\hat{\beta}}
\newcommand{\Yhat}{\widehat{Y}}
\newcommand{\yhat}{\hat{y}}
\newcommand{\Xhat}{\widehat{X}}
\newcommand{\xhat}{\hat{x}}
\newcommand{\E}[1]{\operatorname{E}\left[ #1 \right]}
%\newcommand{\Var}[1]{\operatorname{Var}\left( #1 \right)}
\newcommand{\Var}{\operatorname{Var}}
\newcommand{\Cov}[2]{\operatorname{Cov}\left( #1,#2 \right)}
\newcommand{\N}[2][1=\mu, 2=\sigma^2]{\operatorname{N}\left( #1,#2 \right)}
\newcommand{\bp}[1]{\left( #1 \right)}
\newcommand{\bsb}[1]{\left[ #1 \right]}
\newcommand{\bcb}[1]{\left\{ #1 \right\}}
\newcommand*{\permcomb}[4][0mu]{{{}^{#3}\mkern#1#2_{#4}}}
\newcommand*{\perm}[1][-3mu]{\permcomb[#1]{P}}
\newcommand*{\comb}[1][-1mu]{\permcomb[#1]{C}}


%%%%%%%%%%%%% For explanatory bubbles, use the following code %%%%%%%%%%%%%
%% \usepackage{tikz} %% For explanatory bubbles
%% \usepackage{xparse}
%% \usetikzlibrary{shapes.callouts,ocgx}

%% \newcommand{\tikzmark}[1]{\tikz[overlay,remember picture,baseline=0.5ex] \node (#1) {};}

%% % \explainword: #1= identifier to mark the word, #2 text
%% \NewDocumentCommand{\explainword}{r[] m}{
%%     \switchocg{#1}{#2}\tikzmark{#1}
%% }

%% \tikzset{my callout style/.style={
%%         draw,rectangle callout,anchor=pointer,callout relative pointer={(230:1cm)},
%%         rounded corners,align=center,text width=2cm,fill=cyan!20, 
%%     }
%% }

%% % \mycallout: #1 opacity style, #2 pointer base position, #3= text
%% \NewDocumentCommand{\mycallout}{O{opacity=0.8,text opacity=1} m m}{%
%% \begin{tikzpicture}[remember picture, overlay]
%%  \begin{scope}[ocg={ref=#2,status=invisible,name={#3}}]
%% \node[my callout style,#1]at (#2) {#3};
%% \end{scope}
%% \end{tikzpicture}
%% }
%%%%%%%%%%%%%%%%%%%%%%%%%%%%%%%%%%%%%%%%%%%%%%%%%%%%%%%%%%%%%%%%%

%%%%%%%%%%%%%%%%%%%%%%%% TITLE PAGE %%%%%%%%%%%%%%%%%%%%%%%%
\DeclarePairedDelimiter\ceil{\lceil}{\rceil}
\title[Joint Modeling of Functional and Survival Data]{Functional Principal Component Analysis for Longitudinal and Survival Data \\ Fang Yao\\ Statistica Sinica, 2007}
\author{Subhrangshu Nandi}
\institute[Stat 741]{Stat 741, Spring 2015 \\
  Department of Statistics \\
 University of Wisconsin-Madison}
\date{April 21, 2015}

\begin{document}
\setlength{\baselineskip}{16truept}
\frame{\maketitle}

%The presentation must follow this template.
%1. Background 
%2. Problem identification
%3. Data examples for the Problem
%4. Innovation and significance of the solution
%5. The solution: both theoretical and computational
%6. Results (both simulation and real data analysis)
%7. Limitation of the solution (conditions of the theorem, computational limitation, model assumption, etc)

%%%%%%%%%%%% Slide 1 %%%%%%%%%%%%
\begin{frame}
\frametitle{Background}
\begin{block}{Cox Proportional Hazard Model}
\[ \lambda_i(t|{\mathbf{Z}}) = \lambda_0(t) \exp\{ {\bf{Z_i}}^T {\mathbf{\beta}}\}, \ \ i = 1, \dots n,\ p \text{ covariates} \]
\end{block}
\pause
\begin{block}{Cox Model, with Time-varying covariate}
\[ \lambda_i(t|{\mathbf{Z_i(s)}}; 0 \leq s \leq t) = \lambda_0(t) \exp\{ {\bf{Z_i(t)}}^T {\mathbf{\beta}}\} \]
\end{block}
\pause
\begin{block}{Cox Model, with Functional covariate}
\begin{itemize}
\item One of the covariates functional variable
\item Measurement error in the functional covariate
\item Need evolution history of functional covariate 
\item \textcolor{green}{Joint modeling} of functional covariate and hazard 
\end{itemize}
\end{block}
\end{frame}
%%%%%%%%%%%%%%%%%%%%%%%%%%%%%%%%%

%%%%%%%%%%%% Slide 2 %%%%%%%%%%%%
\begin{frame}
\frametitle{Example}
Functional covariate of two types of patients:
\begin{center}
\includegraphics[scale=0.17, page=1]{SimPlots.pdf} 
\includegraphics[scale=0.17, page=7]{SimPlots.pdf} 
\includegraphics[scale=0.17, page=11]{SimPlots.pdf} 
\includegraphics[scale=0.17, page=15]{SimPlots.pdf} \\

\includegraphics[scale=0.17, page=6]{SimPlots.pdf} 
\includegraphics[scale=0.17, page=10]{SimPlots.pdf} 
\includegraphics[scale=0.17, page=12]{SimPlots.pdf} 
\includegraphics[scale=0.17, page=18]{SimPlots.pdf} 
\end{center}
\end{frame}

%%%%%%%%%%%% Slide 3 %%%%%%%%%%%%
\begin{frame}
\frametitle{Application to CD4 count data}
\begin{itemize}
\item 467 HIV-infected patients - diagnosed with AIDS or two CD4 counts $< 300$
\item Two different drugs for treatment - ddC, ddI
\item CD4 observed at 5 time points - 0, 2, 6, 12, 18 months (Observed longitudinal covariate $Y_{i}(t_{ij})$)
\item $X_{i}(t_{ij}): $ Unobserved longitudinal covariate is the physiological response to the treatment
\item Goal: Investigate relationship between treatment effect of two drugs with survival
\end{itemize}
\end{frame}

\begin{frame}
\frametitle{CD4 count data, 8 random patients}
% Results for CD4 count
\begin{center}
\includegraphics[scale=0.33]{CD4_Patients.jpg} \\
\end{center}
\end{frame}

%%%%%%%%%%%% Slide 3 %%%%%%%%%%%%
\begin{frame}
\frametitle{Problem: ``Joint modeling of longitudinal and survival data''}
Previous work: {\bf{Brown, E. R.}}, Ibrahim, J. G. and DeGruttola, V. (2005). A flexible B-spline model for multiple
longitudinal biomarkers and survival. {\emph{Biometrics}} {\bf{61}}, 64-73. \\
\pause

{\underline{Contribution of this paper:}} \\
Proposes a nonparametric approach for {\underline{jointly}} modelling longitudinal and survival data using {\underline{functional principal components}}. Some key aspects:\\
\begin{itemize}
\item Fast and efficient implementation, by choosing few FPC's
\item No known structure of underlying longitudinal process
\item Iterative selection procedure for choosing tuning parameters (\# of eigen functions, \# of knots of spline models), using AIC 
\item EM algorithm used for joint estimation of parameters
\end{itemize}
\end{frame}

%%%%%%%%%%%% Slide 4 %%%%%%%%%%%%
\begin{frame}
\frametitle{Background - Functional PCA}
%\begin{columns}
%\column{.5\textwidth} 
\begin{table}[H]
\centering
\begin{tabular}{l|l|l}
\hline
Element & In PCA & in FPCA (Functional PCA) \\ 
\hline
Data & $X \in \Real ^p $ & $X \in \mathcal{H} = L^2(\I)$ \\
Dimension & $p < \infty$ & $\infty$ \\
Mean & $\mu = \Exp (X) $ & $\mu(t) = \Exp(X(t))$ \\
Covariance & $\text{Cov}(X) = \Sigma_{p x p}$ &  $\text{Cov}(X(s), X(t)) = G(s, t)$  \\
Eigenvalues &  $\lambda_1, \lambda_2, \dots, \lambda_p$  & $ \lambda_1, \lambda_2, \dots $\\
Eigen- vectors/functions & $ \mathbf{v}_1, \mathbf{v}_2, \dots, \mathbf{v}_p $ & $ \varphi_1(t), \varphi_2(t), \dots $\\
Inner Products & $ \langle \mathbf{X}, \mathbf{Y} \rangle = \sum_{k=1}^p X_k Y_k $ & $ \langle X, Y \rangle = \int_\mathcal{I} X(t) Y(t) dt $\\
\multirow{2}*{PCAs} & $ z_k = \langle X - \mu, \mathbf{v_k} \rangle,$ & $ \xi_k = \langle X - \mu, \varphi_k\rangle, $ \\ 
& ${ k = 1, 2, \dots, p }$ & $ k = 1, 2, \dots $ \\ 
\hline
\end{tabular}
\caption{Comparing PCA and fPCA}
\end{table}
%\column{.5\textwidth} 
%\end{columns}

$ \text{Cov}(X(s), X(t)) = G(s, t) = \sum_{k=1}^{\infty} \lambda_k \varphi_k(s) \varphi_k(t),\ \ s, t \in [0, \tau]$
\end{frame}

\begin{frame}
\frametitle{FPCA: Precipitation of 35 Canadian cities, over 365 days}
%\begin{columns}
%\column{.4\textwidth} 
\begin{center}
\includegraphics[scale=0.4, page=1]{PCAPlot_1.pdf} \\
\end{center}
\end{frame}

\begin{frame}
\frametitle{FPCA: Precipitation of 35 Canadian cities, over 365 days}
%\begin{columns}
%\column{.4\textwidth} 
\begin{center}
\includegraphics[scale=0.2, page=1]{PCAPlot_1.pdf} \\
%\column{.6\textwidth} 
\includegraphics[scale=0.23, page=1]{PCAPlot_3fpca.pdf}
\includegraphics[scale=0.23, page=2]{PCAPlot_3fpca.pdf}
\includegraphics[scale=0.23, page=3]{PCAPlot_3fpca.pdf} 
\end{center}
%\end{columns}
\end{frame}

%%%%%%%%%%%% Slide 5 %%%%%%%%%%%%
\begin{frame}
\frametitle{Background - B-Spline fitting}
% Formula and plots with different knots
Any spline function of order $k$ on a given set of {\textcolor{red}{knots}} can be expressed as a linear combination of B-splines. 
\[ S_{k,t}(x) =\sum_i \alpha_i B_{i,k}(x),\ \ t_0 \leq x \leq t_m \]
\pause
\begin{columns}
\column{.5\textwidth} 
\begin{center}
More knots - Overfitting?
%\includegraphics[scale=0.25, page=1]{BSplinePlot.pdf} \\
\includegraphics[scale=0.25, page=2]{BSplinePlot.pdf}
\end{center}

\column{.5\textwidth} 
\begin{center}
Fewer knots - Underfitting?
%\includegraphics[scale=0.25, page=3]{BSplinePlot.pdf} \\
\includegraphics[scale=0.25, page=4]{BSplinePlot.pdf}
\end{center}
\end{columns}
\end{frame}

%% Notes
%% number of basis functions = order + number of interior knots
%% The usefulness of B-splines lies in the fact that any spline function of order k on a given set of knots can be expressed as a linear combination of B-splines.
%%  a spline is a numeric function that is piecewise-defined by polynomial functions, and which possesses a sufficiently high degree of smoothness at the places where the polynomial pieces connect (which are known as knots)

%%%%%%%%%%%% Slide 3 %%%%%%%%%%%%
\begin{frame}
\frametitle{Problem: ``Joint modeling of longitudinal and survival data''}
Revisiting\\
\vspace{1cm}

{\underline{Contribution of this paper:}} \\
Proposes a nonparametric approach for {\underline{jointly}} modelling longitudinal and survival data using {\underline{functional principal components}}. Some key aspects:\\
\begin{itemize}
\item Fast and efficient implementation, by choosing few FPC's
\item No known structure of underlying longitudinal process
\item Iterative selection procedure for choosing tuning parameters (\# of eigen functions, \# of knots of spline models), using AIC 
\item EM algorithm used for joint estimation of parameters
\end{itemize}
\end{frame}

%%%%%%%%%%%% Slide 6 %%%%%%%%%%%%
\begin{frame}
\frametitle{Longitudinal Process}
% Breakdown of the likelihood by the different variables
\begin{itemize}
\pause \item Let $\mu(t):$ the overall mean function \\
$Z_i(t):$ vector of covariates (example, treatment indicator), Then
\[ \mu_i(t) = \mu(t|Z_i) = \mu(t) + \beta^T Z_i(t), \ \ t \in [0, \tau] \]
\pause \item $X_i(t):$ the realization of the latent longitudinal process for the $i^{th}$ individual \\
$G(s, t) = \Cov{X_i(s)}{X_i(t)}$ covariance structure of $Z_i (t)$
\[ X_i (t) = \mu_i (t) + \sum_{k=1}^{K} \xi_{ik} \phi_k (t) \]
\pause \item $\epsilon _{ij} \sim  \mathcal{N}(0, \sigma^2): $ Uncorrelated measurement errors, then the sub-model for the longitudinal covariate $Y_{ij}$ is 
\[ Y_{ij} =  X_i(t_{ij}) + \epsilon _{ij} = \mu(t_{ij}) + Z_i(t_{ij})^T\beta + \sum_{k=1}^{K} \xi_{ik} \phi_k (t_{ij})+ \epsilon_{ij}\]
\end{itemize}

\end{frame}

%%%%%%%%%%%% Slide 7 %%%%%%%%%%%%
\begin{frame}
\frametitle{Longitudinal Process and Hazard}
\begin{itemize}
\pause \item Fitting overall mean function, with B-Splines \\
\begin{center}
$ \mu(t) =  \mathbf{B}_p(t)^T \mathbf{\alpha},\ \ \alpha = (\alpha_1, \dots, \alpha_p)^T $
\end{center}
\pause \item Fitting each eigen function, with another set of B-Splines \\
\begin{center}
$ \sum_{k=1}^{K} \xi_{ik} \phi_k (t_{ij}) =  \sum_{k=1}^{K} \xi_{ik} \widetilde{\mathbf{B}}_q(t)^T \mathbf{\theta _k},\ \ \theta_{k} = (\theta_{1k}, \dots, \theta_{qk})^T $
\end{center}
\pause \item Cox Model, with some more covariates $V(t)$
\begin{center}
$ h_i(t) = h_0(t)\exp\{\gamma X_i(t) + V_i(t)^T \zeta \}$
\end{center}
\pause \item Final parameter vector of interest:
\begin{center}
$ \Omega = \left\{ \gamma, \zeta, h_0(\cdot), \alpha, \beta, \Theta, \Lambda, \sigma^2 \right\} $
\end{center}
\item Tuning parameters: $p, q, K$
\end{itemize}
\end{frame}

%%%%%%%%%%%% Slide 8 %%%%%%%%%%%%
\begin{frame}
\frametitle{Solution/Estimation}
\begin{itemize}
\item Iterative selection procedure for choosing knots (p, q) and number of PCAs (K)
\item Equi-quantile knots chosen, instead of equi-distant (to not underweight time points with fewer observations)
\item AIC used for choosing these tuning parameters
\pause \item EM algorithm for parameter estimation $ \Omega = \left\{ \gamma, \zeta, h_0(\cdot), \alpha, \beta, \Theta, \Lambda, \sigma^2 \right\} $
\item In E-step, Monte Carlo integration used instead of Gaussian-Hermite quadrature
\pause \item Newton-Raphson algorithm used to estimate the Cox-model 
\[ \hat{\eta}^{(l)} = \hat{\eta}^{(l-1)} + I^{-1}_{\hat{\eta}^{(l-1)}} S_{\hat{\eta}^{(l-1)}},\ \ \eta = (\gamma, \zeta^T)\]
\end{itemize}
\pause
\begin{center}
Three iterative procedures!! 
\end{center}
\end{frame}

%%%%%%%%%%%% Slide 9 %%%%%%%%%%%%
\begin{frame}
\frametitle{Simulation}
\begin{columns}
\column{.6\textwidth} 
Longitudinal process: \\ 
$ Y_{ij} =  X_i(t_{ij}) + \epsilon _{ij} 
= \underbrace{\mu(t_{ij})}_{\text{True mean}} + Z_i(t_{ij})^T\beta + \sum_{k=1}^{K} \xi_{ik} \phi_k (t_{ij})+ \epsilon_{ij}$ \\
\column{.4\textwidth} 
\includegraphics[scale=0.2, page=7]{SimPlots.pdf} \\
\end{columns}
\vspace{0.5cm}
\pause
Cox PH model: $ h_i(t) = h_0(t)\exp\{\gamma X_i(t) + V_i(t)^T \zeta \}$ \\
$\gamma: $ Parameter of interest \\
\pause
Simulation scenarios:
\begin{enumerate}
\item TRUE: true mean known
\item IDEAL: true trajectories ($X_i(t)$) used (no measurement error)
\item FPCA: Proposed model
\end{enumerate}
\end{frame}

\begin{frame}
\frametitle{Simulation Results}
\begin{itemize}
\item True $\gamma = -1$
\item 200 samples with FPCA scores $\sim \mathcal{N}(0, \lambda)$
\item 200 samples with FPCA scores $\sim$ mixture distribution\item Each sample with 200 i.i.d. individuals
\item Baseline $h_0(t) = t^2/100 $
\item Independent censoring - 20\% dropouts at $t=6$, 70\% dropouts at $t=9$
\item True mean of longitudinal covariate: $\mu(t) = \frac{1}{3}\sin\left(\frac{3 \pi t}{40} \right)$
\item 2 Eigen functions, with eigen values: 10, 1
\end{itemize}
\begin{table}[H]
\centering
\begin{tabular}{c|c|c|c|c}
\hline
& \multicolumn{2}{|c|}{Normal} & \multicolumn{2}{|c}{Mixture} \\ 
\hline
& Mean & SD & Mean & SD \\
\hline
TRUE & -0.998 & 0.114 & -1.03 & 0.117\\
\hline
IDEAL & -1.02 & 0.119 & -1.02 & 0.108\\
\hline
\rowcolor{gray}
FPCA & -1.05 & 0.112 & -0.997 & 0.115\\
\hline
\end{tabular}
\caption{Simulation Results}
\end{table}

\end{frame}

%%%%%%%%%%%% Slide 10 %%%%%%%%%%%%
\begin{frame}
\frametitle{Application to CD4 count data}
\begin{itemize}
\item 467 HIV-infected patients - diagnosed with AIDS or two CD4 counts $< 300$
\item Two different drugs for treatment - ddC, ddI
\item CD4 observed at 5 time points - 0, 2, 6, 12, 18 months (Observed longitudinal covariate $Y_{i}(t_{ij})$)
\item $X_{i}(t_{ij}): $ Unobserved longitudinal covariate is the physiological response to the treatment
\item Goal: Investigate relationship between treatment effect of two drugs with survival
\end{itemize}

\end{frame}

\begin{frame}
\frametitle{CD4 count data, 8 random patients}
% Results for CD4 count
\begin{center}
\includegraphics[scale=0.33]{CD4_Patients.jpg} \\
\end{center}
\end{frame}

\begin{frame}
\frametitle{CD4 count data - Survival function}
\begin{center}
\includegraphics[scale=0.32]{CD4_Survival.jpg} \\
\end{center}
\end{frame}

%%%%%%%%%%%% Slide 11 %%%%%%%%%%%%
\begin{frame}
\frametitle{Critical appreciation}
\begin{itemize}
\item Contribution not significant \\
\pause
\includegraphics[scale=0.35]{ImprovementText.jpg} \\
\pause \item Marginal improvement over using just linear fit and random effects
\pause \item Simulation not comprehensive
\pause \item Could have applied this to more complex data
\pause \item No large sample results
\pause \item Only 11 citations (google scholar) in 8 years
\end{itemize}
\end{frame}

\end{document}

