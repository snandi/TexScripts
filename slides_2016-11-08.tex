%\documentclass[10pt,dvipsnames,table, handout]{beamer} % To printout the slides without the animations
\documentclass[10pt,dvipsnames,table]{beamer} 
%\usetheme{Luebeck} 
%\usetheme{Madrid} 
%\usetheme{Marburg} 
%\usetheme{Warsaw} 
%\setbeamercolor{structure}{fg=cyan!90!white}
%\setbeamercolor{normal text}{fg=white, bg=black}
\usetheme{CambridgeUS}
%\setbeamercolor{structure}{fg=cyan!90!white}
%\setbeamercolor{normal text}{fg=white, bg=black}
\setbeamercolor{block title}{bg=red!80,fg=white}

%%%%%%%%%%%%%%%%%%%%%%%%%%%%%%%%%%%%%%%%%%%%%%%%%%%%%%%%%%%%%%%%%%%%%%
%% Input header file 
%%%%%%%%%%%%%%%%%%%%%%%%%%%%%%%%%%%%%%%%%%%%%%%%%%%%%%%%%%%%%%%%%%%%%%
%%%%%%%%%%%%%%%%%%%%%%%% Packages %%%%%%%%%%%%%%%%%%%%%%%%
\usepackage{amscd}
\usepackage{amsmath}
\usepackage{amssymb}
\usepackage{amsthm}
\usepackage{amsxtra}
\usepackage{animate}
\usepackage{bbold}
%\usepackage{bigints}
\usepackage{color, colortbl}
\usepackage{dsfont}
\usepackage{enumerate}
\usepackage[mathscr]{eucal}
%\usepackage{fancyhdr}
\usepackage{float}
%\usepackage{fullpage} %% Dont use this for beamer presentations
\usepackage{geometry}
\usepackage{graphicx}
\usepackage{hyperref}
\usepackage{indentfirst}
\usepackage{latexsym}
\usepackage{listings}
\usepackage{lscape}
\usepackage{mathtools}
\usepackage{microtype}
\usepackage{multirow}
\usepackage{natbib}
\usepackage{pdfpages}
\usepackage{verbatim}
\usepackage{wrapfig}
\usepackage{xargs}
\usepackage{xcolor}
\DeclareGraphicsExtensions{.pdf,.png,.jpg, .jpeg}
\definecolor{LightCyan}{rgb}{0.88,1,1}

%%%%%%%%%%%%%%%%%%%%%%%% Commands %%%%%%%%%%%%%%%%%%%%%%%%
\newcommand{\Sup}{\textsuperscript}
\newcommand{\Exp}{\mathds{E}}
\newcommand{\Prob}{\mathds{P}}
\newcommand{\Z}{\mathds{Z}}
\newcommand{\Ind}{\mathds{1}}
\newcommand{\A}{\mathcal{A}}
\newcommand{\F}{\mathcal{F}}
\newcommand{\G}{\mathcal{G}}
\newcommand{\I}{\mathcal{I}}
\newcommand{\R}{\mathcal{R}}
\newcommand{\Real}{\mathbb{R}}
\newcommand{\be}{\begin{equation}}
\newcommand{\ee}{\end{equation}}
\newcommand{\bes}{\begin{equation*}}
\newcommand{\ees}{\end{equation*}}
\newcommand{\union}{\bigcup}
\newcommand{\intersect}{\bigcap}
\newcommand{\Ybar}{\overline{Y}}
\newcommand{\ybar}{\bar{y}}
\newcommand{\Xbar}{\overline{X}}
\newcommand{\xbar}{\bar{x}}
\newcommand{\betahat}{\hat{\beta}}
\newcommand{\Yhat}{\widehat{Y}}
\newcommand{\yhat}{\hat{y}}
\newcommand{\Xhat}{\widehat{X}}
\newcommand{\xhat}{\hat{x}}
\newcommand{\E}[1]{\operatorname{E}\left[ #1 \right]}
%\newcommand{\Var}[1]{\operatorname{Var}\left( #1 \right)}
\newcommand{\Var}{\operatorname{Var}}
\newcommand{\Cov}[2]{\operatorname{Cov}\left( #1,#2 \right)}
\newcommand{\N}[2][1=\mu, 2=\sigma^2]{\operatorname{N}\left( #1,#2 \right)}
\newcommand{\bp}[1]{\left( #1 \right)}
\newcommand{\bsb}[1]{\left[ #1 \right]}
\newcommand{\bcb}[1]{\left\{ #1 \right\}}
\newcommand*{\permcomb}[4][0mu]{{{}^{#3}\mkern#1#2_{#4}}}
\newcommand*{\perm}[1][-3mu]{\permcomb[#1]{P}}
\newcommand*{\comb}[1][-1mu]{\permcomb[#1]{C}}


%%%%%%%%%%%%%%%%%%%%%%%%%%%%%%%%%%%%%%%%%%%%%%%%%%%%%%%%%%%%%%%%%%%%%%
%% TITLE PAGE 
%%%%%%%%%%%%%%%%%%%%%%%%%%%%%%%%%%%%%%%%%%%%%%%%%%%%%%%%%%%%%%%%%%%%%%
\DeclarePairedDelimiter\ceil{\lceil}{\rceil}
\title[Fluoroscanning]{Fluoroscanning: {\emph{Comparison of FScan Consensus and Genomic Sequences}}}
\author{Subhrangshu Nandi}
%\institute[Stat 741]{Stat 741, Spring 2015 \\
% Department of Statistics \\
% University of Wisconsin-Madison}
\date{November 08, 2016}

\begin{document}
\setlength{\baselineskip}{16truept}
\frame{\maketitle}

%%%%%%%%%%%%%% Slide 1 %%%%%%%%%%%%%%
\begin{frame}
\frametitle{Distance between consensus signals}
Similarity index $\rho(f, g)$ between two consensus FScan signals $f$ and $g$:
\[ \rho(f, g) = \frac{\int _{S_{fg}}f(s)g(s) ds}{\int _{S_{fg}}f(s)^2 ds \int _{S_{fg}}g(s)^2 ds} \]
where, $S_{fg} = S_f \cap S_g$, the intersection of Sobolev spaces formed by the functions $f, g$.\\
$-1 \leq \rho(f, g) \leq 1$ \\
Higher values of $\rho$ implies more {\emph{similarity}} between two consensus

\end{frame}

%%%%%%%%%%%%%% Slide 2 %%%%%%%%%%%%%%
\begin{frame}
\frametitle{Distance between sequences of two intervals: Similarity between GC Profiles}
\begin{figure}
\centering
\includegraphics[width=0.42\textwidth]{chr19_frag2481_pg8.png}
\includegraphics[width=0.42\textwidth]{chr19_frag3756_pg8.png}
\caption{$\rho_{gc} = 0.84, \rho_{fscan} = 0.70$}
\end{figure}

\end{frame}

%%%%%%%%%%%%%% Slide 3 %%%%%%%%%%%%%%
\begin{frame}
\frametitle{Dissimilar GC and Fscan profiles}
\begin{figure}
\centering
\includegraphics[width=0.42\textwidth]{chr19_frag2481_pg8.png}
\includegraphics[width=0.42\textwidth]{chr19_frag5423_pg8.png}
\caption{$\rho_{gc} = -0.56, \rho_{fscan} = -0.21$}
\end{figure}

\end{frame}

%%%%%%%%%%%%%% Slide 4 %%%%%%%%%%%%%%
\begin{frame}
\frametitle{Pairwise comparison results}
\begin{itemize}
\item In chromosome 19, there are 457 sub-intervals that are 10.3 kb long, with at least 15 Nmaps depth
\item $ \binom {457} {2} = 104,196 $ pairs to compare. 
\item Mixed effects model to estimate relationship between consensus FScans ($\rho_{fscan}$) and GC profile scores $\rho_{gc}$:
\[ 
\rho_{fs_{i,j}} = \alpha_i + \gamma_j + \beta \rho_{gc_{i,j}} + \epsilon_{i,j}\ \ \ i, j = 1, \dots, 457,\ \ i \ne j
\]
$\alpha_i \sim \mathcal{N}(0, \tau_1^2), \ \ \gamma_j \sim \mathcal{N}(0, \tau_2^2), \ \ \epsilon_{i,j} \sim \mathcal{N}(0, \sigma^2)$ \\
$ \alpha_i \indep \gamma_j \indep \epsilon_{i,j}$
\end{itemize}

\begin{block}{Random effect model estimate}
$\hat{\beta} = 0.149, \text{p-value} = 0.0$ (likelihood-ratio test)
\end{block}
\end{frame}

%%%%%%%%%%%%%% Slide 5 %%%%%%%%%%%%%%
\begin{frame}
\frametitle{Rank based regression}
To fit the linear model
\[ y = \alpha + X\beta + \epsilon\]
the rank estimator of $\beta$ is defined as
\[ \hat{\beta}_{\varphi} = \text{Arg min } \| y - X\beta\|_{\varphi}\]
where, $\| \cdot \|_{\varphi}$ is Jaeckel’s dispersion function, a pseudo-norm defined by
\[ \| u \|_{\varphi} = \sum\limits_{i=1}^n a R(u_i)u_i\]
where, $R$ denotes rank and $a(t) = \varphi(\frac{t}{n+1})$, and $\varphi$ is a
non-decreasing, square-integrable score function defined on the interval $(0, 1)$.

\begin{block}{Rank based estimate}
$ \hat{\beta}_{\varphi} = 0.156$, p-value = 0 (Wald test)
\end{block}

\end{frame}

%%%%%%%%%%%%%% Slide 6 %%%%%%%%%%%%%%
\begin{frame}
\frametitle{Estimate of fewer intervals with signals}
Using around 20\% of the original dataset

\begin{block}{Random effect model estimate}
$\hat{\beta} = 0.261, \text{p-value} = 0.0$ (likelihood-ratio test)
\end{block}

\begin{block}{Rank based estimate}
$ \hat{\beta}_{\varphi} = 0.277$, p-value = 0 (Wald test)
\end{block}
\end{frame}

%%%%%%%%%%%%%% Slide 7 %%%%%%%%%%%%%%
\begin{frame}
\frametitle{Scatterplot with fewer intervals with signals}
\begin{figure}
\centering
\includegraphics[width=0.4\textwidth, page=1]{ScatterPlots.pdf}
\includegraphics[width=0.4\textwidth, page=2]{ScatterPlots.pdf}
\end{figure}

\end{frame}

\end{document}

