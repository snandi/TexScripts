%\documentclass[10pt,dvipsnames,table, handout]{beamer} % To printout the slides without the animations
\documentclass[10pt,dvipsnames,table]{beamer} 
%\usetheme{Luebeck} 
%\usetheme{Madrid} 
%\usetheme{Marburg} 
%\usetheme{Warsaw} 
%\setbeamercolor{structure}{fg=cyan!90!white}
%\setbeamercolor{normal text}{fg=white, bg=black}
\usetheme{CambridgeUS}
%\setbeamercolor{structure}{fg=cyan!90!white}
%\setbeamercolor{normal text}{fg=white, bg=black}
\setbeamercolor{block title}{bg=red!80,fg=white}

%%%%%%%%%%%%%%%%%%%%%%%%%%%%%%%%%%%%%%%%%%%%%%%%%%%%%%%%%%%%%%%%%%%%%%
%% Input header file 
%%%%%%%%%%%%%%%%%%%%%%%%%%%%%%%%%%%%%%%%%%%%%%%%%%%%%%%%%%%%%%%%%%%%%%
\input{HeaderfileTexSlides}

%%%%%%%%%%%%%%%%%%%%%%%%%%%%%%%%%%%%%%%%%%%%%%%%%%%%%%%%%%%%%%%%%%%%%%
%% TITLE PAGE 
%%%%%%%%%%%%%%%%%%%%%%%%%%%%%%%%%%%%%%%%%%%%%%%%%%%%%%%%%%%%%%%%%%%%%%
\DeclarePairedDelimiter\ceil{\lceil}{\rceil}
\title[Fluoroscanning]{Fluoroscanning: {\emph{Predicting Fscan from Sequence}}}
\author[S. Nandi]{Subhrangshu Nandi}
%\institute[Stat 741]{Stat 741, Spring 2015 \\
% Department of Statistics \\


% University of Wisconsin-Madison}
\date{January 12, 2016}

\begin{document}
\setlength{\baselineskip}{16truept}
\frame{\maketitle}

%%%%%%%%%%%%%%%%%%%%%% SLIDE 1 %%%%%%%%%%%%%%%%%%%%%%
\begin{frame}
\frametitle{Predicting Fscan from Sequence}
\begin{itemize}
\item Random Forest
\vspace{1in}
\item Stochastic Gradient Boosting
\end{itemize}
\end{frame}

%%%%%%%%%%%%%%%%%%%%%% SLIDE 2 %%%%%%%%%%%%%%%%%%%%%%
\begin{frame}
\frametitle{Predicting Fscan from Sequence - Set up}
\begin{itemize}
\item Total sample size: 3,860 sub-intervals from chromosomes 17, 18, 19, 20, 21.
\item Each sub-interval: 10kb
\item Counts of 1 to 5-mers
\item Training set: 90\%, Testing set: 10\%
\pause
\item $\rho_{c}:$ {\emph{Similarity}} between average training {\bf{c}}onsensus Fscan and testing sub-interval, $|\rho_{c}| \leq 1$
\item $\rho_{p}:$ {\emph{Similarity}} between average {\bf{p}}redicted Fscan and testing sub-interval, $|\rho_{p}| \leq 1$
\item $\gamma = \frac{\rho_{p} - \rho_{c}}{1 - \rho_{c}} $ {\emph{Similarity}} improvement or prediction improvement
\end{itemize}
\end{frame}

%%%%%%%%%%%%%%%%%%%%%% SLIDE 3 %%%%%%%%%%%%%%%%%%%%%%
\begin{frame}
\frametitle{Example: chr19, interval 5558 (part of testing set)}
\begin{figure}
\includegraphics[page=1, scale=0.3]{chr19_5558_1_predictedBySeed.pdf}
\includegraphics[page=3, scale=0.3]{PredictionComparisonPlots_RFRun03.pdf}
\end{figure}

\begin{block}{{\emph{Similarity}} between average training cFscan and true Fscan}
 $\rho_{c} = 0.0745$
\end{block}
\end{frame}

%%%%%%%%%%%%%%%%%%%%%% SLIDE 4 %%%%%%%%%%%%%%%%%%%%%%
\begin{frame}
\frametitle{Example: chr19, interval 5558 (part of testing set)}
\begin{columns}[t]
\column{0.6\textwidth}
\begin{figure}
\includegraphics[page=3, scale=0.35]{chr19_5558_1_predictedBySeed.pdf}
\end{figure}

\column{0.4\textwidth}
\begin{block}{{\emph{Similarity}} between predicted Fscan and true Fscan}
$\rho_{p} = 0.8937$
\end{block}

\begin{block}{Prediction Improvement}
$\gamma = \frac{\rho_{p} - \rho_{c}}{1 - \rho_{c}} = 0.8851 $
\end{block}

\end{columns}
\end{frame}

%%%%%%%%%%%%%%%%%%%%%% SLIDE 5 %%%%%%%%%%%%%%%%%%%%%%
\begin{frame}
\frametitle{Example: chr18, interval 5869 (part of testing set)}
\begin{columns}[t]
\column{0.6\textwidth}
\begin{figure}
\includegraphics[page=3, scale=0.35]{chr18_5869_1_predictedBySeed.pdf}
\end{figure}

\column{0.4\textwidth}
\begin{block}{{\emph{Similarity}} between predicted Fscan and true Fscan}
 $\rho_{p} = 0.9415$
\end{block}

\begin{block}{Prediction Improvement}
$\gamma = \frac{\rho_{p} - \rho_{c}}{1 - \rho_{c}} = 0.8112 $
\end{block}

\end{columns}
\end{frame}

%%%%%%%%%%%%%%%%%%%%%% SLIDE 6 %%%%%%%%%%%%%%%%%%%%%%
\begin{frame}
\frametitle{Example: chr18, interval 5869 (part of testing set)}

\begin{columns}
\column{0.5\textwidth}
\begin{figure}
\includegraphics[page=3, scale=0.35]{chr18_5869_1_predictedBySeed_RFRun03.pdf}
\end{figure}
 $\rho_{p} = 0.9181$

\column{0.5\textwidth}
\begin{figure}
\includegraphics[page=3, scale=0.35]{chr18_5869_1_predictedBySeed.pdf}
\end{figure}
 $\rho_{p} = 0.9415$
\end{columns}
\end{frame}

%%%%%%%%%%%%%%%%%%%%%% SLIDE 7 %%%%%%%%%%%%%%%%%%%%%%
\begin{frame}
\frametitle{Prediction Improvement - Random Forest}
\begin{figure}
\includegraphics[page=1, scale=0.35]{PredictionComparisonPlots_RFRun03.pdf}
\end{figure}
\end{frame}

%%%%%%%%%%%%%%%%%%%%%% SLIDE 8 %%%%%%%%%%%%%%%%%%%%%%
\begin{frame}
\frametitle{Prediction Improvement - Random Forest \& Gradient Boosting}
\begin{figure}
\includegraphics[page=1, scale=0.33]{PredictionComparisonPlots_RFRun03.pdf}
\includegraphics[page=1, scale=0.33]{PredictionComparisonPlots_GBMRun01.pdf}
\end{figure}
\end{frame}

%%%%%%%%%%%%%%%%%%%%%% SLIDE 9 %%%%%%%%%%%%%%%%%%%%%%
\begin{frame}
\frametitle{When prediction is worse than mean model}
Example: chr19, interval 936 \\
 $\rho_{p, RF} = -0.0619$ \hspace{1.5in} $\rho_{p, GBM} = -0.0066$
\begin{figure}
\includegraphics[page=3, scale=0.3]{chr19_936_0_predictedBySeed_RFRun03.pdf}
\includegraphics[page=1, scale=0.3]{chr19_936_0_predictedBySeed_GBMRun01.pdf}
\end{figure}
\end{frame}

%%%%%%%%%%%%%%%%%%%%%% SLIDE 10 %%%%%%%%%%%%%%%%%%%%%%
\begin{frame}
\frametitle{When prediction is worse than mean model}
Example: chr19, interval 936 \\
\begin{figure}
\includegraphics[page=3, scale=0.3]{chr19_936_0_predictedBySeed_RFRun03.pdf}
\includegraphics[page=8, scale=0.3]{chr19_frag936_0_registered.pdf}
\end{figure}

\end{frame}

%%%%%%%%%%%%%%%%%%%%%% SLIDE 11 %%%%%%%%%%%%%%%%%%%%%%
\begin{frame}
\frametitle{Application}
\begin{itemize}
\item Could we use $\gamma$ as a measure of prediction improvement?
\vspace{1cm}
\item When $\rho_{p} < -0.5$, could this help us detect structural variation? How small?
\end{itemize}
\end{frame}

%%%%%%%%%%%%%%%%%%%%%% SLIDE 12 %%%%%%%%%%%%%%%%%%%%%%
\begin{frame}
\frametitle{Fluoroscanning Workflow}
\vspace{-0.5cm}
\begin{figure}
\includegraphics[scale=0.4]{FluoroscanningWorkflow_v2.pdf}
\end{figure}

\end{frame}

\end{document}

