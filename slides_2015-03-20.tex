%\documentclass[10pt,dvipsnames,table, handout]{beamer} % To printout the slides without the animations
\documentclass[10pt,dvipsnames,table]{beamer} 
%\usetheme{Luebeck} 
\usetheme{Madrid} 
%\usetheme{Marburg} 
\setbeamercolor{structure}{fg=cyan!90!white}
\setbeamercolor{normal text}{fg=white, bg=black}

%%%%%%%%%%%%%%%%%%%%%%%% Packages %%%%%%%%%%%%%%%%%%%%%%%%
\usepackage{amscd}
\usepackage{amsmath}
\usepackage{amssymb}
\usepackage{amsthm}
\usepackage{amsxtra}
\usepackage{bbold}
%\usepackage{bigints}
\usepackage{color}
\usepackage{dsfont}
\usepackage{enumerate}
\usepackage[mathscr]{eucal}
%\usepackage{fancyhdr}
\usepackage{float}
%\usepackage{fullpage} %% Dont use this for beamer presentations
\usepackage{geometry}
\usepackage{graphicx}
\usepackage{hyperref}
\usepackage{indentfirst}
\usepackage{latexsym}
\usepackage{listings}
\usepackage{lscape}
\usepackage{mathtools}
\usepackage{microtype}
\usepackage{natbib}
\usepackage{pdfpages}
\usepackage{verbatim}
\usepackage{wrapfig}
\usepackage{xargs}
\usepackage{xcolor}
\DeclareGraphicsExtensions{.pdf,.png,.jpg, .jpeg}

%%%%%%%%%%%%%%%%%%%%%%%% Commands %%%%%%%%%%%%%%%%%%%%%%%%
\newcommand{\Sup}{\textsuperscript}
\newcommand{\Exp}{\mathds{E}}
\newcommand{\Prob}{\mathds{P}}
\newcommand{\Z}{\mathds{Z}}
\newcommand{\Ind}{\mathds{1}}
\newcommand{\A}{\mathcal{A}}
\newcommand{\F}{\mathcal{F}}
\newcommand{\G}{\mathcal{G}}
\newcommand{\I}{\mathcal{I}}
\newcommand{\be}{\begin{equation}}
\newcommand{\ee}{\end{equation}}
\newcommand{\bes}{\begin{equation*}}
\newcommand{\ees}{\end{equation*}}
\newcommand{\union}{\bigcup}
\newcommand{\intersect}{\bigcap}
\newcommand{\Ybar}{\overline{Y}}
\newcommand{\ybar}{\bar{y}}
\newcommand{\Xbar}{\overline{X}}
\newcommand{\xbar}{\bar{x}}
\newcommand{\betahat}{\hat{\beta}}
\newcommand{\Yhat}{\widehat{Y}}
\newcommand{\yhat}{\hat{y}}
\newcommand{\Xhat}{\widehat{X}}
\newcommand{\xhat}{\hat{x}}
\newcommand{\E}[1]{\operatorname{E}\left[ #1 \right]}
%\newcommand{\Var}[1]{\operatorname{Var}\left( #1 \right)}
\newcommand{\Var}{\operatorname{Var}}
\newcommand{\Cov}[2]{\operatorname{Cov}\left( #1,#2 \right)}
\newcommand{\N}[2][1=\mu, 2=\sigma^2]{\operatorname{N}\left( #1,#2 \right)}
\newcommand{\bp}[1]{\left( #1 \right)}
\newcommand{\bsb}[1]{\left[ #1 \right]}
\newcommand{\bcb}[1]{\left\{ #1 \right\}}
\newcommand*{\permcomb}[4][0mu]{{{}^{#3}\mkern#1#2_{#4}}}
\newcommand*{\perm}[1][-3mu]{\permcomb[#1]{P}}
\newcommand*{\comb}[1][-1mu]{\permcomb[#1]{C}}

%%%%%%%%%%%%% For explanatory bubbles, use the following code %%%%%%%%%%%%%
%% \usepackage{tikz} %% For explanatory bubbles
%% \usepackage{xparse}
%% \usetikzlibrary{shapes.callouts,ocgx}

%% \newcommand{\tikzmark}[1]{\tikz[overlay,remember picture,baseline=0.5ex] \node (#1) {};}

%% % \explainword: #1= identifier to mark the word, #2 text
%% \NewDocumentCommand{\explainword}{r[] m}{
%%     \switchocg{#1}{#2}\tikzmark{#1}
%% }

%% \tikzset{my callout style/.style={
%%         draw,rectangle callout,anchor=pointer,callout relative pointer={(230:1cm)},
%%         rounded corners,align=center,text width=2cm,fill=cyan!20, 
%%     }
%% }

%% % \mycallout: #1 opacity style, #2 pointer base position, #3= text
%% \NewDocumentCommand{\mycallout}{O{opacity=0.8,text opacity=1} m m}{%
%% \begin{tikzpicture}[remember picture, overlay]
%%  \begin{scope}[ocg={ref=#2,status=invisible,name={#3}}]
%% \node[my callout style,#1]at (#2) {#3};
%% \end{scope}
%% \end{tikzpicture}
%% }
%%%%%%%%%%%%%%%%%%%%%%%%%%%%%%%%%%%%%%%%%%%%%%%%%%%%%%%%%%%%%%%%%

%%%%%%%%%%%%%%%%%%%%%%%% TITLE PAGE %%%%%%%%%%%%%%%%%%%%%%%%
\DeclarePairedDelimiter\ceil{\lceil}{\rceil}
\title[Status Update Mar '15]{Status Update Meeting}
\author{S. Nandi}
\institute[LMCG]{LMCG \\
 University of Wisconsin-Madison}
\date{March 20, 2015}

\begin{document}
\setlength{\baselineskip}{16truept}
\frame{\maketitle}

%%%%%%%%%%%% Slide 1 %%%%%%%%%%%%
\begin{frame}
\frametitle{Outline}
\begin{itemize}       
\item Ribosomal DNA 
\item Simulation results with new similarity matrix
\end{itemize}
\end{frame}

%%%%%%%%%%%% Slide 2 %%%%%%%%%%%%
\begin{frame}
\frametitle{Ribosomal DNA}
\begin{center}
\includegraphics[page=1, scale=0.45]{RibosomalDNA_2015-03-19.pdf} 
\end{center}
\end{frame}

%%%%%%%%%%%% Slide 3 %%%%%%%%%%%%
\begin{frame}
%\frametitle{Ribosomal DNA, with Punctates}
\begin{center}
\includegraphics[page=2, scale=0.5]{RibosomalDNA_2015-03-19.pdf} 
\end{center}
\end{frame}

%%%%%%%%%%%% Slide 4 %%%%%%%%%%%%
\begin{frame}
\[ \rho(f_i, f_j) = \frac{\int _{S_{ij}}f'_i(s)f'_j(s) ds}{\int _{S_{ij}}f'_i(s)^2 ds \int _{S_{ij}}f'_j(s)^2 ds} \] \\
\begin{center}
\includegraphics[page=1, scale=0.2]{2curves_Sim_0_4994.pdf}
\includegraphics[page=1, scale=0.2]{2curves_Sim_0_5986.pdf}
\includegraphics[page=1, scale=0.2]{2curves_Sim_0_6732.pdf} \\
\includegraphics[page=1, scale=0.2]{2curves_Sim_0_7463.pdf}
\includegraphics[page=1, scale=0.2]{2curves_Sim_0_9569.pdf}
\includegraphics[page=1, scale=0.2]{2curves_Sim_1.pdf}
\end{center}
\end{frame}

%%%%%%%%%%%% Slide 5 %%%%%%%%%%%%
\begin{frame}
\frametitle{Simulation - True Curves}
\begin{center}
\includegraphics[page=1, scale=0.2]{6curves_Sim_0_7308.pdf}
\includegraphics[page=1, scale=0.2]{6curves_Sim_0_8775.pdf}
\includegraphics[page=1, scale=0.2]{6curves_Sim_0_9096.pdf} \\
\includegraphics[page=1, scale=0.2]{6curves_Sim_0_948.pdf}
\includegraphics[page=1, scale=0.2]{6curves_Sim_0_9796.pdf}
\includegraphics[page=1, scale=0.2]{6curves_Sim_0_9952.pdf}
\end{center}
\end{frame}

%%%%%%%%%%%% Slide 6 %%%%%%%%%%%%
\begin{frame}
\frametitle{Simulation - Cycle}
\begin{center}
\includegraphics[page=1, scale=0.23]{Allcurves_Sim_0_984.pdf}
\includegraphics[page=2, scale=0.23]{Allcurves_Sim_0_984.pdf} \\
\includegraphics[page=3, scale=0.23]{Allcurves_Sim_0_984.pdf}
\includegraphics[page=4, scale=0.23]{Allcurves_Sim_0_984.pdf}
\end{center}
\end{frame}

%%%%%%%%%%%% Slide 7 %%%%%%%%%%%%
\begin{frame}
\frametitle{Simulation - Cycle}
\begin{center}
\includegraphics[page=1, scale=0.23]{Allcurves_Sim_0_9835.pdf}
\includegraphics[page=2, scale=0.23]{Allcurves_Sim_0_9835.pdf} \\
\includegraphics[page=3, scale=0.23]{Allcurves_Sim_0_9835.pdf}
\includegraphics[page=4, scale=0.23]{Allcurves_Sim_0_9835.pdf}
\end{center}
\end{frame}

%%%%%%%%%%%% Slide 7a %%%%%%%%%%%%
\begin{frame}
\frametitle{Simulation - 20 curves}
\begin{center}
\includegraphics[page=1, scale=0.23]{Allcurves_Sim_0_9885.pdf}
\includegraphics[page=2, scale=0.23]{Allcurves_Sim_0_9885.pdf} \\
\includegraphics[page=3, scale=0.23]{Allcurves_Sim_0_9885.pdf}
\includegraphics[page=4, scale=0.23]{Allcurves_Sim_0_9885.pdf}
\end{center}
\end{frame}

%%%%%%%%%%%% Slide 8 %%%%%%%%%%%%
\begin{frame}
\frametitle{Warping Similarity vs Optimal Iteration}
\begin{center}
\includegraphics[page=1, scale=0.45]{BoxPlot_SimVsIter_2015-03-21.pdf} 
\end{center}
\end{frame}

%%%%%%%%%%%% Slide 9 %%%%%%%%%%%%
\begin{frame}
\frametitle{Warping Similarity vs Optimal Iteration}
\begin{center}
\includegraphics[page=2, scale=0.45]{BoxPlot_SimVsIter_2015-03-21.pdf} 
\end{center}
\end{frame}

%%%%%%%%%%%% Slide 10 %%%%%%%%%%%%
\begin{frame}
\frametitle{Warping Similarity vs Optimal Iteration}
\begin{center}
\includegraphics[page=1, scale=0.3]{BoxPlot_SimVsIter_2015-03-21.pdf} 
\includegraphics[page=2, scale=0.3]{BoxPlot_SimVsIter_2015-03-21.pdf} 
\end{center}
\end{frame}

%%%%%%%%%%%% Slide 10 %%%%%%%%%%%%
\begin{frame}
\frametitle{Next steps}
\begin{itemize}
\item Try out Sangalli et al's simultaneous clustering and registration
\item Try on chr 13 intervals
\item Use the similarity distance metric to discard Nmaps with low average pairwise distance
\end{itemize}
\end{frame}
\end{document}

