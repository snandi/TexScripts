\documentclass[11pt]{article}
%\usepackage{fancyhdr}
\usepackage[hmargin=3cm,vmargin=3cm]{geometry}
\usepackage{dsfont}
\usepackage{bbold}
\usepackage{graphicx, float}
\usepackage{verbatim}
\usepackage{amssymb, amsmath}
%\usepackage[hmargin=3cm,vmargin=3cm]{geometry}
\usepackage{wrapfig}
\usepackage{xcolor}
\DeclareGraphicsExtensions{.pdf,.png,.jpg, .jpeg}

\begin{document}

\title{Instructions for using LMCG code libraries to access necessary data}
\author{Subhrangshu Nandi\\
  Department of Statistics\\
  Laboratory of Molecular and Computational Genomics\\
  nandi@stat.wisc.edu}
%\date{July 13, 2013}
\maketitle

\begin{enumerate}
\item
In general, all perl libraries written by Steve, will be in the folder \begin{verbatim} /home/steveg/bin \end{verbatim}
\item
To extract data on nanocoding, particularly, for the GC-Intensity project, go to 
\begin{verbatim} 
/home/steveg/bin/nMaps/GC_content/ 
\end{verbatim} 
Use the file \texttt{alignmentLocations.pl} to produce tables like 
\begin{verbatim} 
/aspen/steveg/human_nMaps/GC_content/alignmentChunks.withLength.all7134Groups.goldOnly 
\end{verbatim} 
which look like
\begin{verbatim}
refID refStartIndex refEndIndex opID opStartIndex opEndIndex refStartCoor 
refEndCoord opStartCoor opEndCoord orient lengthRatio

chr1 1 2 2388301_734_2060415 18 17 17621 19740 121536 119548 -1 1.08472434266327
chr1 1 2 2393020_734_2060368 15 14 17621 19740 113319 111465 -1 1.04471447859582
chr1 1 2 2396217_734_2060303 0 1 17621 19740 0 2060 1 0.95170660205211
chr1 1 2 2398509_734_2060043 0 1 17621 19740 0 2266 1 0.941316312241508
chr1 2 3 2387802_734_2061088 16 15 19740 28925 106350 99103 -1 1.0932487070992
chr1 2 3 2388301_734_2060415 17 16 19740 28925 119548 111623 -1 1.08472434266327
chr1 2 3 2388598_734_2060536 14 13 19740 28925 112074 102403 -1 1.03741278084123
chr1 2 3 2393020_734_2060368 14 13 19740 28925 111465 103128 -1 1.04471447859582
chr1 2 3 2396217_734_2060303 1 2 19740 28925 2060 11742 1 0.95170660205211
chr1 2 3 2398509_734_2060043 1 2 19740 28925 2266 11742 1 0.941316312241508
\end{verbatim}
\item
To list intensities at each pixel, for any fragment, use the file \texttt{fluorIntensityFor\_nMaps.byPixel.pl}. This script parses the mol\_N.txt files and produces a table like the one named \begin{verbatim} pixelIntensities.eachFragment.all7134Groups \end{verbatim}
which for a fragment with 4 pixels looks like
\begin{verbatim}
moleculeID      fragmentIndex   fragmentLength  pixelIntensities
2327101_734_2060000     0       3.708   24211   23480   23352   23029
\end{verbatim}
\item
To create the dataset to be read into R, do the following:\\
\begin{verbatim}
./intensityForAlignedFrags.pl -ch 1 -f 2 -a 
alignmentChunks.withLength.all7134Groups.goldOnly 
-nM pixelIntensities.eachFragment.all7134Groups
\end{verbatim}
To be more specific, the following is the complete command, including the correct folder paths
\begin{verbatim}
intensityForAlignedFrags.pl -ch 3 -f 13762 -a 
/aspen/steveg/human_nMaps/GC_content/alignmentChunks.withLength.all7134Groups.goldOnly 
-nM /aspen/steveg/human_nMaps/GC_content/subdivideFragments/
pixelIntensities.eachFragment.all7134Groups > chr3_frag13762_intensities
\end{verbatim}

\item
The human build 37 reference genome is located at \begin{verbatim} /omm/data/sequence/human_wchr-b37/Human_wchr-b37-all.fa \end{verbatim}

\item
In order to extract sequence data for a given window of nucleotide locations using the perl script \begin{verbatim} getSequenceForLocs.pl \end{verbatim} 
is done by the following command:
\begin{verbatim} cat /omm/data/sequence/human_wchr-b37/Human_wchr-b37-all.fa | 
getSequenceForLocs.pl -locFile my-loctrac > output.fa 
\end{verbatim}

\item
In the folder \begin{verbatim} ~/Project_GC_Content/RData/ \end{verbatim}, 
the files \begin{verbatim} chr*_fragIndexList_Min20 \end{verbatim} 
contains the list of fragments in that chromosome that have at least 20 molecules aligned to them, in the reference. These files are created by \begin{verbatim} RScript11_AlignmentChunks.R \end{verbatim}. 

\item
The above fragIndexList files are used by the shell script \begin{verbatim} _generateExtractionCommandFile.sh \end{verbatim}
to use the syntax mentioned in Step 4 to get the intensities of all the molecules aligned to a particular fragment index location. These fragment index locations are read in from the fragIndexList files created by R. This shell script creates a shell script for each chromosome, containing the commands for extracting intensities for one fragment per line. For example, these shell scripts are named \begin{verbatim} _extractIntensities_chr3.sh \end{verbatim}.

\item
{\bf{Instructions to create the MFlorum data}}\\
\begin{itemize}
\item
Before alignment, create a mapset file, either with one conversion factor or with multiple conversion factors. The syntax for creating this mapset file is:
\begin{verbatim}
cd /aspen/nandi/MF_cap348/maps_inca34/
cat group1-*-inca34-outputs/FragmentLengthsApprox3.maps | 
makeMapsetFromPixel_Steve.pl 
-factor 209 -run 734 -minSize 10 -minFrags 5 
> MF_cap348_inca34_cf209_minSize10_minFrag5.maps
\end{verbatim}
This mapset file will serve as the input to alignment.
\item
Edit the soma parameter file, with appropriate inputfile name (the mapset file created in the above set), the reference filename (mesoplasm\_ref.map) and the output filename (the xml output, to be read by subsequent scripts), and appropriate parameters, and do an alignment. The command for running an alignment (with soma parameter filename soma.param\_cf209) is:
\begin{verbatim}
soma_align.LINUX soma.param_cf209
\end{verbatim}
\item
To view the result of the alignment in genspect, which reads in the xml output of the previous step, run the following command:
\begin{verbatim}
genspectx MF_cap348_inca34_cf209_minSize10_minFrag5_Aligned.xml
\end{verbatim}
\item
To see some basic statistics before and after alignment, use the command 
\begin{verbatim}
mass3 inputmapset.maps, or
mass3 alignedmapset.maps
\end{verbatim}
To produce the alignedmapset.maps file, use the following shellscript:
\begin{verbatim}
GetAlignedMaps.sh -f AlignmentOutput.xml > AlignmentOutput.maps
\end{verbatim}

\item
The MFlorum alignment data are in the folders
\begin{verbatim}
/aspen/nandi/MF_asp74   and   /aspen/nandi/MF_cap348
\end{verbatim}
\item
All data extracted from these alignment data will be in 
\begin{verbatim}
/home/nandi/mflorum_nMaps/GC_Content
\end{verbatim}
Filenames and folder structures will be same as those of the human genome data. 
\item
To create the file {\it{pixelIntensities.eachFragment.asp74\_inca34}}, use the modified version of Steve's perl script: 
\begin{verbatim}
fluorIntensityFor_nMaps.byPixel_forMFlorum.pl > 
                             pixelIntensities.eachFragment.asp74_inca34
\end{verbatim}
The above script looks into the molecule.txt files and extracts the intensity numbers.
Use appropriate directory in line 31 of this perl script to extract data from asp or cap mount, inca 34 or 36.
\item
The ``alignmentChunks'' file is produced by the following command:
\begin{verbatim}
alignmentLocations.pl 5 MF_cap348_inca34_cf209_minSize50_minFrag5_Aligned.xml > 
		MF_cap348_inca34_cf209_minSize50_minFrag5_Aligned.alignmentChunks
\end{verbatim}
The first argument is the minimum number of fragments and the second argument is a list of xml files of the alignment. 
\item
Below is an example to produce the final dataset for each fragment, to be read in by the R script:
\begin{verbatim}
intensityForAlignedFrags.pl -ch 1 -f 11 -a 
/aspen/nandi/MF_cap348/maps_inca34/
MF_cap348_inca34_cf209_minSize50_minFrag5_Aligned.alignmentChunks 
-nM ~/mflorum_nMaps/GC_Content/pixelIntensities.eachFragment.cap348_inca34 
> ~/Project_GC_Content/RData/mflorum/chr1_frag11_intensities
\end{verbatim}
These lines have been scripted into a shell script named {\bf{createFinalIntensityFiles.sh}} in the folder 
\begin{verbatim} 
~/Project_GC_Content/RData/mflorum/ 
\end{verbatim}. 
Modify this file with the appropriate input folder paths and output file names and the intensity files for each fragment will be created. These files will be ready to be read by the RScript, like RScript13\_2 or RScript16\_2. 
\end{itemize}

\item
{\bf{Instructions to extract and plot LINE segments}}
\begin{itemize}
\item 
\emph{Get Repeats data:} \\ 
Go to the UCSC Genome Browser, Tools, Table Browser. \\
(a) For ``region'', check ``position'', \\
(b) For ``groups'', choose ``repeats'', \\
(c) For ``Track'', choose ``RepeatMaster'', and \\
(d) enter a chromosome number and click on ``get output''. \\ 
This would produce an text file like 
\begin{verbatim}
/home/nandi/Project_GC_Content/LINE/Chr11_Repeats
\end{verbatim}
\item
\emph{Get LINE coordinates:} \\ 
Run a simple shell command to extract the rows from the Repeats table that pertain only to LINE segments.
\begin{verbatim}
grep LINE Chr11_Repeats > Chr11_LINE
\end{verbatim}
\item
\emph{Create loctrac file:} \\ 
Use the dataset produced above to create loctracs, preferably for particular families, by running 
\begin{verbatim} 
RScript17-1_LINE_Seq_L1.R 
\end{verbatim}. 
An example of the options for creating a loctrac file for all L1 LINE segments in Chr17, that are at least 4KB, is:
\begin{verbatim}
Chr <- `Chr17'
ChrNum <- 17
RepFamily <- `L1'
RepName <- `L1PA4'
SeqLenStart <- 4000
SeqLenEnd <- 7000
\end{verbatim}
\item
\emph{Get Sequences:} \\ 
Run the following perl script to extract the sequence of the L1 LINE segments, in the loctrac file created above:
\begin{verbatim}
cat /omm/data/sequence/human_wchr-b37/Human_wchr-b37-all.fa | 
getSequenceForLocs.pl -locFile Chr17_L1_loctrac > L1_Chr17.fa
\end{verbatim}
\item
\emph{Split the sequences by LINE locations:} \\
Use the following perl command (by Steve) to split the big sequence file: 
\begin{verbatim}
perl -nle 'BEGIN{$i=0} if (/^>/) 
{  close OUT; $i++; ($fileName = $ARGV)=~s/\.fa//; open OUT, 
">$fileName.sequence$i.fa";} print OUT'  L1_Chr17.fa
\end{verbatim}
This command will produce files named \begin{verbatim} L1_Chr17.sequence*.fa \end{verbatim}
\item
\emph{Plot the LINE sequences in R:} \\
Use the datasets produced above to plot the sequences, by running 
\begin{verbatim} 
RScript17-0_LINE_Seq.R 
\end{verbatim}. 
\end{itemize}

\item
{\bf{Data extraction - mm52 (winter vacation, 2015)}}
\begin{itemize}
\item \emph{Creating the bploc file} \\
Extracted columns 1, 2, 7 and 8 from the file 
\begin{verbatim}
alignmentChunks.withLength.all7134Groups.goldOnly
\end{verbatim}
in the folder
\begin{verbatim}
/aspen/steveg/human_nMaps/GC_content
\end{verbatim}
and created the file 
\begin{verbatim}
~/human_nMaps/GC_Content/mm52_all7431.goldOnly.bploc
\end{verbatim} 
by keeping only the unique rows. This file will be read in by RScripts to give the beginning and ending base pair locations for any chromosome and any fragment.
\item \emph{Creating data for 1 or 3 pixels around backbone} \\
There are a total $7,341$ groups. Extracted the list of groups into the file ``Groups\_mm52''. Then divided this file into smaller files with 200 groups in each file. This division was done by the scripts 
\begin{verbatim}
_createSplitGroup.sh,  _splitGroups.sh
\end{verbatim}
in the folder 
\begin{verbatim}
/aspen/nandi/mm52-all7341/maps_inca34_3pixel/
\end{verbatim}
Parallel instances of the python program 
\begin{verbatim}
~/PythonProgramming/P05_readMoleculetxt.py
\end{verbatim}
were run on different servers, by the following shell script, to convert the 5 pixel data to 3 and 1 pixel data, using the following command:
\begin{verbatim}
cd ~/PythonProgramming
$nohup ./_loopthroughGroups_mm52_3px.sh Groups_mm52_1To200 &
\end{verbatim}
\item \emph{Alignmentchunks file} \\
The alignment chunks file has nothing to do with the number of pixels around the backbone. Hence, the same file can be used for 1, 3 and 5 pixels around the backbone. The symbolic link 
\begin{verbatim}
alignmentChunks.withLength.all7134Groups.goldOnly
\end{verbatim}
links to the right file in Steve's folder on /aspen.
\item \emph{pixelintensities file} \\
Change the folder path in the ``fluorIntensityFor\_nMaps.byPixel.pl'' script and run it to produce the pixelIntensities file:
\begin{verbatim}
./fluorIntensityFor_nMaps.byPixel_1pixel.pl > pixelIntensities.all7134Groups_1pixel
\end{verbatim}
Do this for 1 and 3 pixel around the backbone separately. 
\item \emph{Molecules towards the edges of frames} \\
Molecules that are close to the edge of the frames do not have intensity numbers in the files produced by INCA3.4. A check was added in the python scripts to discard these molecules. There were a series of shell scripts written to identify the groups that have molecules with this problem and their conversion was done again. The scripts can be found in:
\begin{verbatim}
/aspen/nandi/mm52-all7341/Groups_Redo/
\end{verbatim}
\end{itemize}

\item
{\bf{Ribosomal DNA data extraction \& analysis}}
\begin{itemize}
\item Directory location: All files are located in the directory:
\begin{verbatim}
/aspen/nandi/mm52-all7341/RiboDNA
\end{verbatim}

\item \emph{Ribosomal DNA repeat unit} \\
This sequence was provided by Dave, and the filename is 
\begin{verbatim}
rDNA_repeatUnit.fa
\end{verbatim}
It was concatenated 10 and 20 times to create longer contigs, for aligning as many Nmaps as possible. These filenames are 
\begin{verbatim}
rDNA_repeatUnit_X10.fa, rDNA_repeatUnit_X20.fa
\end{verbatim}

\item \emph{Create in silico reference maps} \\
The following command was run by Mike Place, to create a reference map of the ribosomal DNA sequence:
\begin{verbatim}
javarun.sh edu.wisc.lmcg.seq.flash.CreatePMap -f rDNA_repeatUnit_X20.fa 
-enzyme ntsapi -label UTPF2 -resmap rDNA_X20.maps
\end{verbatim}

\item \emph{Aligning sequence to in silico reference} \\
Created the soma parameter file from some examples recommended by Mike. The examples are saved as symbolic links. The file I created is named 
\begin{verbatim}
soma.param_RiboDNA
\end{verbatim}
Use the above parameter file to align the rDNA sequence to the {\emph{in silico}} reference
\begin{verbatim}
soma_align.LINUX soma.param_RiboDNA
\end{verbatim}

\end{itemize}
\end{enumerate}
\end{document}
