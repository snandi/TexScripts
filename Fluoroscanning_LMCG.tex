\documentclass[11pt]{extarticle} %extarticle for fontsizes other than 10, 11 And 12
%\documentclass[11p]{article}

%%%%%%%%%%%%%%%%%%%%%%%% Packages %%%%%%%%%%%%%%%%%%%%%%%%
\usepackage{amscd}
\usepackage{amsmath}
\usepackage{amssymb}
\usepackage{amsthm}
\usepackage{amsxtra}
\usepackage{bbold}
%\usepackage{bigints}
\usepackage{color}
\usepackage{dsfont}
\usepackage{enumerate}
\usepackage[mathscr]{eucal}
%\usepackage{fancyhdr}
\usepackage{float}
%\usepackage{fullpage} %% Dont use this for beamer presentations
\usepackage{geometry}
\usepackage{graphicx}
\usepackage{hyperref}
\usepackage{indentfirst}
\usepackage{latexsym}
\usepackage{listings}
\usepackage{lscape}
\usepackage{mathtools}
\usepackage{microtype}
\usepackage{natbib}
\usepackage{pdfpages}
\usepackage{verbatim}
\usepackage{wrapfig}
\usepackage{xargs}
\DeclareGraphicsExtensions{.pdf,.png,.jpg, .jpeg}

%%%%%%%%%%%%%%%%%%%%%%%% Commands %%%%%%%%%%%%%%%%%%%%%%%%
\newcommand{\Sup}{\textsuperscript}
\newcommand{\Exp}{\mathds{E}}
\newcommand{\Prob}{\mathds{P}}
\newcommand{\Z}{\mathds{Z}}
\newcommand{\Ind}{\mathds{1}}
\newcommand{\A}{\mathcal{A}}
\newcommand{\F}{\mathcal{F}}
\newcommand{\G}{\mathcal{G}}
\newcommand{\I}{\mathcal{I}}
\newcommand{\be}{\begin{equation}}
\newcommand{\ee}{\end{equation}}
\newcommand{\bes}{\begin{equation*}}
\newcommand{\ees}{\end{equation*}}
\newcommand{\union}{\bigcup}
\newcommand{\intersect}{\bigcap}
\newcommand{\Ybar}{\overline{Y}}
\newcommand{\ybar}{\bar{y}}
\newcommand{\Xbar}{\overline{X}}
\newcommand{\xbar}{\bar{x}}
\newcommand{\betahat}{\hat{\beta}}
\newcommand{\Yhat}{\widehat{Y}}
\newcommand{\yhat}{\hat{y}}
\newcommand{\Xhat}{\widehat{X}}
\newcommand{\xhat}{\hat{x}}
\newcommand{\E}[1]{\operatorname{E}\left[ #1 \right]}
%\newcommand{\Var}[1]{\operatorname{Var}\left( #1 \right)}
\newcommand{\Var}{\operatorname{Var}}
\newcommand{\Cov}[2]{\operatorname{Cov}\left( #1,#2 \right)}
\newcommand{\N}[2][1=\mu, 2=\sigma^2]{\operatorname{N}\left( #1,#2 \right)}
\newcommand{\bp}[1]{\left( #1 \right)}
\newcommand{\bsb}[1]{\left[ #1 \right]}
\newcommand{\bcb}[1]{\left\{ #1 \right\}}
\newcommand*{\permcomb}[4][0mu]{{{}^{#3}\mkern#1#2_{#4}}}
\newcommand*{\perm}[1][-3mu]{\permcomb[#1]{P}}
\newcommand*{\comb}[1][-1mu]{\permcomb[#1]{C}}

%%%%%%%%%%%%%%%%%%% To change the margins and stuff %%%%%%%%%%%%%%%%%%%
\geometry{left=1in, right=1in, top=1in, bottom=0.8in}
%\setlength{\voffset}{0.5in}
%\setlength{\hoffset}{-0.4in}
%\setlength{\textwidth}{7.6in}
%\setlength{\textheight}{10in}
%%%%%%%%%%%%%%%%%%%%%%%%%%%%%%%%%%%%%%%%%%%%%%%%%%%%%%%%%%%%%%%%%%%%%%%

\begin{document}
%\SweaveOpts{concordance=TRUE}
\bibliographystyle{plain}  %Choose a bibliograhpic style

\title{Nanocoding and Fluoroscanning: Next Generation Precision Genomics}
\author{Subhrangshu Nandi\\
  Statistics PhD Student, \\
  Research Assistant,
  Laboratory of Molecular and Computational Genomics, \\
  University of Wisconsin - Madison}
\date{May 31, 2015}
%\date{}

\maketitle

%% \begin{center}
%% {\Large{Comparing different clustering techniques in analyzing gene-expression data}}\\
%% Subhrangshu Nandi\\
%% Stat 760: Multivariate Analysis\\
%% Project Proposal\\
%% November 25, 2014
%% \end{center}

\subsection*{Aim}
To leverage ``Nanocoding'' and ``Fluoroscanning'' to quickly and economically analyze polymorphisms and mutations in human genomes. 

\subsection*{Fluoroscanning}
A critical step in producing nanocoded maps (Nmaps) is adding the dye Oxazole Yellow (YOYO) to the DNA molecules and then capturing the images of the molecules stretched on the positively charged surfaces, once they fluoresce. The dye molecules intercalates between the bases of the DNA molecules. Fluorescence properties of YOYO depends strongly on DNA sequence. In fact, quantum yield of YOYO bound to “GC” rich sequences is almost twice as large as those bound to “AT” rich sequences; binding constants also vary with sequence composition. 



%\newpage
%\bibliography{bibTex_Reference}
%\bibliography{research}

\end{document}
