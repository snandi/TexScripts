\documentclass[11pt]{extarticle} %extarticle for fontsizes other than 10, 11 and 12
%\documentclass[11p]{article}
\usepackage{amsmath}
\usepackage{amsthm}
\usepackage{amssymb}
\usepackage{bbold}
\usepackage{dsfont}
\usepackage{enumerate}
%\usepackage{fancyhdr}
\usepackage{float}
\usepackage{fullpage}
\usepackage[vmargin=2cm, hmargin=2cm]{geometry}
\usepackage{graphicx}
\usepackage{lscape}
\usepackage{mathtools}
\usepackage{microtype}
\usepackage{pdfpages}
\usepackage{verbatim}
\usepackage{wrapfig}
\usepackage{xargs}
\DeclareGraphicsExtensions{.pdf,.png,.jpg, .jpeg}
\newcommand{\Sup}{\textsuperscript}
\newcommand{\Exp}{\mathds{E}}
\newcommand{\Prob}{\mathds{P}}
\newcommand{\Z}{\mathds{Z}}
\newcommand{\Ind}{\mathds{1}}
\newcommand{\A}{\mathcal{A}}
\newcommand{\F}{\mathcal{F}}
\newcommand{\G}{\mathcal{G}}
\newcommand{\be}{\begin{equation}}
\newcommand{\ee}{\end{equation}}
\newcommand{\bes}{\begin{equation*}}
\newcommand{\ees}{\end{equation*}}
\newcommand{\union}{\bigcup}
\newcommand{\intersect}{\bigcap}
\newcommand{\Ybar}{\overline{Y}}
\newcommand{\ybar}{\bar{y}}
\newcommand{\Xbar}{\overline{X}}
\newcommand{\xbar}{\bar{x}}
\newcommand{\betahat}{\hat{\beta}}
\newcommand{\Yhat}{\widehat{Y}}
\newcommand{\yhat}{\hat{y}}
\newcommand{\Xhat}{\widehat{X}}
\newcommand{\xhat}{\hat{x}}
\newcommand{\E}[1]{\operatorname{E}\left[ #1 \right]}
%\newcommand{\Var}[1]{\operatorname{Var}\left( #1 \right)}
\newcommand{\Var}{\operatorname{Var}}
\newcommand{\Cov}[2]{\operatorname{Cov}\left( #1,#2 \right)}
\newcommand{\N}[2][1=\mu, 2=\sigma^2]{\operatorname{N}\left( #1,#2 \right)}
\newcommand{\bp}[1]{\left( #1 \right)}
\newcommand{\bsb}[1]{\left[ #1 \right]}
\newcommand{\bcb}[1]{\left\{ #1 \right\}}
%\newcommand{\infint}{\int_{-\infty}^{\infty}}
%\usepackage{Sweave}
%%%%%%%%%%%%%%%%%%% To change the margins and stuff %%%%%%%%%%%%%%%%%%%
%\setlength{\voffset}{-0.5in}
%\setlength{\hoffset}{-0.4in}
%\setlength{\textwidth}{7.6in}
%\setlength{\textheight}{10in}
%%%%%%%%%%%%%%%%%%%%%%%%%%%%%%%%%%%%%%%%%%%%%%%%%%%%%%%%%%%%%%%%%%%%%%%

\begin{document}
%\SweaveOpts{concordance=TRUE}

\title{Statistical and supervised machine learning methods for single molecule nanocoding data to advance personalized genomics}
\author{Subhrangshu Nandi\\
%  Department of Statistics\\
%  nandi@stat.wisc.edu}
%\date{September 18, 2014}
\date{}
}
\maketitle

The Human Genome Project (HGP), completed in 2003, is considered one of the greatest accomplishments of exploration in history of science. Since then thousands of genomes have been sequenced. However, no individual human genome has been annotated to completion. Developed in the mid nineties, Optical Mapping, is a novel system for physical mapping genomes, using measurements of single DNA molecules to infer a high-resolution genome-wide restriction map, whose representation of genome structure complements genome sequences to yield biological insight. ``Nanocoding systems'' which were developed by Laboratory of Molecular and Computational Genomics (LMCG), UW Madison, is the next generation technology of Optical Mapping. Nanocoding is scalable, robust, reliable system that can be used for {\it{de novo}} sequence assembly of complex genomes. In addition, Nanocoding produces signals (Nmap signals) from each genomic interval which carry information about the sequences in them. Upon decoding and quantifying this relationship between these signals and sequence features, Nanocoding systems would become an essential system along with next generation sequencing quickly perform high-resolution genotyping that include profound structural variants across the entire genome. For my doctoral training, my focus has been, and will remain, on decoding the relationship between Nmap signals and sequence features, using appropriate statistical and machine learning tools. 

\end{document}
