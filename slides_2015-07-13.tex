%\documentclass[10pt,dvipsnames,table, handout]{beamer} % To printout the slides without the animations
\documentclass[10pt,dvipsnames,table]{beamer} 
%\usetheme{Luebeck} 
%\usetheme{Madrid} 
%\usetheme{Marburg} 
\usetheme{Warsaw} 
%\setbeamercolor{structure}{fg=cyan!90!white}
%\setbeamercolor{normal text}{fg=white, bg=black}

%%%%%%%%%%%%%%%%%%%%%%%% Packages %%%%%%%%%%%%%%%%%%%%%%%%
\usepackage{amscd}
\usepackage{amsmath}
\usepackage{amssymb}
\usepackage{amsthm}
\usepackage{amsxtra}
\usepackage{animate}
\usepackage{bbold}
%\usepackage{bigints}
\usepackage{color, colortbl}
\usepackage{dsfont}
\usepackage{enumerate}
\usepackage[mathscr]{eucal}
%\usepackage{fancyhdr}
\usepackage{float}
%\usepackage{fullpage} %% Dont use this for beamer presentations
\usepackage{geometry}
\usepackage{graphicx}
\usepackage{hyperref}
\usepackage{indentfirst}
\usepackage{latexsym}
\usepackage{listings}
\usepackage{lscape}
\usepackage{mathtools}
\usepackage{microtype}
\usepackage{multirow}
\usepackage{natbib}
\usepackage{pdfpages}
\usepackage{verbatim}
\usepackage{wrapfig}
\usepackage{xargs}
\usepackage{xcolor}
\DeclareGraphicsExtensions{.pdf,.png,.jpg, .jpeg}
\definecolor{LightCyan}{rgb}{0.88,1,1}

%%%%%%%%%%%%%%%%%%%%%%%% Commands %%%%%%%%%%%%%%%%%%%%%%%%
\newcommand{\Sup}{\textsuperscript}
\newcommand{\Exp}{\mathds{E}}
\newcommand{\Prob}{\mathds{P}}
\newcommand{\Z}{\mathds{Z}}
\newcommand{\Ind}{\mathds{1}}
\newcommand{\A}{\mathcal{A}}
\newcommand{\F}{\mathcal{F}}
\newcommand{\G}{\mathcal{G}}
\newcommand{\I}{\mathcal{I}}
\newcommand{\R}{\mathcal{R}}
\newcommand{\Real}{\mathbb{R}}
\newcommand{\be}{\begin{equation}}
\newcommand{\ee}{\end{equation}}
\newcommand{\bes}{\begin{equation*}}
\newcommand{\ees}{\end{equation*}}
\newcommand{\union}{\bigcup}
\newcommand{\intersect}{\bigcap}
\newcommand{\Ybar}{\overline{Y}}
\newcommand{\ybar}{\bar{y}}
\newcommand{\Xbar}{\overline{X}}
\newcommand{\xbar}{\bar{x}}
\newcommand{\betahat}{\hat{\beta}}
\newcommand{\Yhat}{\widehat{Y}}
\newcommand{\yhat}{\hat{y}}
\newcommand{\Xhat}{\widehat{X}}
\newcommand{\xhat}{\hat{x}}
\newcommand{\E}[1]{\operatorname{E}\left[ #1 \right]}
%\newcommand{\Var}[1]{\operatorname{Var}\left( #1 \right)}
\newcommand{\Var}{\operatorname{Var}}
\newcommand{\Cov}[2]{\operatorname{Cov}\left( #1,#2 \right)}
\newcommand{\N}[2][1=\mu, 2=\sigma^2]{\operatorname{N}\left( #1,#2 \right)}
\newcommand{\bp}[1]{\left( #1 \right)}
\newcommand{\bsb}[1]{\left[ #1 \right]}
\newcommand{\bcb}[1]{\left\{ #1 \right\}}
\newcommand*{\permcomb}[4][0mu]{{{}^{#3}\mkern#1#2_{#4}}}
\newcommand*{\perm}[1][-3mu]{\permcomb[#1]{P}}
\newcommand*{\comb}[1][-1mu]{\permcomb[#1]{C}}

%%%%%%%%%%%%%%%%%%%%%%%% TITLE PAGE %%%%%%%%%%%%%%%%%%%%%%%%
\DeclarePairedDelimiter\ceil{\lceil}{\rceil}
\title[Curve Registration (constrained warping)]{Iterated Curve Registration with Constrained Warping}
\author{Subhrangshu Nandi}
%\institute[Stat 741]{Stat 741, Spring 2015 \\
%  Department of Statistics \\
% University of Wisconsin-Madison}
%\date{April 21, 2015}

\begin{document}
\setlength{\baselineskip}{16truept}
\frame{\maketitle}

%The presentation must follow this outline.
%1. Registration details 
%2. Constrained warping
%3. Illustration of simulation example
%4. Iteration procedure
%5. 
%6. 
%7. 

%%%%%%%%%%%% Slide 1 %%%%%%%%%%%%
\begin{frame}
\frametitle{Registration details}
\begin{enumerate}
\item Let $x(t)$ be the true curve; $y(t)$ be the curve to register. 
\item Let $h(t)$ be the warping function; $y[h(t)]$ be the registered curve.
\item $y[h(t)]$ and and $x(t)$ differ only in terms of amplitude variation, i.e., their values are proportional to one another across the range of $t$
\item Then, PCA of the following order two matrix should reveal essentially one component (smallest eigen value $\approx 0$) \\
\[C(h) = 
\begin{bmatrix}
\int \{x(t) \}^2dt & \int x(t) y[h(t)]dt \vspace{0.5cm} \\ 
\int x(t) y[h(t)]dt & \int \{y[h(t)]\}^2dt
\end{bmatrix}
\]
\item Choose $h(t)$ that minimizes the eigen value
\end{enumerate}
\end{frame}
%%%%%%%%%%%%%%%%%%%%%%%%%%%%%%%%%

%%%%%%%%%%%% Slide 2 %%%%%%%%%%%%
\begin{frame}
\frametitle{Estimation of the warping function}
\begin{enumerate}
\item $h(t)$ is a strictly {\emph{monotone increasing}} function; Hence {\emph{invertible}}.
\item Let $w(t) = \frac{h''(t)}{h'(t)}$ be the relative acceleration of the warping function. \\
This will be used to {\emph{constrain}} the warping
\item The objective function is:
\[ F_{\lambda}(y,x|h) = \int \|y(t) - x\{h(t)\} \|^2 dt + \lambda \int w^2(t) dt \]
Penalizing $w(t)$ ensures smoothness and monotonicity
\item Express $w(t)$ in terms of B-Spline expansion: 
\[w(t) = \sum \limits_{k=0}^{K} c_k B_k(t)\]
Solve for the coefficients $c_k$
\end{enumerate}
\end{frame}
%%%%%%%%%%%%%%%%%%%%%%%%%%%%%%%%%

%%%%%%%%%%%% Slide 2 %%%%%%%%%%%%
\begin{frame}
\frametitle{Constrained warping, $\lambda = 1$}
\begin{center}
\includegraphics[scale=0.55, page=20]{Warpings_Seed10_Lambda_1.pdf}
\end{center}
\end{frame}

\begin{frame}
\frametitle{Constrained warping, $\lambda = 0.1$}
\begin{center}
\includegraphics[scale=0.55, page=20]{Warpings_Seed10_Lambda_1e-01.pdf}
\end{center}
\end{frame}

\begin{frame}
\frametitle{Constrained warping, $\lambda = 0.01$}
\begin{center}
\includegraphics[scale=0.55, page=20]{Warpings_Seed10_Lambda_1e-02.pdf}
\end{center}
\end{frame}

%%%%%%%%%%%% Slide 3 %%%%%%%%%%%%
\begin{frame}
\frametitle{Iterated Clustering \& Registration outline}
\begin{enumerate}
\item Based on ``pairwise distance'' between curves, form two initial clusters. 
\item Estimate mean/median of two clusters
\item Do constrained registration of each curve, to the mean that is closest to them
\item Use MDS to observe the results of the new cluster
\item Iterate, with a smaller $\lambda$, until no further improvement in the cluster distinction
\end{enumerate}
\end{frame}

%%%%%%%%%%%% Slide 4 %%%%%%%%%%%%
\begin{frame}
\frametitle{Illustration}
\begin{center}
\includegraphics[scale=0.32, page=21]{Warpings_Seed58_Lambda_1e-02.pdf}
\includegraphics[scale=0.32, page=23]{Warpings_Seed58_Lambda_1e-02.pdf}
\end{center}
\end{frame}

\begin{frame}
\frametitle{Illustration}
\begin{center}
\includegraphics[scale=0.32, page=22]{Warpings_Seed58_Lambda_1e-02.pdf}
\includegraphics[scale=0.32, page=24]{Warpings_Seed58_Lambda_1e-02.pdf}
\end{center}
\end{frame}

%%%%%%%%%%%% Slide 5 %%%%%%%%%%%%
\begin{frame}
\frametitle{Work with Vikas Singh, Sathya, Deepti}
\begin{enumerate}
\item Permutation Diffusion Maps
\begin{itemize}
\item Consistently matching landmarks across multiple images of the same object
\item Finding clusters of nearby images
\item Uses group structure of image to image relationships
\item Uses eigen vectors of smallest eigenvalues of graph Laplacian 
\item Loss function is a weighted Frobenius norm of pairwise permutation (or distance) matrix
\end{itemize}
\item Multiway matching by permutation synchronization
\begin{itemize}
\item Jointly find all matchings between two images
\item Uses pairwise warping information (noisy warping functions)
\item Computationally efficient algorithm - might obviate iterated registration (?)
\item Need difference distance metric for Nmaps (or any smooth curves)
\end{itemize}
\end{enumerate}
\end{frame}

%%%%%%%%%%%% Slide 6 %%%%%%%%%%%%
\begin{frame}
\frametitle{Next Steps}
\begin{itemize}
\item Work with Vikas and Sathya with simulation and real data
\item Finish JSM poster by 07/22
\item Simulate data, similar to what is observed in Nmaps (Bayesian?)
\item Complete iterative registration/clustering methodology on simulated data
\item Incorporate error structure from adjacent intervals, of same Nmap
\end{itemize}

\end{frame}

\end{document}

