%\documentclass[10pt,dvipsnames,table, handout]{beamer} % To printout the slides without the animations
\documentclass[10pt,dvipsnames,table]{beamer} 
%\usetheme{Luebeck} 
%\usetheme{Madrid} 
%\usetheme{Marburg} 
\usetheme{Warsaw} 
%\setbeamercolor{structure}{fg=cyan!90!white}
%\setbeamercolor{normal text}{fg=white, bg=black}

%%%%%%%%%%%%%%%%%%%%%%%%%%%%%%%%%%%%%%%%%%%%%%%%%%%%%%%%%%%%%%%%%%%%%%
%% Input header file 
%%%%%%%%%%%%%%%%%%%%%%%%%%%%%%%%%%%%%%%%%%%%%%%%%%%%%%%%%%%%%%%%%%%%%%
%%%%%%%%%%%%%%%%%%%%%%%% Packages %%%%%%%%%%%%%%%%%%%%%%%%
\usepackage{amscd}
\usepackage{amsmath}
\usepackage{amssymb}
\usepackage{amsthm}
\usepackage{amsxtra}
\usepackage{animate}
\usepackage{bbold}
%\usepackage{bigints}
\usepackage{color, colortbl}
\usepackage{dsfont}
\usepackage{enumerate}
\usepackage[mathscr]{eucal}
%\usepackage{fancyhdr}
\usepackage{float}
%\usepackage{fullpage} %% Dont use this for beamer presentations
\usepackage{geometry}
\usepackage{graphicx}
\usepackage{hyperref}
\usepackage{indentfirst}
\usepackage{latexsym}
\usepackage{listings}
\usepackage{lscape}
\usepackage{mathtools}
\usepackage{microtype}
\usepackage{multirow}
\usepackage{natbib}
\usepackage{pdfpages}
\usepackage{verbatim}
\usepackage{wrapfig}
\usepackage{xargs}
\usepackage{xcolor}
\DeclareGraphicsExtensions{.pdf,.png,.jpg, .jpeg}
\definecolor{LightCyan}{rgb}{0.88,1,1}

%%%%%%%%%%%%%%%%%%%%%%%% Commands %%%%%%%%%%%%%%%%%%%%%%%%
\newcommand{\Sup}{\textsuperscript}
\newcommand{\Exp}{\mathds{E}}
\newcommand{\Prob}{\mathds{P}}
\newcommand{\Z}{\mathds{Z}}
\newcommand{\Ind}{\mathds{1}}
\newcommand{\A}{\mathcal{A}}
\newcommand{\F}{\mathcal{F}}
\newcommand{\G}{\mathcal{G}}
\newcommand{\I}{\mathcal{I}}
\newcommand{\R}{\mathcal{R}}
\newcommand{\Real}{\mathbb{R}}
\newcommand{\be}{\begin{equation}}
\newcommand{\ee}{\end{equation}}
\newcommand{\bes}{\begin{equation*}}
\newcommand{\ees}{\end{equation*}}
\newcommand{\union}{\bigcup}
\newcommand{\intersect}{\bigcap}
\newcommand{\Ybar}{\overline{Y}}
\newcommand{\ybar}{\bar{y}}
\newcommand{\Xbar}{\overline{X}}
\newcommand{\xbar}{\bar{x}}
\newcommand{\betahat}{\hat{\beta}}
\newcommand{\Yhat}{\widehat{Y}}
\newcommand{\yhat}{\hat{y}}
\newcommand{\Xhat}{\widehat{X}}
\newcommand{\xhat}{\hat{x}}
\newcommand{\E}[1]{\operatorname{E}\left[ #1 \right]}
%\newcommand{\Var}[1]{\operatorname{Var}\left( #1 \right)}
\newcommand{\Var}{\operatorname{Var}}
\newcommand{\Cov}[2]{\operatorname{Cov}\left( #1,#2 \right)}
\newcommand{\N}[2][1=\mu, 2=\sigma^2]{\operatorname{N}\left( #1,#2 \right)}
\newcommand{\bp}[1]{\left( #1 \right)}
\newcommand{\bsb}[1]{\left[ #1 \right]}
\newcommand{\bcb}[1]{\left\{ #1 \right\}}
\newcommand*{\permcomb}[4][0mu]{{{}^{#3}\mkern#1#2_{#4}}}
\newcommand*{\perm}[1][-3mu]{\permcomb[#1]{P}}
\newcommand*{\comb}[1][-1mu]{\permcomb[#1]{C}}


%%%%%%%%%%%%%%%%%%%%%%%%%%%%%%%%%%%%%%%%%%%%%%%%%%%%%%%%%%%%%%%%%%%%%%
%% TITLE PAGE 
%%%%%%%%%%%%%%%%%%%%%%%%%%%%%%%%%%%%%%%%%%%%%%%%%%%%%%%%%%%%%%%%%%%%%%
\DeclarePairedDelimiter\ceil{\lceil}{\rceil}
\title[Uniqueness and Reproducibility of Nmaps]{Iterated Curve Registration with Constrained (Penalized) Warping}
\author{Subhrangshu Nandi}
%\institute[Stat 741]{Stat 741, Spring 2015 \\
%  Department of Statistics \\
% University of Wisconsin-Madison}
%\date{April 21, 2015}

\begin{document}
\setlength{\baselineskip}{16truept}
\frame{\maketitle}

%%%%%%%%%%%%%%%%%%%%%%%%%%%%%%%%%%%%%%%%%%%%%%%%%%%%%%%%%%%%%%%%%%%%%%
%% This is for 2015-07-31 meeting, with Prof. Schwartz & Prof. Newton
%% Updating them on statistical evidence that iterated registration 
%% works better than one-time registration, which is much better than
%% no registration
%%%%%%%%%%%%%%%%%%%%%%%%%%%%%%%%%%%%%%%%%%%%%%%%%%%%%%%%%%%%%%%%%%%%%%

%% Outline for this presentation:

%%%%%%%%%%%% Slide 1 %%%%%%%%%%%%
\begin{frame}
\frametitle{Goal}
{\bf{GOAL: Establish uniqueness and reproducibility of intensity profiles of Nmaps aligned to any locations on the genome}}
\end{frame}

%%%%%%%%%%%% Slide 2 %%%%%%%%%%%%
\begin{frame}
\frametitle{Estimation Steps ({\emph{M.florum}} data)}
\begin{enumerate}
\item Select the theoretical number of pixels for any length on the genome. Allow for $\pm 5\%$ stretch. 
\item Make all Nmaps (inside $\pm 5\%$ stretch) of the same length, {\textcolor{blue}{by fitting B-Splines}}
\item Pre-process the Nmaps to select the ones somewhat ``similar'' to each other, {\textcolor{blue}{by pairwise distance measures}}
\item Choose Nmaps that are not completely flat, {\textcolor{blue}{by choosing top 25\% or 30\% of most variable Nmaps, per interval/ fragment}}
\item Test for uniqueness, by comparing Nmaps from two intervals of same length, {\textcolor{blue}{by permutation test}}
\item Align the intensity profiles of Nmaps from the same interval, {\textcolor{blue}{by iterated penalized registration}}
\item Test for uniqueness again, {\textcolor{blue}{by permutation test}}
\end{enumerate}
\end{frame}

%%%%%%%%%%%% Slide 3 %%%%%%%%%%%%
\begin{frame}
\frametitle{Base pair to pixel conversion}
$\text{PixelLength} = \frac{\text{BasePairLength}}{\text{ConversionFactor} (209)} $
\begin{table}[ht]
\footnotesize
\centering
\begin{tabular}{r|r|r|r|r}
  \hline
 Frag & CoordStart & CoordEnd & BasePairLength & PixelLength \\
  \hline
  \hline
  0 &            0 &        81621 &        81621 &            391 \\
  1 &        81622 &       100297 &        18675 &             89 \\
  2 &       100298 &       159702 &        59404 &            284 \\
  3 &       159703 &       173646 &        13943 &             67 \\ 
  4 &       173647 &       182672 &         9025 &             43 \\
  5 &       182673 &       187715 &         5042 &             24 \\
  . &            . &            . &            . &              . \\
  . &            . &            . &            . &              . \\
\rowcolor{Gray}
 30 &       582668 &       619168 &        36500 &            175 \\
  . &            . &            . &            . &              . \\
  . &            . &            . &            . &              . \\
 38 &       775344 &       793223 &        17879 &             86 \\
 \hline
 \hline
\end{tabular}
\end{table}
\pause
So, for fragment 30, $175 \pm 5\% =(166, 184)$, Nmaps from 166 pixels to 184 pixels will be considered. \\
\pause
Take off 5 pixels from each end (punctates). Left with Nmaps between 156 and 174 pixels. 
\end{frame}

%%%%%%%%%%%% Slide 3 %%%%%%%%%%%%
\begin{frame}
\begin{figure}
\includegraphics[scale=0.52, page=1]{Plots_forMeeting.pdf}
\end{figure}
\end{frame}

%%%%%%%%%%%% Slide 4 %%%%%%%%%%%%
\begin{frame}
\begin{figure}
\includegraphics[scale=0.52, page=2]{Plots_forMeeting.pdf}
\end{figure}
\end{frame}

%%%%%%%%%%%% Slide 4 %%%%%%%%%%%%
\begin{frame}
\frametitle{B-Spline fitting}

\end{frame}

%%%%%%%%%%%% Slide 5 %%%%%%%%%%%%
\begin{frame}
\frametitle{All Nmaps aligned to fragment 30, within acceptable stretch limits}
\begin{figure}
\includegraphics[scale=0.43, page=3]{Plots_forMeeting.pdf}
\end{figure}
\end{frame}

%%%%%%%%%%%% Slide 6 %%%%%%%%%%%%
\begin{frame}
\frametitle{Pre-processing - Minimum variability}
Discarding Nmaps that are flat. Taking top $25\%$ of most variable Nmaps.
\begin{figure}
\includegraphics[scale=0.42, page=4]{Plots_forMeeting.pdf}
\end{figure}
\end{frame}

%%%%%%%%%%%% Slide 7 %%%%%%%%%%%%
\begin{frame}
\frametitle{Pre-processing - Similarity between Nmaps }
\[ \rho(f_i, f_j) = \frac{\int _{S_{ij}}f'_i(s)f'_j(s) ds}{\int _{S_{ij}}f'_i(s)^2 ds \int _{S_{ij}}f'_j(s)^2 ds} \]
where, $S_{ij} = S_i \cap S_j$, the intersection of Sobolev spaces formed by the functions $f_i, f_j$

%%% Add more materials here

\end{frame}

%%%%%%%%%%%% Slide 8 %%%%%%%%%%%%
\begin{frame}
\begin{center}
\includegraphics[page=1, scale=0.25]{2curves_Sim_0_4994.pdf}
\includegraphics[page=1, scale=0.25]{2curves_Sim_0_6732.pdf} \\
\includegraphics[page=1, scale=0.25]{2curves_Sim_0_9569.pdf}
\includegraphics[page=1, scale=0.25]{2curves_Sim_1.pdf}
\end{center}
\end{frame}

%%%%%%%%%%%% Slide 9 %%%%%%%%%%%%
\begin{frame}
\frametitle{Pairwise Similarity between Nmaps}
\begin{figure}
\includegraphics[scale=0.47, page=5]{Plots_forMeeting.pdf}
\end{figure}
\end{frame}

%%%%%%%%%%%% Slide 10 %%%%%%%%%%%%
\begin{frame}
\frametitle{Pairwise Similarity between Nmaps, after cutoff}
\begin{figure}
\includegraphics[scale=0.47, page=6]{Plots_forMeeting.pdf}
\end{figure}
\end{frame}

%%%%%%%%%%%% Slide 11 %%%%%%%%%%%%
\begin{frame}
\frametitle{Ready to register, establish uniqueness}
\begin{figure}
\includegraphics[scale=0.47, page=7]{Plots_forMeeting.pdf}
\end{figure}
\end{frame}

%%%%%%%%%%%% Slide 12 %%%%%%%%%%%%
\begin{frame}
\frametitle{Compare two sets of Nmaps}
\begin{center}
\begin{figure}
\includegraphics[scale=0.33, page=1]{Regist_MF_Frag_30_Pixels6To45_2015-07-30.pdf}
\pause
\includegraphics[scale=0.33, page=1]{Regist_MF_Frag_30_Pixels46To85_2015-07-30.pdf}
\end{figure}
\end{center}
{\bf{Question: How do you statistically determine if the two sets of Nmaps are unique to the group?}}
\end{frame}

%%%%%%%%%%%% Slide 12 %%%%%%%%%%%%
\begin{frame}
\frametitle{Permutation test - T-type test statistic}
\begin{enumerate}
\item $\bar{X}_1(t): $ Pointwise mean of Nmaps of interval $1$
\item $\bar{X}_2(t): $ Pointwise mean of Nmaps of interval $2$
\item $n_1, n_2$: Number of Nmaps in intervals 1 and 2
\item $\Var[x_1(t)]: $ Variance of Nmaps of interval 1 at each point $t$ 
\item $\Var[x_2(t)]: $ Variance of Nmaps of interval 2 at each point $t$ 
\item $T(t) = \frac{|\bar{X}_1(t) - \bar{X}_2(t)|}{\sqrt{\frac{1}{n_1}\Var[x_1(t)] + 
\frac{1}{n_2}\Var[x_2(t)]}}$: Absolute value of t-statistic at each point
\item $\bar{T}(t):$ Avg of absolute t-statistic is our Test statistic (measure of difference of two sets of Nmaps)
\item To get a null distribution, permute the Nmaps, and repeat steps 1 through 7
\end{enumerate}
\end{frame}

%%%%%%%%%%%% Slide 13 %%%%%%%%%%%%
\begin{frame}
\begin{figure}
\includegraphics[scale=0.52, page=9]{Plots_forMeeting.pdf}
\end{figure}
\end{frame}

%%%%%%%%%%%% Slide 14 %%%%%%%%%%%%
\begin{frame}
\begin{figure}
\includegraphics[scale=0.52, page=8]{Plots_forMeeting.pdf}
\end{figure}
\end{frame}

%%%%%%%%%%%% Slide 15 %%%%%%%%%%%%
\begin{frame}
\begin{figure}
\includegraphics[scale=0.25, page=1]{Regist_MF_Frag_30_Pixels6To45_2015-07-30.pdf}
\includegraphics[scale=0.25, page=1]{Regist_MF_Frag_30_Pixels46To85_2015-07-30.pdf} \\
\includegraphics[scale=0.25, page=8]{Plots_forMeeting.pdf}
\includegraphics[scale=0.25, page=9]{Plots_forMeeting.pdf}
\end{figure}
\end{frame}

%%%%%%%%%%%% Slide 16 %%%%%%%%%%%%
\begin{frame}
\frametitle{After 1 iteration}
\begin{figure}
\includegraphics[scale=0.33, page=2]{Regist_MF_Frag_30_Pixels6To45_2015-07-30.pdf}
\includegraphics[scale=0.33, page=2]{Regist_MF_Frag_30_Pixels46To85_2015-07-30.pdf} \\
\end{figure}
\end{frame}

%%%%%%%%%%%% Slide 17 %%%%%%%%%%%%
\begin{frame}
\begin{figure}
\includegraphics[scale=0.25, page=2]{Regist_MF_Frag_30_Pixels6To45_2015-07-30.pdf}
\includegraphics[scale=0.25, page=2]{Regist_MF_Frag_30_Pixels46To85_2015-07-30.pdf} \\
\includegraphics[scale=0.25, page=10]{Plots_forMeeting.pdf}
\includegraphics[scale=0.25, page=11]{Plots_forMeeting.pdf}
\end{figure}
\end{frame}

%%%%%%%%%%%% Slide 18 %%%%%%%%%%%%
\begin{frame}
\begin{figure}
\includegraphics[scale=0.25, page=8]{Regist_MF_Frag_30_Pixels6To45_2015-07-30.pdf}
\includegraphics[scale=0.25, page=8]{Regist_MF_Frag_30_Pixels46To85_2015-07-30.pdf} \\
\includegraphics[scale=0.25, page=14]{Plots_forMeeting.pdf}
\includegraphics[scale=0.25, page=15]{Plots_forMeeting.pdf}
\end{figure}
\end{frame}

%%%%%%%%%%%% Slide 19 %%%%%%%%%%%%
\begin{frame}
\frametitle{After 7 iterations}
\begin{figure}
\includegraphics[scale=0.33, page=20]{Regist_MF_Frag_30_Pixels6To45_2015-07-30.pdf}
\includegraphics[scale=0.33, page=20]{Regist_MF_Frag_30_Pixels46To85_2015-07-30.pdf} \\
\end{figure}
\end{frame}

%%%%%%%%%%%% Slide 20 %%%%%%%%%%%%
\begin{frame}
\begin{figure}
\includegraphics[scale=0.25, page=20]{Regist_MF_Frag_30_Pixels6To45_2015-07-30.pdf}
\includegraphics[scale=0.25, page=20]{Regist_MF_Frag_30_Pixels46To85_2015-07-30.pdf} \\
\includegraphics[scale=0.25, page=20]{Plots_forMeeting.pdf}
\includegraphics[scale=0.25, page=21]{Plots_forMeeting.pdf}
\end{figure}
\end{frame}

%%%%%%%%%%%% Slide 21 %%%%%%%%%%%%
\begin{frame}
\frametitle{After 7 iterations}
\begin{figure}
\includegraphics[scale=0.33, page=20]{Regist_MF_Frag_30_Pixels6To45_2015-07-30.pdf}
\includegraphics[scale=0.33, page=21]{Regist_MF_Frag_30_Pixels6To45_2015-07-30.pdf}
\end{figure}
\end{frame}

\end{document}

