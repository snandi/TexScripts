%\documentclass[10pt,dvipsnames,table, handout]{beamer} % To printout the slides without the animations
\documentclass[10pt,dvipsnames,table]{beamer} 
%\usetheme{Luebeck} 
%\usetheme{Madrid} 
%\usetheme{Marburg} 
\usetheme{Warsaw} 
%\setbeamercolor{structure}{fg=cyan!90!white}
%\setbeamercolor{normal text}{fg=white, bg=black}

%%%%%%%%%%%%%%%%%%%%%%%%%%%%%%%%%%%%%%%%%%%%%%%%%%%%%%%%%%%%%%%%%%%%%%
%% Input header file 
%%%%%%%%%%%%%%%%%%%%%%%%%%%%%%%%%%%%%%%%%%%%%%%%%%%%%%%%%%%%%%%%%%%%%%
\input{HeaderfileTexSlides}

%%%%%%%%%%%%%%%%%%%%%%%%%%%%%%%%%%%%%%%%%%%%%%%%%%%%%%%%%%%%%%%%%%%%%%
%% TITLE PAGE 
%%%%%%%%%%%%%%%%%%%%%%%%%%%%%%%%%%%%%%%%%%%%%%%%%%%%%%%%%%%%%%%%%%%%%%
\DeclarePairedDelimiter\ceil{\lceil}{\rceil}
\title[Uniqueness and Reproducibility of Nmaps]{Iterated Curve Registration with Constrained (Penalized) Warping}
\author{Subhrangshu Nandi}
%\institute[Stat 741]{Stat 741, Spring 2015 \\
%  Department of Statistics \\
% University of Wisconsin-Madison}
%\date{April 21, 2015}

\begin{document}
\setlength{\baselineskip}{16truept}
\frame{\maketitle}

%%%%%%%%%%%%%%%%%%%%%%%%%%%%%%%%%%%%%%%%%%%%%%%%%%%%%%%%%%%%%%%%%%%%%%
%% This is for 2015-07-31 meeting, with Prof. Schwartz & Prof. Newton
%% Updating them on statistical evidence that iterated registration 
%% works better than one-time registration, which is much better than
%% no registration
%%%%%%%%%%%%%%%%%%%%%%%%%%%%%%%%%%%%%%%%%%%%%%%%%%%%%%%%%%%%%%%%%%%%%%

%% Outline for this presentation:

%%%%%%%%%%%% Slide 1 %%%%%%%%%%%%
\begin{frame}
\frametitle{Goal}
{\bf{GOAL: Establish uniqueness and reproducibility of intensity profiles of Nmaps aligned to any locations on the genome}}
\end{frame}

%%%%%%%%%%%% Slide 2 %%%%%%%%%%%%
\begin{frame}
\frametitle{Estimation Steps ({\emph{M.florum}} data)}
\begin{enumerate}
\item Select the theoretical number of pixels for any length on the genome. Allow for $\pm 5\%$ stretch. 
\item Make all Nmaps (inside $\pm 5\%$ stretch) of the same length, {\textcolor{blue}{by fitting B-Splines}}
\item Preprocess the Nmaps to select the ones somewhat ``similar'' to each other, {\textcolor{blue}{by pairwise distance measures}}
\item Choose Nmaps that are not completely flat, {\textcolor{blue}{by choosing top 25\% or 30\% of most variable Nmaps, per interval/ fragment}}
\item Test for uniqueness, by comparing Nmaps from two intervals of same length, {\textcolor{blue}{by permutation test}}
\item Align the intensity profiles of Nmaps from the same inteval, {\textcolor{blue}{by iterated penalized registration}}
\item Test for uniqueness again, {\textcolor{blue}{by permutation test}}
\end{enumerate}
\end{frame}

%%%%%%%%%%%% Slide 3 %%%%%%%%%%%%
\begin{frame}
\frametitle{}

\end{frame}

\end{document}

