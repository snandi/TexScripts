%\documentclass[10pt,dvipsnames,table, handout]{beamer} % To printout the slides without the animations
\documentclass[10pt,dvipsnames,table]{beamer} 
%\usetheme{Luebeck} 
\usetheme{Madrid} 
%\usetheme{Marburg} 
\setbeamercolor{structure}{fg=cyan!90!white}
\setbeamercolor{normal text}{fg=white, bg=black}

%%%%%%%%%%%%%%%%%%%%%%%% Packages %%%%%%%%%%%%%%%%%%%%%%%%
\usepackage{amscd}
\usepackage{amsmath}
\usepackage{amssymb}
\usepackage{amsthm}
\usepackage{amsxtra}
\usepackage{bbold}
%\usepackage{bigints}
\usepackage{color}
\usepackage{dsfont}
\usepackage{enumerate}
\usepackage[mathscr]{eucal}
%\usepackage{fancyhdr}
\usepackage{float}
%\usepackage{fullpage} %% Dont use this for beamer presentations
\usepackage{geometry}
\usepackage{graphicx}
\usepackage{hyperref}
\usepackage{indentfirst}
\usepackage{latexsym}
\usepackage{listings}
\usepackage{lscape}
\usepackage{mathtools}
\usepackage{microtype}
\usepackage{natbib}
\usepackage{pdfpages}
\usepackage{verbatim}
\usepackage{wrapfig}
\usepackage{xargs}
\usepackage{xcolor}
\DeclareGraphicsExtensions{.pdf,.png,.jpg, .jpeg}

%%%%%%%%%%%%%%%%%%%%%%%% Commands %%%%%%%%%%%%%%%%%%%%%%%%
\newcommand{\Sup}{\textsuperscript}
\newcommand{\Exp}{\mathds{E}}
\newcommand{\Prob}{\mathds{P}}
\newcommand{\Z}{\mathds{Z}}
\newcommand{\Ind}{\mathds{1}}
\newcommand{\A}{\mathcal{A}}
\newcommand{\F}{\mathcal{F}}
\newcommand{\G}{\mathcal{G}}
\newcommand{\I}{\mathcal{I}}
\newcommand{\be}{\begin{equation}}
\newcommand{\ee}{\end{equation}}
\newcommand{\bes}{\begin{equation*}}
\newcommand{\ees}{\end{equation*}}
\newcommand{\union}{\bigcup}
\newcommand{\intersect}{\bigcap}
\newcommand{\Ybar}{\overline{Y}}
\newcommand{\ybar}{\bar{y}}
\newcommand{\Xbar}{\overline{X}}
\newcommand{\xbar}{\bar{x}}
\newcommand{\betahat}{\hat{\beta}}
\newcommand{\Yhat}{\widehat{Y}}
\newcommand{\yhat}{\hat{y}}
\newcommand{\Xhat}{\widehat{X}}
\newcommand{\xhat}{\hat{x}}
\newcommand{\E}[1]{\operatorname{E}\left[ #1 \right]}
%\newcommand{\Var}[1]{\operatorname{Var}\left( #1 \right)}
\newcommand{\Var}{\operatorname{Var}}
\newcommand{\Cov}[2]{\operatorname{Cov}\left( #1,#2 \right)}
\newcommand{\N}[2][1=\mu, 2=\sigma^2]{\operatorname{N}\left( #1,#2 \right)}
\newcommand{\bp}[1]{\left( #1 \right)}
\newcommand{\bsb}[1]{\left[ #1 \right]}
\newcommand{\bcb}[1]{\left\{ #1 \right\}}
\newcommand*{\permcomb}[4][0mu]{{{}^{#3}\mkern#1#2_{#4}}}
\newcommand*{\perm}[1][-3mu]{\permcomb[#1]{P}}
\newcommand*{\comb}[1][-1mu]{\permcomb[#1]{C}}

%%%%%%%%%%%%% For explanatory bubbles, use the following code %%%%%%%%%%%%%
%% \usepackage{tikz} %% For explanatory bubbles
%% \usepackage{xparse}
%% \usetikzlibrary{shapes.callouts,ocgx}

%% \newcommand{\tikzmark}[1]{\tikz[overlay,remember picture,baseline=0.5ex] \node (#1) {};}

%% % \explainword: #1= identifier to mark the word, #2 text
%% \NewDocumentCommand{\explainword}{r[] m}{
%%     \switchocg{#1}{#2}\tikzmark{#1}
%% }

%% \tikzset{my callout style/.style={
%%         draw,rectangle callout,anchor=pointer,callout relative pointer={(230:1cm)},
%%         rounded corners,align=center,text width=2cm,fill=cyan!20, 
%%     }
%% }

%% % \mycallout: #1 opacity style, #2 pointer base position, #3= text
%% \NewDocumentCommand{\mycallout}{O{opacity=0.8,text opacity=1} m m}{%
%% \begin{tikzpicture}[remember picture, overlay]
%%  \begin{scope}[ocg={ref=#2,status=invisible,name={#3}}]
%% \node[my callout style,#1]at (#2) {#3};
%% \end{scope}
%% \end{tikzpicture}
%% }
%%%%%%%%%%%%%%%%%%%%%%%%%%%%%%%%%%%%%%%%%%%%%%%%%%%%%%%%%%%%%%%%%

%%%%%%%%%%%%%%%%%%%%%%%% TITLE PAGE %%%%%%%%%%%%%%%%%%%%%%%%
\DeclarePairedDelimiter\ceil{\lceil}{\rceil}
\title[Status Update Feb, 2015]{Status Update Meeting}
\author{S. Nandi}
\institute[LMCG]{LMCG \\
 University of Wisconsin-Madison}
\date{February 6, 2014}

\begin{document}
\setlength{\baselineskip}{16truept}
\frame{\maketitle}

%%%%%%%%%%%% Slide 1 %%%%%%%%%%%%
\begin{frame}
\frametitle{Outline}
\begin{itemize}
\item Discussion with Adi
\item LINE segments of Chr 13 vs Chr 3
\item Multidimensional scaling (for clustering)
\end{itemize}
\end{frame}

%%%%%%%%%%%% Slide 2 %%%%%%%%%%%%
\begin{frame}
\frametitle{Update on data download/extraction}
\begin{itemize}
\item Resolved download speed using shell scripts and Steve's help
\item Initial estimate of 2 months - eventually took 3 days
\item Data ready for the whole genome (with at least 10 Nmaps aligned) \\
  (with 1, 3, or 5 pixels around the backbone)
\end{itemize}
\end{frame}

%%%%%%%%%%%% Slide 3 %%%%%%%%%%%%
\begin{frame}
\frametitle{Discussions with Adi}
\begin{itemize}
\pause \item Went through most structural variations in Pre \& Post samples
\pause \item Chr 13 copy number 1, Chr 3 copy number 2
\pause \item Look at intact LINE sequences in Chr 13 and Chr 3
\pause \item Identify locations of insertions/deletions in LINEs of Chr 3
\end{itemize}
\end{frame}

%%%%%%%%%% Chr 13 LINE segments %%%%%%%%%%
%\begin{columns}[t]
%\column{.5\textwidth}
%\centering
%\includegraphics[width=5cm,height=3.5cm]{lab1}\\
%\includegraphics[width=5cm,height=4cm]{lab2}
%\column{.5\textwidth}
%\centering
%\includegraphics[width=5cm,height=4cm]{lab3}\\
%\includegraphics[width=5cm,height=4cm]{lab4}
%\end{columns}
\begin{frame}
\frametitle{L1 LINE segments of Chr 13, at least 6 KB long}
\begin{figure}
\centering
\includegraphics[page=137, scale=0.45]{L1_Chr13_Sequences.pdf}
\end{figure}
\end{frame}

\begin{frame}
\frametitle{L1 LINE segments of Chr 13, at least 6 KB long}
\begin{figure}
\centering
\includegraphics[page=1, scale=0.2]{L1_Chr13_Sequences.pdf}
\includegraphics[page=7, scale=0.2]{L1_Chr13_Sequences.pdf} 
\includegraphics[page=25, scale=0.2]{L1_Chr13_Sequences.pdf} 
\end{figure}
\begin{figure}
\includegraphics[page=8, scale=0.2]{L1_Chr13_Sequences.pdf} 
\includegraphics[page=20, scale=0.2]{L1_Chr13_Sequences.pdf}
\includegraphics[page=28, scale=0.2]{L1_Chr13_Sequences.pdf} 
\end{figure}
\end{frame}
%%%%%%%%%%%%%%%%%%%%%%%%%%%%%%%%%%%%%%%%%%

%%%%%%%%%%% Chr 3, Frag 13144 %%%%%%%%%%%%
\begin{frame}
\frametitle{Chr 3, fragment 13144 }
\begin{center}
\includegraphics[page=1, scale=0.35]{MM52_Ch3_Registered_Frag_13144_Class1.pdf} 
\includegraphics[page=1, scale=0.35]{MM52_Ch3_Registered_Frag_13144_Class2.pdf} 
\end{center}
\end{frame}

\begin{frame}
\frametitle{Chr 3, fragment 13144 }
\begin{center}
\includegraphics[page=1, scale=0.4]{MM52_Ch3_MDS_Frag_13144.pdf} 
\end{center}
\end{frame}
%%%%%%%%%%%%%%%%%%%%%%%%%%%%%%%%%%%%%%%%%%

%%%%%%%%%%% Chr 3, Frag 13144 %%%%%%%%%%%%
\begin{frame}
\frametitle{Chr 3, fragment 13141 }
\begin{center}
\includegraphics[page=1, scale=0.35]{MM52_Ch3_Registered_Class1.pdf} 
\includegraphics[page=1, scale=0.35]{MM52_Ch3_Registered_Class2.pdf} 
\end{center}
\end{frame}

\begin{frame}
\frametitle{Chr 3, fragment 13141 }
\begin{center}
\includegraphics[page=1, scale=0.4]{MM52_Ch3_MDS_Frag_13141.pdf} 
\end{center}
\end{frame}
%%%%%%%%%%%%%%%%%%%%%%%%%%%%%%%%%%%%%%%%%%


%%%%%%%%%%%%%%% Next Steps %%%%%%%%%%%%%%%
\begin{frame}
\frametitle{Next Steps}
\begin{itemize}
\pause \item Multi dimensional scaling, to identify clusters in graphs
\pause \item Use simulated data to study and improve ``Iterated registration''
\pause \item Simultaneous registration and clustering into two populations
\end{itemize}
\end{frame}
%%%%%%%%%%%%%%%%%%%%%%%%%%%%%%%%%%%%%%%%%%
\end{document}
