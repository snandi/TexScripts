%\documentclass[10pt,dvipsnames,table, handout]{beamer} % To printout the slides without the animations
\documentclass[10pt,dvipsnames,table]{beamer} 
%\usetheme{Luebeck} 
\usetheme{Madrid} 
%\usetheme{Marburg} 
\setbeamercolor{structure}{fg=cyan!90!white}
\setbeamercolor{normal text}{fg=white, bg=black}

%%%%%%%%%%%%%%%%%%%%%%%% Packages %%%%%%%%%%%%%%%%%%%%%%%%
\usepackage{amscd}
\usepackage{amsmath}
\usepackage{amssymb}
\usepackage{amsthm}
\usepackage{amsxtra}
\usepackage{bbold}
%\usepackage{bigints}
\usepackage{color}
\usepackage{dsfont}
\usepackage{enumerate}
\usepackage[mathscr]{eucal}
%\usepackage{fancyhdr}
\usepackage{float}
%\usepackage{fullpage} %% Dont use this for beamer presentations
\usepackage{geometry}
\usepackage{graphicx}
\usepackage{hyperref}
\usepackage{indentfirst}
\usepackage{latexsym}
\usepackage{listings}
\usepackage{lscape}
\usepackage{mathtools}
\usepackage{microtype}
\usepackage{natbib}
\usepackage{pdfpages}
\usepackage{verbatim}
\usepackage{wrapfig}
\usepackage{xargs}
\usepackage{xcolor}
\DeclareGraphicsExtensions{.pdf,.png,.jpg, .jpeg}

%%%%%%%%%%%%%%%%%%%%%%%% Commands %%%%%%%%%%%%%%%%%%%%%%%%
\newcommand{\Sup}{\textsuperscript}
\newcommand{\Exp}{\mathds{E}}
\newcommand{\Prob}{\mathds{P}}
\newcommand{\Z}{\mathds{Z}}
\newcommand{\Ind}{\mathds{1}}
\newcommand{\A}{\mathcal{A}}
\newcommand{\F}{\mathcal{F}}
\newcommand{\G}{\mathcal{G}}
\newcommand{\I}{\mathcal{I}}
\newcommand{\be}{\begin{equation}}
\newcommand{\ee}{\end{equation}}
\newcommand{\bes}{\begin{equation*}}
\newcommand{\ees}{\end{equation*}}
\newcommand{\union}{\bigcup}
\newcommand{\intersect}{\bigcap}
\newcommand{\Ybar}{\overline{Y}}
\newcommand{\ybar}{\bar{y}}
\newcommand{\Xbar}{\overline{X}}
\newcommand{\xbar}{\bar{x}}
\newcommand{\betahat}{\hat{\beta}}
\newcommand{\Yhat}{\widehat{Y}}
\newcommand{\yhat}{\hat{y}}
\newcommand{\Xhat}{\widehat{X}}
\newcommand{\xhat}{\hat{x}}
\newcommand{\E}[1]{\operatorname{E}\left[ #1 \right]}
%\newcommand{\Var}[1]{\operatorname{Var}\left( #1 \right)}
\newcommand{\Var}{\operatorname{Var}}
\newcommand{\Cov}[2]{\operatorname{Cov}\left( #1,#2 \right)}
\newcommand{\N}[2][1=\mu, 2=\sigma^2]{\operatorname{N}\left( #1,#2 \right)}
\newcommand{\bp}[1]{\left( #1 \right)}
\newcommand{\bsb}[1]{\left[ #1 \right]}
\newcommand{\bcb}[1]{\left\{ #1 \right\}}
\newcommand*{\permcomb}[4][0mu]{{{}^{#3}\mkern#1#2_{#4}}}
\newcommand*{\perm}[1][-3mu]{\permcomb[#1]{P}}
\newcommand*{\comb}[1][-1mu]{\permcomb[#1]{C}}

%%%%%%%%%%%%% For explanatory bubbles, use the following code %%%%%%%%%%%%%
%% \usepackage{tikz} %% For explanatory bubbles
%% \usepackage{xparse}
%% \usetikzlibrary{shapes.callouts,ocgx}

%% \newcommand{\tikzmark}[1]{\tikz[overlay,remember picture,baseline=0.5ex] \node (#1) {};}

%% % \explainword: #1= identifier to mark the word, #2 text
%% \NewDocumentCommand{\explainword}{r[] m}{
%%     \switchocg{#1}{#2}\tikzmark{#1}
%% }

%% \tikzset{my callout style/.style={
%%         draw,rectangle callout,anchor=pointer,callout relative pointer={(230:1cm)},
%%         rounded corners,align=center,text width=2cm,fill=cyan!20, 
%%     }
%% }

%% % \mycallout: #1 opacity style, #2 pointer base position, #3= text
%% \NewDocumentCommand{\mycallout}{O{opacity=0.8,text opacity=1} m m}{%
%% \begin{tikzpicture}[remember picture, overlay]
%%  \begin{scope}[ocg={ref=#2,status=invisible,name={#3}}]
%% \node[my callout style,#1]at (#2) {#3};
%% \end{scope}
%% \end{tikzpicture}
%% }
%%%%%%%%%%%%%%%%%%%%%%%%%%%%%%%%%%%%%%%%%%%%%%%%%%%%%%%%%%%%%%%%%

%%%%%%%%%%%%%%%%%%%%%%%% TITLE PAGE %%%%%%%%%%%%%%%%%%%%%%%%
\DeclarePairedDelimiter\ceil{\lceil}{\rceil}
\title[Status Update Mar '15]{Status Update Meeting}
\author{S. Nandi}
\institute[LMCG]{LMCG \\
 University of Wisconsin-Madison}
\date{March 5, 2014}

\begin{document}
\setlength{\baselineskip}{16truept}
\frame{\maketitle}

%%%%%%%%%%%% Slide 1 %%%%%%%%%%%%
\begin{frame}
\frametitle{Outline}
\begin{itemize}
\item Simulation example
\item Iterated Registration
\item Amplitude and phase variability
\item Distance between two curves
\end{itemize}
\end{frame}

%%%%%%%%%%%% Slide 2 %%%%%%%%%%%%
\begin{frame}
\frametitle{True Curves}
\begin{itemize}
\item Generated curves over the interval $[-3,3]$ of the form
\[ x_i(t) = z_{i_1} e^{\frac{-(t - 1.5)^2}{2}} + z_{i_2} e^{\frac{-(t + 1.5)^2}{2}}, \text{\ \ where \ } z_{i_1}, z_{i_2} \sim \mathcal{N}(1, 0.25^2) \]
\end{itemize}
\begin{center}
\includegraphics[page=1, scale=0.35]{Simulation/Plot1_OriginalCurves.pdf} 
\end{center}
\end{frame}

%%%%%%%%%%%% Slide 3 %%%%%%%%%%%%
\begin{frame}
\frametitle{Warped Curves}
The associated warping functions $h_i$:
\[ h_i(t) = 6 \frac{e^{[a_i(t+3)/6]} - 1}{e^{a_i} - 1} - 3 ,\ \ a_i \in [-1,1]\]
\begin{center}
\includegraphics[page=1, scale=0.35]{Simulation/Plot2_WarpedCurves.pdf} 
\end{center}
\end{frame}

%%%%%%%%%%%% Slide 3 %%%%%%%%%%%%
\begin{frame}
\begin{center}
\includegraphics[page=1, scale=0.35]{Simulation/Plot1_OriginalCurves_6.pdf} 
\includegraphics[page=1, scale=0.35]{Simulation/Plot2_WarpedCurves_6.pdf} 
\end{center}
\end{frame}

%%%%%%%%%%%% Slide 4 %%%%%%%%%%%%
\begin{frame}
\frametitle{Registered: 6 curves, seed 20}
\begin{center}
\includegraphics[page=1, scale=0.48]{Simulation/Plot3_Registered_6_Seed20.pdf} 
\end{center}
\end{frame}

%%%%%%%%%%%% Slide 5 %%%%%%%%%%%%
\begin{frame}
\frametitle{Registered: 6 curves, seed 20}
% latex table generated in R 3.1.2 by xtable 1.7-4 package
% Fri Mar  6 02:32:40 2015

\begin{table}[ht]
\footnotesize
\centering
\begin{tabular}{rrrr}
  \hline
 & MS.Amp & MS.Phase & RSq \\ 
  \hline
1 & 0.2150 & 0.1707 & 0.4426 \\ 
  2 & 0.2038 & 0.1761 & 0.4636 \\ 
  3 & 0.1965 & 0.1408 & 0.4174 \\ 
  4 & 0.1896 & 0.0997 & 0.3446 \\ 
  5 & 0.1859 & 0.0604 & 0.2451 \\ 
   \hline
\end{tabular}
\end{table}

\begin{center}
\includegraphics[page=1, scale=0.3]{Simulation/Plot4_MeanReg_6_Seed20.pdf} 
\end{center}
\end{frame}

%%%%%%%%%%%% Slide 6 %%%%%%%%%%%%
\begin{frame}
\frametitle{Quantification of the {\emph{effectiveness}} of registration}
\begin{itemize}
\item Kneip and Ramsay (2008) - Quantifying the amounts of ``Phase'' and ``Amplitude'' variations
\item Define: $\text{MSE}_{total} = \text{MSE}_{amp} + \text{MSE}_{phase}$
\item Idea: ``Registration'' (or alignment) will result in higher and sharper peaks and valleys
\item $x_i$ be original curves, and $y_i$ be registered versions. $\text{MSE}_{total} = \frac{1}{N}\sum\limits_{i=1}^{N}\int[x_i(t) - \bar{x}(t)]^2dt$ \ \ {\footnotesize{($t:$ pixel position; $i: i^{th}$ Nmap.)}}; 
$\text{MSE}_{amp} = C_R\frac{1}{N}\sum\limits_{i=1}^{N}\int[y_i(t) - \bar{y}(t)]^2dt$ \\
$\text{MSE}_{phase} = C_R\int \bar{y}(t)^2 dt - \int \bar{x}(t)^2 dt$, \\ where, $C_R = 1 + \propto $ (cov of warping fn and registered fn)

\item AmpPhase $R^2 = \frac{\text{MSE}_{phase}}{\text{MSE}_{total}}$
\end{itemize}
\end{frame}

%%%%%%%%%%%% Slide 6 %%%%%%%%%%%%
\begin{frame}
\begin{center}
\includegraphics[page=1, scale=0.2]{Simulation/Plot1_OriginalCurves_6_Seed17.pdf} 
\includegraphics[page=1, scale=0.2]{Simulation/Plot2_WarpedCurves_6_Seed17.pdf} 
\newline
\includegraphics[page=1, scale=0.33]{Simulation/Plot3_Registered_6_Seed17.pdf} 
\end{center}
\end{frame}

\begin{frame}
\frametitle{Registered: 6 curves, seed 17}
% latex table generated in R 3.1.2 by xtable 1.7-4 package
% Fri Mar  6 03:26:04 2015
\begin{table}[ht]
\footnotesize
\centering
\begin{tabular}{rrrr}
  \hline
 & MS.Amp & MS.Phase & RSq \\ 
  \hline
1 & 0.1431 & 0.1215 & 0.4592 \\ 
  2 & 0.1387 & 0.1152 & 0.4538 \\ 
  3 & 0.1387 & 0.1147 & 0.4527 \\ 
  4 & 0.1388 & 0.1154 & 0.4540 \\ 
  5 & 0.1388 & 0.1161 & 0.4555 \\ 
   \hline
\end{tabular}
\end{table}

\begin{center}
\includegraphics[page=1, scale=0.3]{Simulation/Plot4_MeanReg_6_Seed17.pdf} 
\end{center}
\end{frame}

%%%%%%%%%%%% Slide 7 %%%%%%%%%%%%
\begin{frame}
\begin{center}
\includegraphics[page=1, scale=0.2]{Simulation/Plot1_OriginalCurves_6_Seed12.pdf} 
\includegraphics[page=1, scale=0.2]{Simulation/Plot2_WarpedCurves_6_Seed12.pdf} 
\newline
\includegraphics[page=1, scale=0.33]{Simulation/Plot3_Registered_6_Seed12.pdf} 
\end{center}
\end{frame}

\begin{frame}
\frametitle{Registered: 6 curves, seed 12}
% latex table generated in R 3.1.2 by xtable 1.7-4 package
% Fri Mar  6 03:35:16 2015
\begin{table}[ht]
\footnotesize
\centering
\begin{tabular}{rrrr}
  \hline
 & MS.Amp & MS.Phase & RSq \\ 
  \hline
1 & 0.1676 & -0.0424 & -0.3384 \\ 
  2 & 0.1458 & -0.0945 & -1.8427 \\ 
  3 & 0.1434 & -0.1170 & -4.4372 \\ 
  4 & 0.1426 & -0.1264 & -7.8282 \\ 
  5 & 0.1418 & -0.1334 & -15.9766 \\ 
   \hline
\end{tabular}
\end{table}

\begin{center}
\includegraphics[page=1, scale=0.3]{Simulation/Plot4_MeanReg_6_Seed12.pdf} 
\end{center}
\end{frame}

%%%%%%%%%%%% Slide 8 %%%%%%%%%%%%
\begin{frame}
\begin{center}
\includegraphics[page=1, scale=0.2]{Simulation/Plot1_OriginalCurves_20_Seed12.pdf} 
\includegraphics[page=1, scale=0.2]{Simulation/Plot2_WarpedCurves_20_Seed12.pdf} 
\newline
\includegraphics[page=1, scale=0.33]{Simulation/Plot3_Registered_20_Seed12.pdf} 
\end{center}
\end{frame}

\begin{frame}
\frametitle{Registered: 6 curves, seed 12}
% latex table generated in R 3.1.2 by xtable 1.7-4 package
% Fri Mar  6 03:39:04 2015
\begin{table}[ht]
\footnotesize
\centering
\begin{tabular}{rrrr}
  \hline
 & MS.Amp & MS.Phase & RSq \\ 
  \hline
1 & 0.1776 & 0.0221 & 0.1107 \\ 
  2 & 0.1699 & -0.0038 & -0.0231 \\ 
  3 & 0.1696 & -0.0092 & -0.0572 \\ 
  4 & 0.1695 & -0.0101 & -0.0633 \\ 
  5 & 0.1694 & -0.0103 & -0.0645 \\ 
   \hline
\end{tabular}
\end{table}

\begin{center}
\includegraphics[page=1, scale=0.3]{Simulation/Plot4_MeanReg_6_Seed12.pdf} 
\end{center}
\end{frame}

%%%%%%%%%%%% Slide 9 %%%%%%%%%%%%
\begin{frame}
\frametitle{Lessons from Simulation}
\begin{itemize}
\pause \item Registration works
\pause \item Iterated registration makes the consensus signal more precise \\
\textcolor{orange}{\bf{But NOT necessarily accurate}}
\pause \item Kneip-Ramsay's amplitude-phase decomposition quantification has obvious limitations. \\
\textcolor{orange}{\bf{NEED new metric to establish convergence criteria}}
\pause \item Propose new metric to estimate distance between two curves
\end{itemize}
\end{frame}

%%%%%%%%%%%% Slide 10 %%%%%%%%%%%%
\begin{frame}
\frametitle{Distance between two curves}
Fr\'echet distance is a measure of similarity between curves that takes into account the location and ordering of the points along the curves.\\
Let $A$ and $B$ be two given curves in $S$ (metric space). Then, the Fr\'echet distance between $A$ and $B$ is defined as the infimum over all reparameterizations $\alpha$ and $\beta$ of $[0,1]$ of the maximum over all $t \in [0,1]$ of the distance in $S$ between $A(\alpha(t))$ and $B(\beta(t))$. 
\[ F(A,B) = \inf\limits_{\alpha, \beta} \max\limits_{t \in [0,1]} \left\{ d(A(\alpha(t)), B(\beta(t))) \right\}   \]

\begin{center}
\includegraphics[scale=0.18]{Frechet_Image01.jpg} 
\includegraphics[scale=0.23]{Frechet_Image03.jpg} 
\end{center}
\end{frame}

%%%%%%%%%%%% Slide 11 %%%%%%%%%%%%
\begin{frame}
\frametitle{Curvature weighted Fr\'echet Distance}
{\underline{Proposing: Curvature weighted Fr\'echet Distance between two curves}}
\begin{itemize} 
\item Sample more points in the curve with higher curvature
\item Contribution from the whole curve (sum)
\item Compare curves of different lengths
\end{itemize} 

\begin{center}
\includegraphics[scale=0.18]{Frechet_Image02.png} 
\end{center}
\end{frame}

%%%%%%%%%%%% Slide 12 %%%%%%%%%%%%
\begin{frame}
\frametitle{Questions/Comments/Suggestions}
\begin{itemize} 
\item Consider constrained optimization for registration, to limit the phase shifts that are allowed to register
\item Local warping of signals due to non-uniform coupling of dye molecules with the DNA molecules, not because of local stretch difference
\item Discard the fragments from the ends of the NMaps, as their stretches could be different from that of the inner fragments
\item Continue building on Fr\'echet distance to quantify distance between curves, and use it to establish convergence (of iterated registration process)
\item Explore the Bayesian approach for joint clustering and registration
\end{itemize}
\end{frame}

\end{document}
