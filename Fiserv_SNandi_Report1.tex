\documentclass[11pt]{extarticle} %extarticle for fontsizes other than 10, 11 And 12
%\documentclass[11p]{article}

%%%%%%%%%%%%%%%%%%%%%%%%%%%%%%%%%%%%%%%%%%%%%%%%%%%%%%%%%%%%%%%%%%%%%%
%% Input header file 
%%%%%%%%%%%%%%%%%%%%%%%%%%%%%%%%%%%%%%%%%%%%%%%%%%%%%%%%%%%%%%%%%%%%%%
%%%%%%%%%%%%%%%%%%%%%%%% Packages %%%%%%%%%%%%%%%%%%%%%%%%
\usepackage{amscd}
\usepackage{amsmath}
\usepackage{amssymb}
\usepackage{amsthm}
\usepackage{amsxtra}
\usepackage{animate}
\usepackage{bbold}
%\usepackage{bigints}
\usepackage{color, colortbl}
\usepackage{dsfont}
\usepackage{enumerate}
\usepackage[mathscr]{eucal}
%\usepackage{fancyhdr}
\usepackage{float}
%\usepackage{fullpage} %% Dont use this for beamer presentations
\usepackage{geometry}
\usepackage{graphicx}
\usepackage{hyperref}
\usepackage{indentfirst}
\usepackage{latexsym}
\usepackage{listings}
\usepackage{lscape}
\usepackage{mathtools}
\usepackage{microtype}
\usepackage{multirow}
\usepackage{natbib}
\usepackage{pdfpages}
\usepackage{verbatim}
\usepackage{wrapfig}
\usepackage{xargs}
\usepackage{xcolor}
\DeclareGraphicsExtensions{.pdf,.png,.jpg, .jpeg}
\definecolor{LightCyan}{rgb}{0.88,1,1}

%%%%%%%%%%%%%%%%%%%%%%%% Commands %%%%%%%%%%%%%%%%%%%%%%%%
\newcommand{\Sup}{\textsuperscript}
\newcommand{\Exp}{\mathds{E}}
\newcommand{\Prob}{\mathds{P}}
\newcommand{\Z}{\mathds{Z}}
\newcommand{\Ind}{\mathds{1}}
\newcommand{\A}{\mathcal{A}}
\newcommand{\F}{\mathcal{F}}
\newcommand{\G}{\mathcal{G}}
\newcommand{\I}{\mathcal{I}}
\newcommand{\R}{\mathcal{R}}
\newcommand{\Real}{\mathbb{R}}
\newcommand{\be}{\begin{equation}}
\newcommand{\ee}{\end{equation}}
\newcommand{\bes}{\begin{equation*}}
\newcommand{\ees}{\end{equation*}}
\newcommand{\union}{\bigcup}
\newcommand{\intersect}{\bigcap}
\newcommand{\Ybar}{\overline{Y}}
\newcommand{\ybar}{\bar{y}}
\newcommand{\Xbar}{\overline{X}}
\newcommand{\xbar}{\bar{x}}
\newcommand{\betahat}{\hat{\beta}}
\newcommand{\Yhat}{\widehat{Y}}
\newcommand{\yhat}{\hat{y}}
\newcommand{\Xhat}{\widehat{X}}
\newcommand{\xhat}{\hat{x}}
\newcommand{\E}[1]{\operatorname{E}\left[ #1 \right]}
%\newcommand{\Var}[1]{\operatorname{Var}\left( #1 \right)}
\newcommand{\Var}{\operatorname{Var}}
\newcommand{\Cov}[2]{\operatorname{Cov}\left( #1,#2 \right)}
\newcommand{\N}[2][1=\mu, 2=\sigma^2]{\operatorname{N}\left( #1,#2 \right)}
\newcommand{\bp}[1]{\left( #1 \right)}
\newcommand{\bsb}[1]{\left[ #1 \right]}
\newcommand{\bcb}[1]{\left\{ #1 \right\}}
\newcommand*{\permcomb}[4][0mu]{{{}^{#3}\mkern#1#2_{#4}}}
\newcommand*{\perm}[1][-3mu]{\permcomb[#1]{P}}
\newcommand*{\comb}[1][-1mu]{\permcomb[#1]{C}}


\geometry{left=0.8in, right=0.8in, top=1in, bottom=0.8in}
\begin{document}
%\SweaveOpts{concordance=TRUE}
\bibliographystyle{plain}  %Choose a bibliograhpic style

\title{Report 1: Evaluate effectiveness of LDP}
\author{Subhrangshu Nandi\\
  Statistics PhD Student, \\
%  Research Assistant,
%  Laboratory of Molecular and Computational Genomics, \\
  University of Wisconsin - Madison \\
  nands31@gmail.com}
%\date{May 31, 2015}
\date{}

\maketitle

\newpage
\section*{Introduction}
Answers to the following fitness test questions were evaluated, to identify any pattern between leaders who completed {\bf{LDP}} and those who didn't. 
\begin{table}[H]
\centering
\begin{tabular}{l|c|p{5in}}
\hline
Goals 		& Q1 & I have clear agreements with my manager about my identified goals. \\
      		& Q2 & My manager holds me accountable to have SMART goals (specific, motivating, attainable, relevant and trackable). \\
      		& Q3 & I understand how my goals are related to/ directly aligned to the overall organization’s goals. \\
		& Q4 & My mutually agreed upon goals are motivating and appropriately challenging. \\
\hline
Diagnosing	& Q1 & My manager is able to fully assess my competence (knowledge and skill) for each one of my identified goals. \\
		& Q2 & My manager’s leadership style depends on my needs on a specific goal/task. \\
		& Q3 & My manager stays connected with what I am doing and gives me appropriate feedback. \\
		& Q4 & My manager asks how he/ she can support me as a leader.\\
\hline
Matching	& Q1 & My manager is able to assess my commitment (motivation and confidence) on each of my goals. \\
		& Q2 & My manager knows when to allow me to figure situations out and acts as a sounding board, not a provider of directives. \\
		& Q3 & My manager is comfortable being directive. \\
		& Q4 & My manager is comfortable being supportive. \\
\hline
Coaching	& Q1 & My manager is able to effectively coach me through potential challenges as I work towards my goals. \\
		& Q2 & I meet regularly one on one with my manager so I can ask for the direction and support I need. \\
		& Q3 & My manager invites feedback about how he or she could be more effective as a leader. \\
\hline
\end{tabular}
\end{table}

\section*{Model Fitting}
As can be noticed in the boxplots in the end of the report, none of the answers to the 15 questions above illustrate any difference between the leaders who completed {\bf{LDP}} and those who didn't. This observation was consistent when a logistic regression model was fit to the data. Table 1 is a summary of the model fit. As can be seen in the table other than Goals-Q1, none of the answers have any significant association with LDP completion. The p-values are all greater than 0.05, hence, the associations are not statistically significant. 

Individual t-test results between the two groups yield similar results. For example, testing the hypothesis
$$H_0: \mu_{G1_A} = \mu_{G1_B} \hspace{1cm} vs \hspace{1cm} H_a: \mu_{G1_A} < \mu_{G1_B}$$ where $\mu_{G1_A}:$ Mean of answers of Goals, Q1 for leaders who did not complete LDP and $\mu_{G1_B}:$ Mean of answers of Goals, Q1 for leaders who completed LDP, two sample t-test yields 
$t = 0.207, \ df = 65.358, \ \text{p-value} = 0.5817$

% latex table generated in R 3.1.1 by xtable 1.7-4 package
% Wed Nov 11 16:50:21 2015
\begin{table}[H]
\centering
\begin{tabular}{rrrrr}
  \hline
  covariate	& estimate & std. error & z value & p-value \\ 
  \hline
  (Intercept) 	& 2.2582  & 4.2820 & 0.53  & 0.5979 \\ 
  Goals.Q1 	& 2.3669  & 1.1568 & 2.05  & 0.0408 \\ 
  Goals.Q2 	& -0.9224 & 1.2212 & -0.76 & 0.4501 \\ 
  Goals.Q3 	& 0.4169  & 1.2487 & 0.33  & 0.7385 \\ 
  Goals.Q4 	& -1.7714 & 0.9815 & -1.80 & 0.0711 \\ 
  Diagnosing.Q1 & -0.6215 & 0.8610 & -0.72 & 0.4704 \\ 
  Diagnosing.Q2 & -0.2532 & 0.7373 & -0.34 & 0.7313 \\ 
  Diagnosing.Q3 & -1.0259 & 0.8871 & -1.16 & 0.2475 \\ 
  Diagnosing.Q4 & -0.4572 & 0.8111 & -0.56 & 0.5730 \\ 
  Matching.Q1 	& 0.4789  & 1.0533 & 0.45  & 0.6494 \\ 
  Matching.Q2 	& -0.3249 & 0.9813 & -0.33 & 0.7406 \\ 
  Matching.Q3 	& -0.4516 & 0.7969 & -0.57 & 0.5710 \\ 
  Matching.Q4 	& 1.2323  & 1.0729 & 1.15  & 0.2507 \\ 
  Coaching.Q1 	& -0.1621 & 0.8654 & -0.19 & 0.8514 \\ 
  Coaching.Q2 	& 0.5628  & 0.6397 & 0.88  & 0.3789 \\ 
  Coaching.Q3 	& 0.4592  & 0.5343 & 0.86  & 0.3901 \\ 
  ResponsePct 	& 0.3030  & 0.7493 & 0.40  & 0.6859 \\ 
  \hline
\end{tabular}
\caption{Fitting Fitness test questions with LDP completion as the response}
\end{table}

When individual questions didn't yield significant results, average of each category was tested. For example, new covariate $Goals_{Avg}$ was created, which was the average of all the answers related to $Goals$. Similarly, $Diagnosing_{Avg}$, $Matching_{Avg}$ and $Coaching_{Avg}$. Below is the result of the model fit
% latex table generated in R 3.1.1 by xtable 1.7-4 package
% Wed Nov 11 21:44:34 2015
\begin{table}[H]
\centering
\begin{tabular}{rrrrr}
  \hline
 & estimate & std. error & z value & p-value \\ 
  \hline
(Intercept) & 3.7351 & 3.4058 & 1.10 & 0.2728 \\ 
  Goals.Avg & -0.5551 & 0.9138 & -0.61 & 0.5435 \\ 
  Diagnosing.Avg & -1.8530 & 1.1394 & -1.63 & 0.1039 \\ 
  Matching.Avg & 0.7192 & 1.2965 & 0.55 & 0.5791 \\ 
  Coaching.Avg & 0.9661 & 0.7996 & 1.21 & 0.2269 \\ 
   \hline
\end{tabular}
\caption{Fitting Average of Fitness test questions with LDP completion as the response}
\end{table}

Even in this model none of the question categories seem to yield statistically significant associations with LDP completions. 

Performance Rating of 2014 was then tested with LDP completion to detect any effect. Only three types of performance rating had big sample size to compare. 
% latex table generated in R 3.1.1 by xtable 1.7-4 package
% Wed Nov 11 21:58:40 2015
\begin{table}[H]
\centering
\begin{tabular}{lcccc}
 \hline
 & Exceeds Many & Meets Expectations & Meets Most & Total\\ 
 \hline
 LDP Yes &  18 &  32 &   6 & 56\\ 
 LDP No &  14 &  50 &  13 & 77\\ 
 \hline
 Total & 32 & 82 & 19 & 133 \\
 \hline
\end{tabular}
\end{table}
Pearson's Chi-squared test yields $\chi^2 = 3.8093, df = 2, \text{p-value} = 0.1489$. There is no statistically significant association between LDP completion and performance ratings. In this test, leaders who completed their LDP in 2015 were allocated to the group of ``LDP No'', since their performance evaluation was before their LDP completion. If however, there was reason to believe that 2015 performance evaluations would be similar to that of 2014, then the analysis would be different. Below is the table illustrating this:
% latex table generated in R 3.1.1 by xtable 1.7-4 package
% Thu Nov 12 12:36:50 2015
\begin{table}[H]
\centering
\begin{tabular}{lcccc}
\hline
& Exceeds Many & Meets Expectations & Meets Most & Total\\ 
\hline
LDP Yes &  21 &  55 &   6  & 82\\ 
LDP No &  11 &  27 &  13 & 51\\ 
\hline
Total & 32 & 82 & 19 & 133 \\
\hline
\end{tabular}
\end{table}
A Chi-square goodness of fit test does reveal statistically significant association between leaders who completed the LDP training and their receiving superior performance evaluations. The $\chi^2 = 8.5012,\ df = 2,\  \text{p-value} = 0.01426$

\newpage

\section*{Summary Plots}
Below are the boxplots {\footnote{The interpretation of boxplots can be found here: https://en.wikipedia.org/wiki/Box\_plot\#/media/File:Boxplot\_vs\_PDF.svg}} for Goals-Q1
\begin{figure}[H]
\centering 
\includegraphics[scale=0.45, page=1]{../BoxPlots.pdf}
\includegraphics[scale=0.45, page=2]{../BoxPlots.pdf} \\
\includegraphics[scale=0.65, page=3]{../BoxPlots.pdf}
\end{figure}

\newpage
Below are the boxplots for Goals-Q2
\begin{figure}[H]
\centering 
\includegraphics[scale=0.45, page=4]{../BoxPlots.pdf} 
\includegraphics[scale=0.45, page=5]{../BoxPlots.pdf} \\
\includegraphics[scale=0.65, page=6]{../BoxPlots.pdf} \\
\end{figure}

\newpage
Below are the boxplots for Goals-Q3
\begin{figure}[H]
\centering 
\includegraphics[scale=0.45, page=7]{../BoxPlots.pdf} 
\includegraphics[scale=0.45, page=8]{../BoxPlots.pdf} \\
\includegraphics[scale=0.65, page=9]{../BoxPlots.pdf} \\
\end{figure}

\newpage
Below are the boxplots for Goals-Q4
\begin{figure}[H]
\centering 
\includegraphics[scale=0.45, page=10]{../BoxPlots.pdf} 
\includegraphics[scale=0.45, page=11]{../BoxPlots.pdf} \\
\includegraphics[scale=0.65, page=12]{../BoxPlots.pdf} \\
\end{figure}

\newpage
Below are the boxplots for Diagnosing-Q1
\begin{figure}[H]
\centering 
\includegraphics[scale=0.45, page=13]{../BoxPlots.pdf} 
\includegraphics[scale=0.45, page=14]{../BoxPlots.pdf} \\
\includegraphics[scale=0.65, page=15]{../BoxPlots.pdf} \\
\end{figure}

\newpage
Below are the boxplots for Diagnosing-Q2
\begin{figure}[H]
\centering 
\includegraphics[scale=0.45, page=16]{../BoxPlots.pdf} 
\includegraphics[scale=0.45, page=17]{../BoxPlots.pdf} \\
\includegraphics[scale=0.65, page=18]{../BoxPlots.pdf} \\
\end{figure}

\newpage
Below are the boxplots for Diagnosing-Q3
\begin{figure}[H]
\centering 
\includegraphics[scale=0.45, page=19]{../BoxPlots.pdf} 
\includegraphics[scale=0.45, page=20]{../BoxPlots.pdf} \\
\includegraphics[scale=0.65, page=21]{../BoxPlots.pdf} \\
\end{figure}

\newpage
Below are the boxplots for Diagnosing-Q4
\begin{figure}[H]
\centering 
\includegraphics[scale=0.45, page=22]{../BoxPlots.pdf} 
\includegraphics[scale=0.45, page=23]{../BoxPlots.pdf} \\
\includegraphics[scale=0.65, page=24]{../BoxPlots.pdf} \\
\end{figure}

\newpage
Below are the boxplots for Matching-Q1
\begin{figure}[H]
\centering 
\includegraphics[scale=0.45, page=25]{../BoxPlots.pdf} 
\includegraphics[scale=0.45, page=26]{../BoxPlots.pdf} \\
\includegraphics[scale=0.65, page=27]{../BoxPlots.pdf} \\
\end{figure}

\newpage
Below are the boxplots for Matching-Q2
\begin{figure}[H]
\centering 
\includegraphics[scale=0.45, page=28]{../BoxPlots.pdf} 
\includegraphics[scale=0.45, page=29]{../BoxPlots.pdf} \\
\includegraphics[scale=0.65, page=30]{../BoxPlots.pdf} \\
\end{figure}

\newpage
Below are the boxplots for Matching-Q3
\begin{figure}[H]
\centering 
\includegraphics[scale=0.45, page=31]{../BoxPlots.pdf} 
\includegraphics[scale=0.45, page=32]{../BoxPlots.pdf} \\
\includegraphics[scale=0.65, page=33]{../BoxPlots.pdf} \\
\end{figure}

\newpage
Below are the boxplots for Matching-Q4
\begin{figure}[H]
\centering 
\includegraphics[scale=0.45, page=34]{../BoxPlots.pdf} 
\includegraphics[scale=0.45, page=35]{../BoxPlots.pdf} \\
\includegraphics[scale=0.65, page=36]{../BoxPlots.pdf} \\
\end{figure}

\newpage
Below are the boxplots for Coaching-Q1
\begin{figure}[H]
\centering 
\includegraphics[scale=0.45, page=37]{../BoxPlots.pdf} 
\includegraphics[scale=0.45, page=38]{../BoxPlots.pdf} \\
\includegraphics[scale=0.65, page=39]{../BoxPlots.pdf} \\
\end{figure}

\newpage
Below are the boxplots for Coaching-Q2
\begin{figure}[H]
\centering 
\includegraphics[scale=0.45, page=40]{../BoxPlots.pdf} 
\includegraphics[scale=0.45, page=41]{../BoxPlots.pdf} \\
\includegraphics[scale=0.65, page=42]{../BoxPlots.pdf} \\
\end{figure}

\newpage
Below are the boxplots for Coaching-Q3
\begin{figure}[H]
\centering 
\includegraphics[scale=0.45, page=43]{../BoxPlots.pdf} 
\includegraphics[scale=0.45, page=44]{../BoxPlots.pdf} \\
\includegraphics[scale=0.65, page=45]{../BoxPlots.pdf} \\
\end{figure}

\end{document}
