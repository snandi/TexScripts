%\documentclass[10pt,dvipsnames,table, handout]{beamer} % To printout the slides without the animations
\documentclass[10pt,dvipsnames,table]{beamer} 
%\usetheme{Luebeck} 
\usetheme{Madrid} 
%\usetheme{Marburg} 
%\usetheme{Warsaw} 
%\setbeamercolor{structure}{fg=cyan!90!white}
\setbeamercolor{normal text}{fg=white, bg=black}

%%%%%%%%%%%%%%%%%%%%%%%%%%%%%%%%%%%%%%%%%%%%%%%%%%%%%%%%%%%%%%%%%%%%%%
%% Input header file 
%%%%%%%%%%%%%%%%%%%%%%%%%%%%%%%%%%%%%%%%%%%%%%%%%%%%%%%%%%%%%%%%%%%%%%
\input{HeaderfileTexSlides}

%%%%%%%%%%%%%%%%%%%%%%%%%%%%%%%%%%%%%%%%%%%%%%%%%%%%%%%%%%%%%%%%%%%%%%
%% TITLE PAGE 
%%%%%%%%%%%%%%%%%%%%%%%%%%%%%%%%%%%%%%%%%%%%%%%%%%%%%%%%%%%%%%%%%%%%%%
\DeclarePairedDelimiter\ceil{\lceil}{\rceil}
\title[Fluoroscanning: beyond nanocoding]{Fluoroscanning: beyond Nanocoding; \\ {\emph{Realistic path to personalized genomics}}}
\author{Subhrangshu Nandi}
%\institute[Stat 741]{Stat 741, Spring 2015 \\
%  Department of Statistics \\
% University of Wisconsin-Madison}
%\date{April 21, 2015}

\begin{document}
\setlength{\baselineskip}{16truept}
\frame{\maketitle}

%%%%%%%%%%%%%%%%%%%%%%%%%%%%%%%%%%%%%%%%%%%%%%%%%%%%%%%%%%%%%%%%%%%%%%
%% This is for 2015-09-28 Schwartz group meeting:
%% 1. Introduction to fluoroscanning
%% 2. Some statistical challenges of fluoroscanning
%% 3. Quality score & improvement of consensus
%%%%%%%%%%%%%%%%%%%%%%%%%%%%%%%%%%%%%%%%%%%%%%%%%%%%%%%%%%%%%%%%%%%%%%

%% Outline for this presentation:

%%%%%%%%%%%% Slide 1 %%%%%%%%%%%%
\begin{frame}
\frametitle{Nanocoding - brief background}
\begin{figure}[T]
\includegraphics[scale=0.28]{molecule37_frame65.png}
\end{figure}
\end{frame}

%%%%%%%%%%%% Slide 2 %%%%%%%%%%%%
\begin{frame}
\frametitle{Nanocoding - brief background}
\begin{figure}[T]
\includegraphics[scale=0.25]{BarcodeImage.jpg}
\end{figure}

{\emph{M.florum}} reference map: \\
81.6 18.7 59.4 13.9 9.0 5.0 12.3 10.2 15.0 25.4 3.9 20.9 15.6 10.2 9.5 11.1 4.5 
13.7 26.3 38.3 2.1 31.0 19.1 3.6 32.2 41.3 9.8 \textcolor<3>{SpringGreen}{16.4 15.7 6.3 36.5 18.3} 32.1 21.0 
3.3 14.3 51.3 16.0 17.9

\pause
Molecule $2064486\_0\_11$: NtBspQI N  31.24  \textcolor<3>{SpringGreen}{16.80  15.77  6.25  35.08  18.12}

Alignment successful!!
\end{frame}

%%%%%%%%%%%% Slide 3 %%%%%%%%%%%%
\begin{frame}
\frametitle{Nanocoding - brief background}
{\emph{M.florum}} reference map: \\
81.6 18.7 59.4 13.9 9.0 5.0 12.3 10.2 15.0 25.4 3.9 20.9 15.6 10.2 9.5 11.1 4.5 
13.7 26.3 38.3 2.1 31.0 19.1 3.6 32.2 41.3 9.8 \textcolor{SpringGreen}{16.4 15.7 6.3 36.5 18.3} 32.1 21.0 
3.3 14.3 51.3 16.0 17.9

Molecule $2064486\_0\_11$: NtBspQI N  31.24  \textcolor{SpringGreen}{16.80  15.77  6.25  35.08  18.12}

\begin{figure}[T]
\includegraphics[scale=1]{NMaps_Aligned.png}
\end{figure}

\end{frame}

%%%%%%%%%%%% Slide 4 %%%%%%%%%%%%
\begin{frame}
\frametitle{Intensity profile of molecules}
\begin{center}
\begin{tikzpicture}
  \node (img1) {\includegraphics[scale=0.28]{molecule251_frame68.png}};
  \pause
  \node (img2) at (img1.north)[yshift=-3cm] {\includegraphics[scale=0.25, page=1]{Frag38_1.pdf}};
\end{tikzpicture}
\end{center}
\end{frame}

%%%%%%%%%%%% Slide 5 %%%%%%%%%%%%
\begin{frame}
\frametitle{Fluoroscanning}
\begin{columns}
\column{.5\textwidth} 
Dimer of oxazole yellow (YOYO-1) is the fluorescent dye added to the DNA molecules (intercalates) \\
\vspace{1cm}
\begin{figure}
\includegraphics[scale=0.3]{DyeMolecule.jpg}
\end{figure}

\pause
\column{.5\textwidth} 
Fluorescent properties of bound YOYO depends strongly on base sequence \\
\vspace{1cm}
{\emph{Quantum yield of YO bound to “GC” sequences is twice as large as those bound to “AT” sequences; binding constants also vary with sequence}}
\begin{figure}
\includegraphics[scale=0.55]{DNAMolecule.jpg}
\end{figure}
\end{columns}
\end{frame}

%%%%%%%%%%%% Slide 6 %%%%%%%%%%%%
\begin{frame}
\frametitle{Fluoroscanning - Aim}
{\bf{\large Establish unique fluorescence intensity signatures of each segment of the genome}}
\begin{itemize}
\item After controlling for variability in fragments aligned to same interval on the genome, we want to differentiate between fragments aligned to two different intervals
\item We want to align intensity profiles of DNA molecules to the reference intensity profile of the genome
\item We want to use uniqueness of intensity profiles to detect large-scale structural variations
\end{itemize}
\end{frame}

%%%%%%%%%%%% Slide 7 %%%%%%%%%%%%
\begin{frame}
\frametitle{Fluoroscanning - Why the next step to personalized genomics?}
\begin{itemize}
\item Pipeline much simpler than sequencing
\item Faster, more economical, simpler way of identifying complex events in the whole genome
\item Can be used to study genomic variations in the same individual, over time (\textcolor{SpringGreen} {establish fluorescence intensity profiles of individuals, instead of having just one reference genome})
\item At the very least, provides information currently missed by next-generation sequences techniques
\end{itemize}
\end{frame}

%%%%%%%%%%%% Slide 8 %%%%%%%%%%%%
\begin{frame}
\frametitle{Alignment of intensity profiles}
\begin{figure}
\includegraphics[scale=0.45, page=2]{Frag38_1.pdf}
\end{figure}
\end{frame}

%%%%%%%%%%%% Slide 9 %%%%%%%%%%%%
\begin{frame}
\frametitle{Consensus intensity profile of a genomic interval}
\begin{figure}
\includegraphics[scale=0.45, page=3]{Frag37.pdf}
\end{figure}
\end{frame}

%%%%%%%%%%%% Slide 10 %%%%%%%%%%%%
\begin{frame}
\frametitle{Consensus intensity profile of a genomic interval}
\begin{figure}
\includegraphics[scale=0.45, page=4]{Frag37.pdf}
\end{figure}
\end{frame}

%%%%%%%%%%%% Slide 11 %%%%%%%%%%%%
\begin{frame}
\frametitle{Distinguish between intervals}
\begin{figure}
\includegraphics[scale=0.35, page=1]{Permute_Iter_Regist_Frag36P6-45_AND_Frag37P6-45_QSqtl40_2015-09-27.pdf}
\includegraphics[scale=0.35, page=3]{Permute_Iter_Regist_Frag36P6-45_AND_Frag37P6-45_QSqtl40_2015-09-27.pdf}
\end{figure}
%Permute_Iter_Regist_Frag36P6-45_AND_Frag37P6-45_QSqtl40_2015-09-27.pdf
\end{frame}

%%%%%%%%%%%% Slide 12 %%%%%%%%%%%%
\begin{frame}
\frametitle{Distinguish between intervals}
\begin{figure}
\includegraphics[scale=0.35, page=2]{Permute_Iter_Regist_Frag36P6-45_AND_Frag37P6-45_QSqtl40_2015-09-27.pdf}
\includegraphics[scale=0.35, page=4]{Permute_Iter_Regist_Frag36P6-45_AND_Frag37P6-45_QSqtl40_2015-09-27.pdf}
\end{figure}
%Permute_Iter_Regist_Frag36P6-45_AND_Frag37P6-45_QSqtl40_2015-09-27.pdf
\end{frame}

%%%%%%%%%%%% Slide 13 %%%%%%%%%%%%
\begin{frame}
\frametitle{Important to assess quality of images}
\begin{figure}
\includegraphics[scale=0.5, page=2]{group2427979_molecule127_GaussianCluster.pdf}
\end{figure}
\end{frame}

%%%%%%%%%%%% Slide 14 %%%%%%%%%%%%
\begin{frame}
\frametitle{Important to assess quality of images}
\begin{figure}
\includegraphics[scale=0.5, page=2]{group2427983_molecule246_GaussianCluster.pdf}
\end{figure}
\end{frame}

%%%%%%%%%%%% Slide 15 %%%%%%%%%%%%
\begin{frame}
\frametitle{Taking only top quality scores}
\begin{figure}
\includegraphics[scale=0.35, page=4]{Frag37.pdf}
\includegraphics[scale=0.35, page=6]{Frag37.pdf}
\end{figure}
\end{frame}

%%%%%%%%%%%% Slide 16 %%%%%%%%%%%%
\begin{frame}
\frametitle{Consensus estimate}
\begin{figure}
\includegraphics[scale=0.35, page=7]{Frag37.pdf}
\includegraphics[scale=0.35, page=8]{Frag37.pdf}
\end{figure}
\end{frame}

%%%%%%%%%%%% Slide 17 %%%%%%%%%%%%
\begin{frame}
\frametitle{Questions? Comments?}

\end{frame}

\end{document}

