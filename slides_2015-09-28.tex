%\documentclass[10pt,dvipsnames,table, handout]{beamer} % To printout the slides without the animations
\documentclass[10pt,dvipsnames,table]{beamer} 
%\usetheme{Luebeck} 
\usetheme{Madrid} 
%\usetheme{Marburg} 
%\usetheme{Warsaw} 
%\setbeamercolor{structure}{fg=cyan!90!white}
\setbeamercolor{normal text}{fg=white, bg=black}

%%%%%%%%%%%%%%%%%%%%%%%%%%%%%%%%%%%%%%%%%%%%%%%%%%%%%%%%%%%%%%%%%%%%%%
%% Input header file 
%%%%%%%%%%%%%%%%%%%%%%%%%%%%%%%%%%%%%%%%%%%%%%%%%%%%%%%%%%%%%%%%%%%%%%
\input{HeaderfileTexSlides}

%%%%%%%%%%%%%%%%%%%%%%%%%%%%%%%%%%%%%%%%%%%%%%%%%%%%%%%%%%%%%%%%%%%%%%
%% TITLE PAGE 
%%%%%%%%%%%%%%%%%%%%%%%%%%%%%%%%%%%%%%%%%%%%%%%%%%%%%%%%%%%%%%%%%%%%%%
\DeclarePairedDelimiter\ceil{\lceil}{\rceil}
\title[Fluoroscanning: next generation nanocoding]{Fluoroscanning: next-gen Nanocoding; \\ {\emph{Realistic path to personalized genomics}}}
\author{Subhrangshu Nandi}
%\institute[Stat 741]{Stat 741, Spring 2015 \\
%  Department of Statistics \\
% University of Wisconsin-Madison}
%\date{April 21, 2015}

\begin{document}
\setlength{\baselineskip}{16truept}
\frame{\maketitle}

%%%%%%%%%%%%%%%%%%%%%%%%%%%%%%%%%%%%%%%%%%%%%%%%%%%%%%%%%%%%%%%%%%%%%%
%% This is for 2015-09-28 Schwartz group meeting:
%% 1. Introduction to fluoroscanning
%% 2. Some statistical challenges of fluoroscanning
%% 3. Quality score & improvement of consensus
%%%%%%%%%%%%%%%%%%%%%%%%%%%%%%%%%%%%%%%%%%%%%%%%%%%%%%%%%%%%%%%%%%%%%%

%% Outline for this presentation:

%%%%%%%%%%%% Slide 1 %%%%%%%%%%%%
\begin{frame}
\frametitle{Nanocoding - brief background}
\begin{figure}[T]
\includegraphics[scale=0.28]{molecule37_frame65.png}
\end{figure}
\end{frame}

%%%%%%%%%%%% Slide 2 %%%%%%%%%%%%
\begin{frame}
\frametitle{Nanocoding - brief background}
\begin{figure}[T]
\includegraphics[scale=0.25]{BarcodeImage.jpg}
\end{figure}

{\emph{M.florum}} reference map: \\
81.6 18.7 59.4 13.9 9.0 5.0 12.3 10.2 15.0 25.4 3.9 20.9 15.6 10.2 9.5 11.1 4.5 
13.7 26.3 38.3 2.1 31.0 19.1 3.6 32.2 41.3 9.8 \textcolor<3>{SpringGreen}{16.4 15.7 6.3 36.5 18.3} 32.1 21.0 
3.3 14.3 51.3 16.0 17.9

\pause
Molecule $2064486\_0\_11$: NtBspQI N  31.24  \textcolor<3>{SpringGreen}{16.80  15.77  6.25  35.08  18.12}

Alignment successful!!
\end{frame}

%%%%%%%%%%%% Slide 2 %%%%%%%%%%%%
\begin{frame}
\frametitle{Estimation Steps ({\emph{M.florum}} data)}
\begin{enumerate}
\item Select the theoretical number of pixels for any length on the genome. Allow for $\pm 5\%$ stretch. 
\item Make all Nmaps (inside $\pm 5\%$ stretch) of the same length, {\textcolor{yellow}{by fitting B-Splines}}
\item Pre-process the Nmaps to select the ones somewhat ``similar'' to each other, {\textcolor{yellow}{by pairwise distance measures}}
\item Choose Nmaps that are not completely flat, {\textcolor{yellow}{by choosing top 25\% or 30\% of most variable Nmaps, per interval/ fragment}}
\item Test for uniqueness, by comparing Nmaps from two intervals of same length, {\textcolor{yellow}{by permutation test}}
\item Align the intensity profiles of Nmaps from the same interval, {\textcolor{yellow}{by iterated penalized registration}}
\item Test for uniqueness again, {\textcolor{yellow}{by permutation test}}
\end{enumerate}
\end{frame}

%%%%%%%%%%%% Slide 3 %%%%%%%%%%%%
\begin{frame}
\frametitle{Base pair to pixel conversion}
$\text{PixelLength} = \frac{\text{BasePairLength}}{\text{ConversionFactor} (209)} $
\begin{table}[ht]
\footnotesize
\centering
\begin{tabular}{r|r|r|r|r}
  \hline
 Frag & CoordStart & CoordEnd & BasePairLength & PixelLength \\
  \hline
  \hline
  0 &            0 &        81621 &        81621 &            391 \\
  1 &        81622 &       100297 &        18675 &             89 \\
  2 &       100298 &       159702 &        59404 &            284 \\
  3 &       159703 &       173646 &        13943 &             67 \\ 
  4 &       173647 &       182672 &         9025 &             43 \\
  5 &       182673 &       187715 &         5042 &             24 \\
  . &            . &            . &            . &              . \\
  . &            . &            . &            . &              . \\
\rowcolor{Gray}
 30 &       582668 &       619168 &        36500 &            175 \\
  . &            . &            . &            . &              . \\
  . &            . &            . &            . &              . \\
 38 &       775344 &       793223 &        17879 &             86 \\
 \hline
 \hline
\end{tabular}
\end{table}
\pause
So, for fragment 30, $175 \pm 5\% =(166, 184)$, Nmaps from 166 pixels to 184 pixels will be considered. \\
\pause
Take off 5 pixels from each end (punctates). Left with Nmaps between 156 and 174 pixels. 
\end{frame}


%%%%%%%%%%%% Slide 4 %%%%%%%%%%%%
\begin{frame}
\frametitle{B-Spline fitting, with minimum GCV criteria}
Goal: Fit $y_i, i = 1, \dots, n$, by a smooth function $x(t_i)$; \\
B-Spline (or any basis expansion of $x(t_i)$ is: $x(t_i) = \sum\limits_{k = 0}^K c_k \phi_k(t_i)$\\
Model: $y_i = x(t_i) + \epsilon _i = \mathbf{\phi'}(t_i)\mathbf{c} + \epsilon _i$ \\
Estimate: $\hat{c} = (\Phi'\Phi)^{-1}\Phi' y$ \\
SSE: $\sum \limits_{i = 1}^N [y_i - x(t_i)]^2$ \\
\pause 
Pen SSE: $$\sum \limits_{i = 1}^N [y_i - x(t_i)]^2 + \lambda \int[D^mx(t)]^2dt$$
$$ \text{GCV}(\lambda) = \frac{n}{n - df(\lambda)}\frac{SSE}{n - df(\lambda)}$$
Minimize $\lambda_c = \arg \min _{\lambda} \text{GCV}(\lambda)$ \\
\textcolor{blue}{For each N-map, the best lambda is chosen, before fitting with B-spline}
\end{frame}

%%%%%%%%%%%% Slide 4 %%%%%%%%%%%%
\begin{frame}
\begin{figure}
\includegraphics[scale=0.52, page=2]{Plots_forMeeting.pdf}
\end{figure}
\end{frame}

%%%%%%%%%%%% Slide 5 %%%%%%%%%%%%
\begin{frame}
\frametitle{All Nmaps aligned to fragment 30, within acceptable stretch limits}
\begin{figure}
\includegraphics[scale=0.43, page=3]{Plots_forMeeting.pdf}
\end{figure}
\end{frame}

%%%%%%%%%%%% Slide 6 %%%%%%%%%%%%
\begin{frame}
\frametitle{Pre-processing - Minimum variability}
Discarding Nmaps that are flat. Taking top $25\%$ of most variable Nmaps.
\begin{figure}
\includegraphics[scale=0.42, page=4]{Plots_forMeeting.pdf}
\end{figure}
\end{frame}

%%%%%%%%%%%% Slide 7 %%%%%%%%%%%%
\begin{frame}
\frametitle{Pre-processing - Similarity between Nmaps }
\[ \rho(f_i, f_j) = \frac{\int _{S_{ij}}f'_i(s)f'_j(s) ds}{\int _{S_{ij}}f'_i(s)^2 ds \int _{S_{ij}}f'_j(s)^2 ds} \]
where, $S_{ij} = S_i \cap S_j$, the intersection of Sobolev spaces formed by the functions $f_i, f_j$.\\
Properties of $\rho(f_i, f_j)$:
\begin{itemize}
\item $|\rho(f_i, f_j)| \leq 1$
\item $|\rho(f_i, f_j)| = 1 \Leftrightarrow \exists \ \ A \in \Real ^+, B \in \Real, \ni f_i = Af_j + B$
\item Absolute value preserved under affine transformations of functions \\
$\rho(f_i, f_j) = \text{sign}(A_i\cdot A_j) \rho(A_if_i + B_i, A_jf_j + B_j)$
\item Also preserved under affine transformations of the abscissa (x-axis or time axis)
\end{itemize}
\end{frame}

%%%%%%%%%%%% Slide 8 %%%%%%%%%%%%
\begin{frame}
\begin{center}
\includegraphics[page=1, scale=0.25]{2curves_Sim_0_4994.pdf}
\includegraphics[page=1, scale=0.25]{2curves_Sim_0_6732.pdf} \\
\includegraphics[page=1, scale=0.25]{2curves_Sim_0_9569.pdf}
\includegraphics[page=1, scale=0.25]{2curves_Sim_1.pdf}
\end{center}
\end{frame}

%%%%%%%%%%%% Slide 9 %%%%%%%%%%%%
\begin{frame}
\frametitle{Pairwise Similarity between Nmaps}
\begin{figure}
\includegraphics[scale=0.47, page=5]{Plots_forMeeting.pdf}
\end{figure}
\end{frame}

%%%%%%%%%%%% Slide 10 %%%%%%%%%%%%
\begin{frame}
\frametitle{Pairwise Similarity between Nmaps, after cutoff}
\begin{figure}
\includegraphics[scale=0.47, page=6]{Plots_forMeeting.pdf}
\end{figure}
\end{frame}

%%%%%%%%%%%% Slide 11 %%%%%%%%%%%%
\begin{frame}
\frametitle{Ready to register, establish uniqueness}
\begin{figure}
\includegraphics[scale=0.47, page=7]{Plots_forMeeting.pdf}
\end{figure}
\end{frame}

%%%%%%%%%%%% Slide 12 %%%%%%%%%%%%
\begin{frame}
\frametitle{Compare two sets of Nmaps}
\begin{center}
\begin{figure}
\includegraphics[scale=0.33, page=1]{Regist_MF_Frag_30_Pixels6To45_2015-07-30.pdf}
\pause
\includegraphics[scale=0.33, page=1]{Regist_MF_Frag_30_Pixels46To85_2015-07-30.pdf}
\end{figure}
\end{center}
{\bf{Question: How do you statistically determine if the two sets of Nmaps are unique to the group?}}
\end{frame}

%%%%%%%%%%%% Slide 12 %%%%%%%%%%%%
\begin{frame}
\frametitle{Permutation test - T-type test statistic}
\begin{enumerate}
\item $\bar{X}_1(t): $ Pointwise mean of Nmaps of interval $1$
\item $\bar{X}_2(t): $ Pointwise mean of Nmaps of interval $2$
\item $n_1, n_2$: Number of Nmaps in intervals 1 and 2
\item $\Var[x_1(t)]: $ Variance of Nmaps of interval 1 at each point $t$ 
\item $\Var[x_2(t)]: $ Variance of Nmaps of interval 2 at each point $t$ 
\item $T(t) = \frac{|\bar{X}_1(t) - \bar{X}_2(t)|}{\sqrt{\frac{1}{n_1}\Var[x_1(t)] + 
\frac{1}{n_2}\Var[x_2(t)]}}$: Absolute value of t-statistic at each point
\item $\bar{T}(t):$ Avg of absolute t-statistic is our Test statistic (measure of difference of two sets of Nmaps)
\item To get a null distribution, permute the Nmaps, and repeat steps 1 through 7
\end{enumerate}
\end{frame}

%%%%%%%%%%%% Slide 13 %%%%%%%%%%%%
\begin{frame}
\begin{figure}
\includegraphics[scale=0.52, page=9]{Plots_forMeeting.pdf}
\end{figure}
\end{frame}

%%%%%%%%%%%% Slide 14 %%%%%%%%%%%%
\begin{frame}
\begin{figure}
\includegraphics[scale=0.52, page=8]{Plots_forMeeting.pdf}
\end{figure}
\end{frame}

%%%%%%%%%%%% Slide 15 %%%%%%%%%%%%
\begin{frame}
\begin{figure}
\includegraphics[scale=0.25, page=1]{Regist_MF_Frag_30_Pixels6To45_2015-07-30.pdf}
\includegraphics[scale=0.25, page=1]{Regist_MF_Frag_30_Pixels46To85_2015-07-30.pdf} \\
\includegraphics[scale=0.25, page=8]{Plots_forMeeting.pdf}
\includegraphics[scale=0.25, page=9]{Plots_forMeeting.pdf}
\end{figure}
\end{frame}

%%%%%%%%%%%% Slide 16 %%%%%%%%%%%%
\begin{frame}
\frametitle{After 1 iteration}
\begin{figure}
\includegraphics[scale=0.33, page=2]{Regist_MF_Frag_30_Pixels6To45_2015-07-30.pdf}
\includegraphics[scale=0.33, page=2]{Regist_MF_Frag_30_Pixels46To85_2015-07-30.pdf} \\
\end{figure}
\end{frame}

%%%%%%%%%%%% Slide 17 %%%%%%%%%%%%
\begin{frame}
\begin{figure}
\includegraphics[scale=0.25, page=2]{Regist_MF_Frag_30_Pixels6To45_2015-07-30.pdf}
\includegraphics[scale=0.25, page=2]{Regist_MF_Frag_30_Pixels46To85_2015-07-30.pdf} \\
\includegraphics[scale=0.25, page=10]{Plots_forMeeting.pdf}
\includegraphics[scale=0.25, page=11]{Plots_forMeeting.pdf}
\end{figure}
\end{frame}

%%%%%%%%%%%% Slide 18 %%%%%%%%%%%%
\begin{frame}
\begin{figure}
\includegraphics[scale=0.25, page=8]{Regist_MF_Frag_30_Pixels6To45_2015-07-30.pdf}
\includegraphics[scale=0.25, page=8]{Regist_MF_Frag_30_Pixels46To85_2015-07-30.pdf} \\
\includegraphics[scale=0.25, page=14]{Plots_forMeeting.pdf}
\includegraphics[scale=0.25, page=15]{Plots_forMeeting.pdf}
\end{figure}
\end{frame}

%%%%%%%%%%%% Slide 19 %%%%%%%%%%%%
\begin{frame}
\frametitle{After 7 iterations}
\begin{figure}
\includegraphics[scale=0.33, page=20]{Regist_MF_Frag_30_Pixels6To45_2015-07-30.pdf}
\includegraphics[scale=0.33, page=20]{Regist_MF_Frag_30_Pixels46To85_2015-07-30.pdf} \\
\end{figure}
\end{frame}

%%%%%%%%%%%% Slide 20 %%%%%%%%%%%%
\begin{frame}
\begin{figure}
\includegraphics[scale=0.25, page=20]{Regist_MF_Frag_30_Pixels6To45_2015-07-30.pdf}
\includegraphics[scale=0.25, page=20]{Regist_MF_Frag_30_Pixels46To85_2015-07-30.pdf} \\
\includegraphics[scale=0.25, page=20]{Plots_forMeeting.pdf}
\includegraphics[scale=0.25, page=21]{Plots_forMeeting.pdf}
\end{figure}
\end{frame}

%%%%%%%%%%%% Slide 21 %%%%%%%%%%%%
\begin{frame}
\frametitle{After 7 iterations}
\begin{figure}
\includegraphics[scale=0.33, page=20]{Regist_MF_Frag_30_Pixels6To45_2015-07-30.pdf}
\includegraphics[scale=0.33, page=21]{Regist_MF_Frag_30_Pixels6To45_2015-07-30.pdf}
\end{figure}
\end{frame}

%% %%%%%%%%%%%% Slide 22 %%%%%%%%%%%%
\begin{frame}
\frametitle{Single Iteration Results}
\begin{figure}
\includegraphics[scale=0.33, page=2]{Regist_MF_Frag_30_Pixels6To45_OneIteration_2015-07-31.pdf}
\includegraphics[scale=0.33, page=2]{Regist_MF_Frag_30_Pixels46To85_OneIteration_2015-07-31.pdf}
\end{figure}
\end{frame}

%% %%%%%%%%%%%% Slide 23 %%%%%%%%%%%%
\begin{frame}
\frametitle{Single Iteration Uniqueness Comparison}
\begin{figure}
\includegraphics[scale=0.33, page=22]{Plots_forMeeting.pdf}
\includegraphics[scale=0.33, page=23]{Plots_forMeeting.pdf}
\end{figure}
\end{frame}

%%%%%%%%%%%%%% Slide 24 %%%%%%%%%%%%
\begin{frame}
\frametitle{Questions? Comments? Next Steps?}



\end{frame}
\end{document}

