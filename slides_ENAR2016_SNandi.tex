%\documentclass[10pt,dvipsnames,table, handout]{beamer} % To printout the slides without the animations
\documentclass[10pt,dvipsnames,table]{beamer} 
%\usetheme{Luebeck} 
%\usetheme{Madrid} 
%\usetheme{Marburg} 
%\usetheme{Warsaw} 
\usetheme{CambridgeUS}
%\setbeamercolor{structure}{fg=cyan!90!white}
%\setbeamercolor{normal text}{fg=white, bg=black}
\setbeamercolor{block title}{bg=red!80,fg=white}

%%%%%%%%%%%%%%%%%%%%%%%%%%%%%%%%%%%%%%%%%%%%%%%%%%%%%%%%%%%%%%%%%%%%%%
%% Input header file 
%%%%%%%%%%%%%%%%%%%%%%%%%%%%%%%%%%%%%%%%%%%%%%%%%%%%%%%%%%%%%%%%%%%%%%
\input{HeaderfileTexSlides}

\logo{\includegraphics[scale=0.4]{uwlogo_web_sm_fl_wht.png}}
%\logo{\includegraphics[width=\beamer@sidebarwidth,height=\beamer@headheight]{uwlogo_web_sm_fl_wht.png}}
%%%%%%%%%%%%%%%%%%%%%%%%%%%%%%%%%%%%%%%%%%%%%%%%%%%%%%%%%%%%%%%%%%%%%%
%% TITLE PAGE 
%%%%%%%%%%%%%%%%%%%%%%%%%%%%%%%%%%%%%%%%%%%%%%%%%%%%%%%%%%%%%%%%%%%%%%

\DeclarePairedDelimiter\ceil{\lceil}{\rceil}
\title[Functional outliers]{Detecting outliers in functional data}
\author[S. Nandi]{Subhrangshu Nandi}
\institute[UW Madison]{
Department of Statistics \\
University of Wisconsin-Madison \\
\vspace{1cm}
ENAR 2016}
\date{March 8, 2016}

\begin{document}
\setlength{\baselineskip}{16truept}
\setbeamertemplate{logo}{}

\frame{\maketitle}

%%%%%%%%%%%%%% Slide 1 %%%%%%%%%%%%%%
\begin{frame}
\frametitle{Objective of this talk}
{\Large{
\begin{enumerate}
\item Explore some functional data depth concepts
\vspace{0.5cm}
\item Use them to detect outliers (distant realizations) in functional data
\vspace{0.5cm}
\item Apply them to an image-processing data of cells
\end{enumerate}
}}
\end{frame}
%%%%%%%%%%%%%%%%%%%%%%%%%%%%%%%%%%%%%

%%%%%%%%%%%%%% Slide x %%%%%%%%%%%%%%
\begin{frame}
\frametitle{What the data looks like}

%\begin{figure}
%\includegraphics
%\end{figure}

\begin{block}{Goal of the project}
To estimate a consensus profile of these realizations
\end{block}
\end{frame}
%%%%%%%%%%%%%%%%%%%%%%%%%%%%%%%%%%%%%

%%%%%%%%%%%%%% Slide x %%%%%%%%%%%%%%
\begin{frame}
\frametitle{Outliers in univariate data}
\vspace{-0.5cm}
\begin{figure}[t]
\centering
\includegraphics[scale = 0.45]{Outliers_Univ/Slide1.PNG}
\end{figure}

\end{frame}
%%%%%%%%%%%%%%%%%%%%%%%%%%%%%%%%%%%%%

%%%%%%%%%%%%%% Slide x %%%%%%%%%%%%%%
\begin{frame}
\frametitle{Outliers in bivariate data}
\vspace{-0.5cm}
\begin{figure}[t]
\centering
\includegraphics[scale = 0.38]{Outliers_Univ/Slide2.PNG}
\end{figure}
\vspace{-0.5cm}
{\footnotesize{Need a measure of multivariate depth in $\Real ^ p$}}
\end{frame}
%%%%%%%%%%%%%%%%%%%%%%%%%%%%%%%%%%%%%

\end{document}

