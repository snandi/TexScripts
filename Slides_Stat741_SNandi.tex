%\documentclass[10pt,dvipsnames,table, handout]{beamer} % To printout the slides without the animations
\documentclass[10pt,dvipsnames,table]{beamer} 
%\usetheme{Luebeck} 
\usetheme{Madrid} 
%\usetheme{Marburg} 
\setbeamercolor{structure}{fg=cyan!90!white}
\setbeamercolor{normal text}{fg=white, bg=black}

%%%%%%%%%%%%%%%%%%%%%%%% Packages %%%%%%%%%%%%%%%%%%%%%%%%
\usepackage{amscd}
\usepackage{amsmath}
\usepackage{amssymb}
\usepackage{amsthm}
\usepackage{amsxtra}
\usepackage{bbold}
%\usepackage{bigints}
\usepackage{color}
\usepackage{dsfont}
\usepackage{enumerate}
\usepackage[mathscr]{eucal}
%\usepackage{fancyhdr}
\usepackage{float}
%\usepackage{fullpage} %% Dont use this for beamer presentations
\usepackage{geometry}
\usepackage{graphicx}
\usepackage{hyperref}
\usepackage{indentfirst}
\usepackage{latexsym}
\usepackage{listings}
\usepackage{lscape}
\usepackage{mathtools}
\usepackage{microtype}
\usepackage{natbib}
\usepackage{pdfpages}
\usepackage{verbatim}
\usepackage{wrapfig}
\usepackage{xargs}
\usepackage{xcolor}
\DeclareGraphicsExtensions{.pdf,.png,.jpg, .jpeg}

%%%%%%%%%%%%%%%%%%%%%%%% Commands %%%%%%%%%%%%%%%%%%%%%%%%
\newcommand{\Sup}{\textsuperscript}
\newcommand{\Exp}{\mathds{E}}
\newcommand{\Prob}{\mathds{P}}
\newcommand{\Z}{\mathds{Z}}
\newcommand{\Ind}{\mathds{1}}
\newcommand{\A}{\mathcal{A}}
\newcommand{\F}{\mathcal{F}}
\newcommand{\G}{\mathcal{G}}
\newcommand{\I}{\mathcal{I}}
\newcommand{\be}{\begin{equation}}
\newcommand{\ee}{\end{equation}}
\newcommand{\bes}{\begin{equation*}}
\newcommand{\ees}{\end{equation*}}
\newcommand{\union}{\bigcup}
\newcommand{\intersect}{\bigcap}
\newcommand{\Ybar}{\overline{Y}}
\newcommand{\ybar}{\bar{y}}
\newcommand{\Xbar}{\overline{X}}
\newcommand{\xbar}{\bar{x}}
\newcommand{\betahat}{\hat{\beta}}
\newcommand{\Yhat}{\widehat{Y}}
\newcommand{\yhat}{\hat{y}}
\newcommand{\Xhat}{\widehat{X}}
\newcommand{\xhat}{\hat{x}}
\newcommand{\E}[1]{\operatorname{E}\left[ #1 \right]}
%\newcommand{\Var}[1]{\operatorname{Var}\left( #1 \right)}
\newcommand{\Var}{\operatorname{Var}}
\newcommand{\Cov}[2]{\operatorname{Cov}\left( #1,#2 \right)}
\newcommand{\N}[2][1=\mu, 2=\sigma^2]{\operatorname{N}\left( #1,#2 \right)}
\newcommand{\bp}[1]{\left( #1 \right)}
\newcommand{\bsb}[1]{\left[ #1 \right]}
\newcommand{\bcb}[1]{\left\{ #1 \right\}}
\newcommand*{\permcomb}[4][0mu]{{{}^{#3}\mkern#1#2_{#4}}}
\newcommand*{\perm}[1][-3mu]{\permcomb[#1]{P}}
\newcommand*{\comb}[1][-1mu]{\permcomb[#1]{C}}


%%%%%%%%%%%%% For explanatory bubbles, use the following code %%%%%%%%%%%%%
%% \usepackage{tikz} %% For explanatory bubbles
%% \usepackage{xparse}
%% \usetikzlibrary{shapes.callouts,ocgx}

%% \newcommand{\tikzmark}[1]{\tikz[overlay,remember picture,baseline=0.5ex] \node (#1) {};}

%% % \explainword: #1= identifier to mark the word, #2 text
%% \NewDocumentCommand{\explainword}{r[] m}{
%%     \switchocg{#1}{#2}\tikzmark{#1}
%% }

%% \tikzset{my callout style/.style={
%%         draw,rectangle callout,anchor=pointer,callout relative pointer={(230:1cm)},
%%         rounded corners,align=center,text width=2cm,fill=cyan!20, 
%%     }
%% }

%% % \mycallout: #1 opacity style, #2 pointer base position, #3= text
%% \NewDocumentCommand{\mycallout}{O{opacity=0.8,text opacity=1} m m}{%
%% \begin{tikzpicture}[remember picture, overlay]
%%  \begin{scope}[ocg={ref=#2,status=invisible,name={#3}}]
%% \node[my callout style,#1]at (#2) {#3};
%% \end{scope}
%% \end{tikzpicture}
%% }
%%%%%%%%%%%%%%%%%%%%%%%%%%%%%%%%%%%%%%%%%%%%%%%%%%%%%%%%%%%%%%%%%

%%%%%%%%%%%%%%%%%%%%%%%% TITLE PAGE %%%%%%%%%%%%%%%%%%%%%%%%
\DeclarePairedDelimiter\ceil{\lceil}{\rceil}
\title[Joint Modeling of Functional and Survival Data]{Functional Principal Component Analysis for Longitudinal and Survival Data \\ Fang Yao\\ Statistica Sinica, 2007}
\author{Subhrangshu Nandi}
\institute[Stat 741]{Stat 741, Spring 2015 \\
  Department of Statistics \\
 University of Wisconsin-Madison}
\date{April 21, 2015}

\begin{document}
\setlength{\baselineskip}{16truept}
\frame{\maketitle}

%%%%%%%%%%%% Slide 1 %%%%%%%%%%%%
%\begin{frame}
%\frametitle{Outline}
%\begin{itemize}
%\pause \item Introduction to functional data
%\pause \item Motivation of potential problem
%\pause \item Objective: Minimize effets of smoothing in two sample tests
%\pause \item Problem definition and two sample test
%\pause \item Some theoretical results
%\pause \item Computational results
%\pause \item Summary
%\end{itemize}
%\end{frame}

%The presentation must follow this template.
%1. Background 
%2. Problem identification
%3. Data examples for the Problem
%4. Innovation and significance of the solution
%5. The solution: both theoretical and computational
%6. Results (both simulation and real data analysis)
%7. Limitation of the solution (conditions of the theorem, computational limitation, model assumption, etc)

%%%%%%%%%%%% Slide 1 %%%%%%%%%%%%
\begin{frame}
\frametitle{Background}
\begin{block}{Cox Proportional Hazard Model}
\[ \lambda_i(t|{\mathbf{Z}}) = \lambda_0(t) \exp\{ {\bf{Z_i}}^T {\mathbf{\beta}}\}, \ \ i = 1, \dots n,\ p \text{ covariates} \]
\end{block}
\pause
\begin{block}{Cox Model, with Time-varying covariate}
\[ \lambda_i(t|{\mathbf{Z_i(s)}}; 0 \leq s \leq t) = \lambda_0(t) \exp\{ {\bf{Z_i(t)}}^T {\mathbf{\beta}}\} \]
\end{block}
\pause
\begin{block}{Cox Model, with Functional covariate}
\begin{itemize}
\item One of the covariates functional variable
\item Measurement error in the functional covariate
\item Need evolution history of functional covariate 
\item \textcolor{green}{Joint modeling} of functional covariate and hazard 
\end{itemize}
\end{block}
\end{frame}
%%%%%%%%%%%%%%%%%%%%%%%%%%%%%%%%%

%%%%%%%%%%%% Slide 2 %%%%%%%%%%%%
\begin{frame}
\frametitle{Example}
Functional covariate of two types of patients:
\begin{center}
\includegraphics[scale=0.17, page=1]{SimPlots.pdf} 
\includegraphics[scale=0.17, page=7]{SimPlots.pdf} 
\includegraphics[scale=0.17, page=11]{SimPlots.pdf} 
\includegraphics[scale=0.17, page=15]{SimPlots.pdf} \\

\includegraphics[scale=0.17, page=6]{SimPlots.pdf} 
\includegraphics[scale=0.17, page=10]{SimPlots.pdf} 
\includegraphics[scale=0.17, page=12]{SimPlots.pdf} 
\includegraphics[scale=0.17, page=18]{SimPlots.pdf} 
\end{center}
\end{frame}

%%%%%%%%%%%% Slide 2 %%%%%%%%%%%%
\begin{frame}
\frametitle{Problem identification}
This paper proposes a nonparametric approach for {\underline{jointly}} modelling longitudinal and survival data using {\underline{functional principal components}}.


\end{frame}
\end{document}

