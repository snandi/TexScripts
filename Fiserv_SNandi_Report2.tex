\documentclass[11pt]{extarticle} %extarticle for fontsizes other than 10, 11 And 12
%\documentclass[11p]{article}

%%%%%%%%%%%%%%%%%%%%%%%%%%%%%%%%%%%%%%%%%%%%%%%%%%%%%%%%%%%%%%%%%%%%%%
%% Input header file 
%%%%%%%%%%%%%%%%%%%%%%%%%%%%%%%%%%%%%%%%%%%%%%%%%%%%%%%%%%%%%%%%%%%%%%
%%%%%%%%%%%%%%%%%%%%%%%% Packages %%%%%%%%%%%%%%%%%%%%%%%%
\usepackage{amscd}
\usepackage{amsmath}
\usepackage{amssymb}
\usepackage{amsthm}
\usepackage{amsxtra}
\usepackage{animate}
\usepackage{bbold}
%\usepackage{bigints}
\usepackage{color, colortbl}
\usepackage{dsfont}
\usepackage{enumerate}
\usepackage[mathscr]{eucal}
%\usepackage{fancyhdr}
\usepackage{float}
%\usepackage{fullpage} %% Dont use this for beamer presentations
\usepackage{geometry}
\usepackage{graphicx}
\usepackage{hyperref}
\usepackage{indentfirst}
\usepackage{latexsym}
\usepackage{listings}
\usepackage{lscape}
\usepackage{mathtools}
\usepackage{microtype}
\usepackage{multirow}
\usepackage{natbib}
\usepackage{pdfpages}
\usepackage{verbatim}
\usepackage{wrapfig}
\usepackage{xargs}
\usepackage{xcolor}
\DeclareGraphicsExtensions{.pdf,.png,.jpg, .jpeg}
\definecolor{LightCyan}{rgb}{0.88,1,1}

%%%%%%%%%%%%%%%%%%%%%%%% Commands %%%%%%%%%%%%%%%%%%%%%%%%
\newcommand{\Sup}{\textsuperscript}
\newcommand{\Exp}{\mathds{E}}
\newcommand{\Prob}{\mathds{P}}
\newcommand{\Z}{\mathds{Z}}
\newcommand{\Ind}{\mathds{1}}
\newcommand{\A}{\mathcal{A}}
\newcommand{\F}{\mathcal{F}}
\newcommand{\G}{\mathcal{G}}
\newcommand{\I}{\mathcal{I}}
\newcommand{\R}{\mathcal{R}}
\newcommand{\Real}{\mathbb{R}}
\newcommand{\be}{\begin{equation}}
\newcommand{\ee}{\end{equation}}
\newcommand{\bes}{\begin{equation*}}
\newcommand{\ees}{\end{equation*}}
\newcommand{\union}{\bigcup}
\newcommand{\intersect}{\bigcap}
\newcommand{\Ybar}{\overline{Y}}
\newcommand{\ybar}{\bar{y}}
\newcommand{\Xbar}{\overline{X}}
\newcommand{\xbar}{\bar{x}}
\newcommand{\betahat}{\hat{\beta}}
\newcommand{\Yhat}{\widehat{Y}}
\newcommand{\yhat}{\hat{y}}
\newcommand{\Xhat}{\widehat{X}}
\newcommand{\xhat}{\hat{x}}
\newcommand{\E}[1]{\operatorname{E}\left[ #1 \right]}
%\newcommand{\Var}[1]{\operatorname{Var}\left( #1 \right)}
\newcommand{\Var}{\operatorname{Var}}
\newcommand{\Cov}[2]{\operatorname{Cov}\left( #1,#2 \right)}
\newcommand{\N}[2][1=\mu, 2=\sigma^2]{\operatorname{N}\left( #1,#2 \right)}
\newcommand{\bp}[1]{\left( #1 \right)}
\newcommand{\bsb}[1]{\left[ #1 \right]}
\newcommand{\bcb}[1]{\left\{ #1 \right\}}
\newcommand*{\permcomb}[4][0mu]{{{}^{#3}\mkern#1#2_{#4}}}
\newcommand*{\perm}[1][-3mu]{\permcomb[#1]{P}}
\newcommand*{\comb}[1][-1mu]{\permcomb[#1]{C}}


\geometry{left=0.8in, right=0.8in, top=1in, bottom=0.8in}
\begin{document}
%\SweaveOpts{concordance=TRUE}
\bibliographystyle{plain}  %Choose a bibliograhpic style

\title{Report 2: Evaluate Impact of LDP}
\author{Subhrangshu Nandi\\
  Statistics PhD Student, \\
%  Research Assistant,
%  Laboratory of Molecular and Computational Genomics, \\
  University of Wisconsin - Madison \\
  nands31@gmail.com}
\date{December 17, 2015}
%\date{}

\maketitle

\newpage
\section*{Analysis of Your Voice survey}
Answers of 2015 survey outputs of the following categories were analyzed, to detect any impact of LDP training:
Direct Report Score, Manager Effectiveness, Quality, Trust, Growth and Development, Recognition, Business Acumen, Client Focus, Market Insight and  Communication. Wherever available, their 2014 scores were used as a baseline value for their 2015 numbers. For each category, the following analysis were conducted:
\begin{enumerate}
\item Boxplot - To visually detect any difference between people with and without LDP training
\item T-test - To detect any significant difference in scores between people with and without LDP training
\item Linear model - To estimate the effect of LDP training on the 2015 scores, controlling for their baseline 2014 scores. 
\[ y = \beta_0 + \beta_1 x + \beta_2 z\]
where, $y: $ 2015 score; $x: $ 2014 score of the same person; $z: $ If the person completed LDP or not. In each of the analysis, the null hypothesis is that LDP has not made any difference to their scores. If the coefficient $\beta_2$ is statistically significant (p-value $<$ 0.05), then we reject the null hypothesis, and conclude that LDP did have an impact. 

\end{enumerate}
Below are the results: \\
\subsection*{Analysis Results}
\begin{enumerate}
\item {\bf{Direct Report Scores}}\\
% first column
\begin{minipage}[t]{0.3\textwidth}
\begin{figure}[H]
\centering 
\includegraphics[scale=0.3, page=1]{BoxPlots-DR.pdf}
\end{figure}
\end{minipage}
%second column
\begin{minipage}[t]{0.6\textwidth}
\vspace{0.8cm}
t-test: p-value = 0.0000 \\
The model fit is:
% latex table generated in R 3.1.1 by xtable 1.7-4 package
% Wed Dec 16 14:51:50 2015
\begin{table}[H]
\centering
\begin{tabular}{rrrrr}
  \hline
 & Estimate & Std. Error & t value & p-value \\ 
  \hline
  (Intercept) & 0.7022 & 0.0192 & 36.53 & $0.0000^{**}$ \\ 
  Data2014 & 0.1268 & 0.0217 & 5.85 & $0.0000^{**}$ \\ 
  LDP No & -0.0521 & 0.0173 & -3.02 & $0.0027^{**}$ \\ 
   \hline
\end{tabular}
\end{table}
Conclusion: Both, t-test, and linear model, suggest that LDP did have an impact in yielding better direct report scores. p-value is 0.0027. 
\end{minipage}

\item {\bf{Manager Effectiveness}}\\
\begin{minipage}[t]{0.3\textwidth}
\begin{figure}[H]
\centering 
\includegraphics[scale=0.3, page=2]{BoxPlots-DR.pdf}
\end{figure}
\end{minipage}
%second column
\begin{minipage}[t]{0.6\textwidth}
\vspace{0.8cm}
t-test p-value = 0.03124 \\
The model fit is:
% latex table generated in R 3.1.1 by xtable 1.7-4 package
% Wed Dec 16 14:51:50 2015
\begin{table}[H]
\centering
\begin{tabular}{rrrrr}
  \hline
 & Estimate & Std. Error & t value & p-value \\ 
  \hline
(Intercept) & 0.7120 & 0.0201 & 35.39 & 0.0000 \\ 
  Data 2014 & 0.1446 & 0.0204 & 7.10 & 0.0000 \\ 
  LDP (No) & -0.0236 & 0.0181 & -1.30 & 0.1924 \\ 
   \hline
\end{tabular}
\end{table}
Conclusion: Although t-test yields significant different, when controlling for 2014 scores the inference changes. LDP did not seem to improve manager effectiveness.
\end{minipage}

\item {\bf{Quality}}\\
\begin{minipage}[t]{0.3\textwidth}
\begin{figure}[H]
\centering 
\includegraphics[scale=0.3, page=3]{BoxPlots-DR.pdf}
\end{figure}
\end{minipage}
%second column
\begin{minipage}[t]{0.6\textwidth}
\vspace{0.8cm}
t-test p-value = 0.1886 \\
The model fit is:
% latex table generated in R 3.1.1 by xtable 1.7-4 package
% Wed Dec 16 14:51:50 2015
\begin{table}[H]
\centering
\begin{tabular}{rrrrr}
  \hline
 & Estimate & Std. Error & t value & p-value \\ 
  \hline
(Intercept) & 0.6528 & 0.0188 & 34.79 & 0.0000 \\ 
  Data 2014 & 0.1249 & 0.0219 & 5.71 & 0.0000 \\ 
  LDP (No) & -0.0100 & 0.0171 & -0.59 & 0.5579 \\ 
   \hline
\end{tabular}
\end{table}
Conclusion: Neither the t-test, nor the linear model show any impact of LDP on quality. 
\end{minipage}

\item {\bf{Trust}}\\
\begin{minipage}[t]{0.3\textwidth}
\begin{figure}[H]
\centering 
\includegraphics[scale=0.3, page=4]{BoxPlots-DR.pdf}
\end{figure}
\end{minipage}
%second column
\begin{minipage}[t]{0.6\textwidth}
\vspace{0.8cm}
t-test p-value = 0.0066 \\
The model fit is:
% latex table generated in R 3.1.1 by xtable 1.7-4 package
% Wed Dec 16 14:51:51 2015
\begin{table}[H]
\centering
\begin{tabular}{rrrrr}
  \hline
 & Estimate & Std. Error & t value & p-value \\ 
  \hline
(Intercept) & 0.8397 & 0.0166 & 50.50 & 0.0000 \\ 
  Data 2014 & 0.0647 & 0.0151 & 4.28 & 0.0000 \\ 
  LDP (No) & -0.0271 & 0.0145 & -1.87 & 0.0624 \\ 
   \hline
\end{tabular}
\end{table}
Conclusion: The t-test is more significant than the linear model results. As per the linear model, LDP did improve trust, but only at a 0.10 significance level (not the conventional 0.05)
\end{minipage}

\item {\bf{Growth and Development}}\\
\begin{minipage}[t]{0.3\textwidth}
\begin{figure}[H]
\centering 
\includegraphics[scale=0.3, page=5]{BoxPlots-DR.pdf}
\end{figure}
\end{minipage}
%second column
\begin{minipage}[t]{0.6\textwidth}
\vspace{0.8cm}
t-test p-value = 0.0097 \\
The model fit is:
% latex table generated in R 3.1.1 by xtable 1.7-4 package
% Wed Dec 16 14:51:51 2015
\begin{table}[H]
\centering
\begin{tabular}{rrrrr}
  \hline
 & Estimate & Std. Error & t value & p-value \\ 
  \hline
(Intercept) & 0.7744 & 0.0194 & 39.82 & 0.0000 \\ 
  Data 2014 & 0.0953 & 0.0191 & 5.00 & 0.0000 \\ 
  LDP (No) & -0.0296 & 0.0174 & -1.70 & 0.0890 \\ 
   \hline
\end{tabular}
\end{table}
Conclusion: The t-test is more significant than the linear model results. As per the linear model, LDP did improve growth and development, but only at a 0.10 significance level (not the conventional 0.05)
\end{minipage}

\item {\bf{Recognition}}\\
\begin{minipage}[t]{0.3\textwidth}
\begin{figure}[H]
\centering 
\includegraphics[scale=0.3, page=6]{BoxPlots-DR.pdf}
\end{figure}
\end{minipage}
%second column
\begin{minipage}[t]{0.6\textwidth}
\vspace{0.8cm}
t-test p-value = 0.00034 \\
The model fit is:
% latex table generated in R 3.1.1 by xtable 1.7-4 package
% Wed Dec 16 14:51:51 2015
\begin{table}[H]
\centering
\begin{tabular}{rrrrr}
  \hline
 & Estimate & Std. Error & t value & p-value \\ 
  \hline
(Intercept) & 0.8209 & 0.0189 & 43.47 & 0.0000 \\ 
  Data 2014 & 0.0821 & 0.0176 & 4.67 & 0.0000 \\ 
  LDP (No) & -0.0413 & 0.0166 & -2.48 & 0.0134 \\ 
  \hline
\end{tabular}
\end{table}
Conclusion: Both, the t-test and linear model show very significant impact of LDP in improving recognition. p-value is 0.0134. 
\end{minipage}

\item {\bf{Business Acumen}}\\
\begin{minipage}[t]{0.3\textwidth}
\begin{figure}[H]
\centering 
\includegraphics[scale=0.3, page=7]{BoxPlots-DR.pdf}
\end{figure}
\end{minipage}
%second column
\begin{minipage}[t]{0.6\textwidth}
\vspace{0.8cm}
t-test p-value = 0.0000 \\
The model fit is:
% latex table generated in R 3.1.1 by xtable 1.7-4 package
% Wed Dec 16 17:22:36 2015
\begin{table}[H]
\centering
\begin{tabular}{rrrrr}
  \hline
 & Estimate & Std. Error & t value & p-value \\ 
  \hline
(Intercept) & 0.7757 & 0.0207 & 37.56 & 0.0000 \\ 
  Data 2014 & 0.1563 & 0.0190 & 8.23 & 0.0000 \\ 
  LDP (No) & -0.0557 & 0.0182 & -3.06 & 0.0023 \\ 
   \hline
\end{tabular}
\end{table}
Conclusion: Both, the t-test and linear model show very significant impact of LDP in improving business acumen. p-value is 0.0023. 
\end{minipage}

\item {\bf{Client Focus}}\\
\begin{minipage}[t]{0.3\textwidth}
\begin{figure}[H]
\centering 
\includegraphics[scale=0.3, page=8]{BoxPlots-DR.pdf}
\end{figure}
\end{minipage}
%second column
\begin{minipage}[t]{0.6\textwidth}
\vspace{0.8cm}
t-test p-value = 0.0011 \\
The model fit is:
% latex table generated in R 3.1.1 by xtable 1.7-4 package
% Wed Dec 16 14:51:52 2015
\begin{table}[H]
\centering
\begin{tabular}{rrrrr}
  \hline
 & Estimate & Std. Error & t value & p-value \\ 
  \hline
(Intercept) & 0.8509 & 0.0179 & 47.47 & 0.0000 \\ 
  Data 2014 & 0.0816 & 0.0160 & 5.11 & 0.0000 \\ 
  LDP (No) & -0.0345 & 0.0157 & -2.19 & 0.0286 \\ 
   \hline
\end{tabular}
\end{table}
Conclusion: Both, the t-test and linear model show very significant impact of LDP in improving client focus. p-value is 0.0286. 
\end{minipage}

\item {\bf{Market Insight}}\\
\begin{minipage}[t]{0.3\textwidth}
\begin{figure}[H]
\centering 
\includegraphics[scale=0.3, page=9]{BoxPlots-DR.pdf}
\end{figure}
\end{minipage}
%second column
\begin{minipage}[t]{0.6\textwidth}
\vspace{0.8cm}
t-test p-value = 0.0000 \\
The model fit is:
% latex table generated in R 3.1.1 by xtable 1.7-4 package
% Wed Dec 16 14:51:53 2015
\begin{table}[H]
\centering
\begin{tabular}{rrrrr}
  \hline
 & Estimate & Std. Error & t value & p-value \\ 
  \hline
(Intercept) & 0.7512 & 0.0208 & 36.15 & 0.0000 \\ 
  Data 2014 & 0.1723 & 0.0192 & 8.97 & 0.0000 \\ 
  LDP (No) & -0.0448 & 0.0183 & -2.45 & 0.0147 \\ 
   \hline
\end{tabular}
\end{table}
Conclusion: Both, the t-test and linear model show very significant impact of LDP in improving market insight. p-value is 0.0147. 
\end{minipage}

\item {\bf{Communication}}\\
\begin{minipage}[t]{0.3\textwidth}
\begin{figure}[H]
\centering 
\includegraphics[scale=0.3, page=10]{BoxPlots-DR.pdf}
\end{figure}
\end{minipage}
%second column
\begin{minipage}[t]{0.6\textwidth}
\vspace{0.8cm}
t-test p-value = 0.0004 \\
The model fit is:
% latex table generated in R 3.1.1 by xtable 1.7-4 package
% Wed Dec 16 14:51:53 2015
\begin{table}[H]
\centering
\begin{tabular}{rrrrr}
  \hline
 & Estimate & Std. Error & t value & p-value \\ 
  \hline
(Intercept) & 0.6574 & 0.0229 & 28.71 & 0.0000 \\ 
  Data 2014 & 0.1275 & 0.0280 & 4.56 & 0.0000 \\ 
  LDP (No) & -0.0607 & 0.0211 & -2.87 & 0.0042 \\ 
   \hline
\end{tabular}
\end{table}
Conclusion: Both, the t-test and linear model show very significant impact of LDP in improving communication. p-value is 0.0042. 
\end{minipage}
\end{enumerate}

\section*{Analysis of 2015 Bonus compensation}
Below are plots of the 2015 bonus compensation broken down by whether the leaders completed LDP training or not, and also, whether they attended SLM training or not.
\begin{figure}[H]
\centering 
\includegraphics[scale=0.45, page=1]{BoxPlots-Bonus.pdf}
\includegraphics[scale=0.45, page=2]{BoxPlots-Bonus.pdf}
\end{figure}
As is evident, there is not much difference in bonus compensation when broken down by LDP completion. However, SLM seems to have strong influence. Below is a boxplot broken down by LDP and SLM simultaneously.
\begin{figure}[H]
\centering 
\includegraphics[scale=0.55, page=3]{BoxPlots-Bonus.pdf}
\end{figure}
Fitting this model, with SLM and LDP, and controlling for 2014 compensation as a baseline, we have:
% latex table generated in R 3.1.1 by xtable 1.7-4 package
% Wed Dec 16 23:42:06 2015
\begin{table}[H]
\centering
\begin{tabular}{rrrrr}
  \hline
 & Estimate & Std. Error & t value & p-value \\ 
  \hline
  (Intercept) & 47501 & 4731.3547 & 10.04 & 0.0000 \\ 
  Bonus\_2014 & 0.15 & 0.0448 & 3.35 & 0.0009 \\ 
  SLMYes & 35607 & 7439.0652 & 4.79 & 0.0000 \\ 
  LDP.CompletionNo & -7995 & 7070.1123 & -1.13 & 0.2592 \\ 
  \hline
\end{tabular}
\end{table}
The model fit confirms that although leaders who completed LDP receive an average of \$7,995 higher than those who did not complete LDP, but that difference is not statistically significant. Leaders who attended SLM on an average receive \$35,607 higher than who did not. This difference is statistically significant (p-value = 0).

In addition to analyzing the bonus amount, the ``percentage bonus allocated'' variable was also analyzed. Below is a plot of this variable:
\begin{figure}[H]
\centering 
\includegraphics[scale=0.55, page=3]{BoxPlots-Bonus_pct.pdf}
\end{figure}
The model fitting bonus percentage is below:
% latex table generated in R 3.2.2 by xtable 1.8-0 package
% Mon Dec 21 23:22:38 2015
\begin{table}[H]
\centering
\begin{tabular}{rrrrr}
  \hline
 & Estimate & Std. Error & t value & p-valie \\ 
  \hline
(Intercept) & 0.0142 & 0.0055 & 2.58 & 0.0105 \\ 
  Bonus\_2014\_pct & 0.9355 & 0.0173 & 54.13 & 0.0000 \\ 
  SLMYes & 0.0175 & 0.0076 & 2.31 & 0.0216 \\ 
  LDP.CompletionNo & 0.0108 & 0.0072 & 1.49 & 0.1378 \\ 
   \hline
\end{tabular}
\end{table}
Neither the boxplot, nor the linear model indicate any impact of LDP completion on bonus percentage. 


\section*{Summary}
Below are the attributes, ranked in order of decreasing impact of LDP
\begin{table}[H]
\centering
\begin{tabular}{l|c|l}
\hline
\hline
Attribute & p-value & Comment \\
\hline
Business Acumen 	& 0.0023 & Strongly Significant \\
Direct Report Score 	& 0.0027 & Strongly significant \\
Communication 		& 0.0042 & Strongly significant \\
Recognition 		& 0.0134 & Significant \\
Market Insight 		& 0.0147 & Significant \\
Client Focus 		& 0.0286 & Significant \\
Trust 			& 0.0624 & Somewhat significant \\
Growth and Development 	& 0.0890 & Somewhat significant \\
Manager Effectiveness 	& 0.1924 & Not significant \\
Quality 		& 0.5579 & Not significant \\
\hline
\hline
\end{tabular}
\end{table}

Bonus compensation in 2015 was more strongly associated with leaders attending SLM and not as much by LDP completion.

\end{document}
