\documentclass[10pt,dvipsnames,table, handout]{beamer} % To printout the slides without the animations
%\documentclass[10pt,dvipsnames,table]{beamer} 
%\usetheme{Luebeck} 
\usetheme{Madrid} 
%\usetheme{Marburg} 
\setbeamercolor{structure}{fg=cyan!90!white}
%\setbeamercolor{normal text}{fg=white, bg=black}

%%%%%%%%%%%%%%%%%%%%%%%% Packages %%%%%%%%%%%%%%%%%%%%%%%%
\usepackage{amscd}
\usepackage{amsmath}
\usepackage{amssymb}
\usepackage{amsthm}
\usepackage{amsxtra}
\usepackage{bbold}
%\usepackage{bigints}
\usepackage{color}
\usepackage{dsfont}
\usepackage{enumerate}
\usepackage[mathscr]{eucal}
%\usepackage{fancyhdr}
\usepackage{float}
%\usepackage{fullpage} %% Dont use this for beamer presentations
\usepackage{geometry}
\usepackage{graphicx}
\usepackage{hyperref}
\usepackage{indentfirst}
\usepackage{latexsym}
\usepackage{listings}
\usepackage{lscape}
\usepackage{mathtools}
\usepackage{microtype}
\usepackage{natbib}
\usepackage{pdfpages}
\usepackage{verbatim}
\usepackage{wrapfig}
\usepackage{xargs}
\usepackage{xcolor}
\DeclareGraphicsExtensions{.pdf,.png,.jpg, .jpeg}

%%%%%%%%%%%%%%%%%%%%%%%% Commands %%%%%%%%%%%%%%%%%%%%%%%%
\newcommand{\Sup}{\textsuperscript}
\newcommand{\Exp}{\mathds{E}}
\newcommand{\Prob}{\mathds{P}}
\newcommand{\Z}{\mathds{Z}}
\newcommand{\Ind}{\mathds{1}}
\newcommand{\A}{\mathcal{A}}
\newcommand{\F}{\mathcal{F}}
\newcommand{\G}{\mathcal{G}}
\newcommand{\I}{\mathcal{I}}
\newcommand{\be}{\begin{equation}}
\newcommand{\ee}{\end{equation}}
\newcommand{\bes}{\begin{equation*}}
\newcommand{\ees}{\end{equation*}}
\newcommand{\union}{\bigcup}
\newcommand{\intersect}{\bigcap}
\newcommand{\Ybar}{\overline{Y}}
\newcommand{\ybar}{\bar{y}}
\newcommand{\Xbar}{\overline{X}}
\newcommand{\xbar}{\bar{x}}
\newcommand{\betahat}{\hat{\beta}}
\newcommand{\Yhat}{\widehat{Y}}
\newcommand{\yhat}{\hat{y}}
\newcommand{\Xhat}{\widehat{X}}
\newcommand{\xhat}{\hat{x}}
\newcommand{\E}[1]{\operatorname{E}\left[ #1 \right]}
%\newcommand{\Var}[1]{\operatorname{Var}\left( #1 \right)}
\newcommand{\Var}{\operatorname{Var}}
\newcommand{\Cov}[2]{\operatorname{Cov}\left( #1,#2 \right)}
\newcommand{\N}[2][1=\mu, 2=\sigma^2]{\operatorname{N}\left( #1,#2 \right)}
\newcommand{\bp}[1]{\left( #1 \right)}
\newcommand{\bsb}[1]{\left[ #1 \right]}
\newcommand{\bcb}[1]{\left\{ #1 \right\}}
\newcommand*{\permcomb}[4][0mu]{{{}^{#3}\mkern#1#2_{#4}}}
\newcommand*{\perm}[1][-3mu]{\permcomb[#1]{P}}
\newcommand*{\comb}[1][-1mu]{\permcomb[#1]{C}}

%%%%%%%%%%%%% For explanatory bubbles, use the following code %%%%%%%%%%%%%
%% \usepackage{tikz} %% For explanatory bubbles
%% \usepackage{xparse}
%% \usetikzlibrary{shapes.callouts,ocgx}

%% \newcommand{\tikzmark}[1]{\tikz[overlay,remember picture,baseline=0.5ex] \node (#1) {};}

%% % \explainword: #1= identifier to mark the word, #2 text
%% \NewDocumentCommand{\explainword}{r[] m}{
%%     \switchocg{#1}{#2}\tikzmark{#1}
%% }

%% \tikzset{my callout style/.style={
%%         draw,rectangle callout,anchor=pointer,callout relative pointer={(230:1cm)},
%%         rounded corners,align=center,text width=2cm,fill=cyan!20, 
%%     }
%% }

%% % \mycallout: #1 opacity style, #2 pointer base position, #3= text
%% \NewDocumentCommand{\mycallout}{O{opacity=0.8,text opacity=1} m m}{%
%% \begin{tikzpicture}[remember picture, overlay]
%%  \begin{scope}[ocg={ref=#2,status=invisible,name={#3}}]
%% \node[my callout style,#1]at (#2) {#3};
%% \end{scope}
%% \end{tikzpicture}
%% }
%%%%%%%%%%%%%%%%%%%%%%%%%%%%%%%%%%%%%%%%%%%%%%%%%%%%%%%%%%%%%%%%%

%%%%%%%%%%%%%%%%%%%%%%%% TITLE PAGE %%%%%%%%%%%%%%%%%%%%%%%%
\DeclarePairedDelimiter\ceil{\lceil}{\rceil}
\title[Status Update Mar '15]{Status Update Meeting}
\author{S. Nandi}
\institute[LMCG]{LMCG \\
 University of Wisconsin-Madison}
\date{March 13, 2015}

\begin{document}
\setlength{\baselineskip}{16truept}
\frame{\maketitle}

%%%%%%%%%%%% Slide 1 %%%%%%%%%%%%
\begin{frame}
\frametitle{Outline}
\begin{itemize}       
\item Iterated Registration
\item Amplitude and phase variability
\item Distance between two curves
\end{itemize}
\end{frame}

%%%%%%%%%%%% Slide 2 %%%%%%%%%%%%
\begin{frame}
\frametitle{Registration}
\begin{enumerate}
\item Let $x(t)$ be the true curve; $y(t)$ be the curve to register. 
\item Let $h(t)$ be the warping function; $y[h(t)]$ be the registered curve.
\item Objective: Minimize the penalized squared error criterion
\[ F_{\lambda}(y, x|h) = \int \| y(t) - x\{h(t)\} \| ^2 dt + \lambda \int w^2 (t)dt \]
where, $w(t) = \frac{D^2(h)}{D(h)}$. Penalizing $w$ ensures monotinicity and smoothness of the warping function
\item \textcolor{red}{Working on: Registering $x'(t)$ and $x''(t)$ and reconstructing the signals back.}
\end{enumerate}
\end{frame}

%%%%%%%%%%%% Slide 3 %%%%%%%%%%%%
\begin{frame}
\frametitle{Iterated Registration}
Sangalli, etal, 2009, {\emph{JASA}} - A case study in exploratory functional data analysis: geometrical features of the internal carotid artery
\begin{enumerate}
\item Iterated registration: Iterated the warped functions to new mean (EM algorithm)
\item Constrained phase shifts ($\pm 5$ mm) and signal amplifications ($\pm 10\%$) at each ``M'' step
\item Registered the first derivatives
\item Similarity index between functions:
\[ \rho(f_i, f_j) = \frac{\int _{S_{ij}}f'_i(s)f'_j(s) ds}{\int _{S_{ij}}f'_i(s)^2 ds \int _{S_{ij}}f'_j(s)^2 ds} \]
where, $S_{ij} = S_i \cap S_j$, the intersection of Sobolev spaces formed by the functions $f_i, f_j$
\end{enumerate}
%\pause
\begin{center}
\textcolor{red}{Holy grail!!}
\end{center}
\end{frame}

%%%%%%%%%%%% Slide 3 %%%%%%%%%%%%
\begin{frame}
\frametitle{Phase and Amplitude Variability}
Vantini, 2012, {\emph{Test}} - On the definition of phase and amplitude variability in functional data analysis
\begin{enumerate}
\item Criticism of Kneip and Ramsay's approach. 
\item Introduces phase, amplitude and ancillary variabilities
\item Introduces amplitude-to-total variability ratio
\item Quite abstract, but useful
\end{enumerate}
\end{frame}

%%%%%%%%%%%% Slide 4 %%%%%%%%%%%%
\begin{frame}
\frametitle{Bayesian Hierarchical Joint clustering and registration}
Zhang, Telesca, 2014 - Joint Clustering and Registration of Functional Data
\begin{enumerate}
\item Uses Dirichlet process mixture model for shape function parameters - does not require specification of number of clusters beforehand
\item Uses B-splines to represent shape and time transformation functions
\item Provides exact inference with MCMC
\item R-package exists on model based clustering of functional data (MFDA)
\end{enumerate}
\end{frame}

%%%%%%%%%%%% Slide 5 %%%%%%%%%%%%
\begin{frame}
\frametitle{Next Steps}
\begin{enumerate}
\item Implement Sangalli's methods
\item Registers first and second derivatives
\item Initial mean - nonlinearly weighted by stretch
\item Investigate how much phase variability is too much
\end{enumerate}
\end{frame}

%%%%%%%%%%%% Slide 6 %%%%%%%%%%%%
\begin{frame}
\frametitle{Some facts - Chr Y}
Around 2,522 fragments
\vspace{1cm}
Around 970 fragments have at least 10 Nmaps aligned (gold) to them
\vspace{1cm}
Around 285 of them have either \\
``genomic signal'' \\ or \\ ``some recapitulating feature in the Nmap intensities''
\vspace{1cm}
At best, we can discern $11\%$ of Chr Y.
\end{frame}

%%%%%%%%%%%% Slide 7 %%%%%%%%%%%%
\begin{frame}
\frametitle{Misc}
\begin{enumerate}
\item IDP
\vspace{2cm}
\item Project Goals
\vspace{2cm}
\item Personal Goals
\end{enumerate}
\end{frame}

\end{document}

