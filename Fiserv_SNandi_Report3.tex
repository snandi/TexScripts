\documentclass[11pt]{extarticle} %extarticle for fontsizes other than 10, 11 And 12
%\documentclass[11p]{article}

%%%%%%%%%%%%%%%%%%%%%%%%%%%%%%%%%%%%%%%%%%%%%%%%%%%%%%%%%%%%%%%%%%%%%%
%% Input header file 
%%%%%%%%%%%%%%%%%%%%%%%%%%%%%%%%%%%%%%%%%%%%%%%%%%%%%%%%%%%%%%%%%%%%%%
%%%%%%%%%%%%%%%%%%%%%%%% Packages %%%%%%%%%%%%%%%%%%%%%%%%
\usepackage{amscd}
\usepackage{amsmath}
\usepackage{amssymb}
\usepackage{amsthm}
\usepackage{amsxtra}
\usepackage{animate}
\usepackage{bbold}
%\usepackage{bigints}
\usepackage{color, colortbl}
\usepackage{dsfont}
\usepackage{enumerate}
\usepackage[mathscr]{eucal}
%\usepackage{fancyhdr}
\usepackage{float}
%\usepackage{fullpage} %% Dont use this for beamer presentations
\usepackage{geometry}
\usepackage{graphicx}
\usepackage{hyperref}
\usepackage{indentfirst}
\usepackage{latexsym}
\usepackage{listings}
\usepackage{lscape}
\usepackage{mathtools}
\usepackage{microtype}
\usepackage{multirow}
\usepackage{natbib}
\usepackage{pdfpages}
\usepackage{verbatim}
\usepackage{wrapfig}
\usepackage{xargs}
\usepackage{xcolor}
\DeclareGraphicsExtensions{.pdf,.png,.jpg, .jpeg}
\definecolor{LightCyan}{rgb}{0.88,1,1}

%%%%%%%%%%%%%%%%%%%%%%%% Commands %%%%%%%%%%%%%%%%%%%%%%%%
\newcommand{\Sup}{\textsuperscript}
\newcommand{\Exp}{\mathds{E}}
\newcommand{\Prob}{\mathds{P}}
\newcommand{\Z}{\mathds{Z}}
\newcommand{\Ind}{\mathds{1}}
\newcommand{\A}{\mathcal{A}}
\newcommand{\F}{\mathcal{F}}
\newcommand{\G}{\mathcal{G}}
\newcommand{\I}{\mathcal{I}}
\newcommand{\R}{\mathcal{R}}
\newcommand{\Real}{\mathbb{R}}
\newcommand{\be}{\begin{equation}}
\newcommand{\ee}{\end{equation}}
\newcommand{\bes}{\begin{equation*}}
\newcommand{\ees}{\end{equation*}}
\newcommand{\union}{\bigcup}
\newcommand{\intersect}{\bigcap}
\newcommand{\Ybar}{\overline{Y}}
\newcommand{\ybar}{\bar{y}}
\newcommand{\Xbar}{\overline{X}}
\newcommand{\xbar}{\bar{x}}
\newcommand{\betahat}{\hat{\beta}}
\newcommand{\Yhat}{\widehat{Y}}
\newcommand{\yhat}{\hat{y}}
\newcommand{\Xhat}{\widehat{X}}
\newcommand{\xhat}{\hat{x}}
\newcommand{\E}[1]{\operatorname{E}\left[ #1 \right]}
%\newcommand{\Var}[1]{\operatorname{Var}\left( #1 \right)}
\newcommand{\Var}{\operatorname{Var}}
\newcommand{\Cov}[2]{\operatorname{Cov}\left( #1,#2 \right)}
\newcommand{\N}[2][1=\mu, 2=\sigma^2]{\operatorname{N}\left( #1,#2 \right)}
\newcommand{\bp}[1]{\left( #1 \right)}
\newcommand{\bsb}[1]{\left[ #1 \right]}
\newcommand{\bcb}[1]{\left\{ #1 \right\}}
\newcommand*{\permcomb}[4][0mu]{{{}^{#3}\mkern#1#2_{#4}}}
\newcommand*{\perm}[1][-3mu]{\permcomb[#1]{P}}
\newcommand*{\comb}[1][-1mu]{\permcomb[#1]{C}}


\geometry{left=0.8in, right=0.8in, top=1in, bottom=0.8in}
\begin{document}
%\SweaveOpts{concordance=TRUE}
\bibliographystyle{plain}  %Choose a bibliograhpic style

\title{Evaluation of Impact of LDP - Final Report}
\author{Subhrangshu Nandi\\
  Statistics PhD Student, \\
%  Research Assistant,
%  Laboratory of Molecular and Computational Genomics, \\
  University of Wisconsin - Madison \\
  nands31@gmail.com}
\date{December 17, 2015}
%\date{}

\maketitle
\newpage

\tableofcontents
\newpage
\section{Analysis of Fitness Test results}
Answers to the following fitness test questions were evaluated, to identify any pattern between leaders who completed {\bf{LDP}} and those who didn't. 
\begin{table}[H]
\centering
\begin{tabular}{l|c|p{5in}}
\hline
Goals 		& Q1 & I have clear agreements with my manager about my identified goals. \\
      		& Q2 & My manager holds me accountable to have SMART goals (specific, motivating, attainable, relevant and trackable). \\
      		& Q3 & I understand how my goals are related to/ directly aligned to the overall organization’s goals. \\
		& Q4 & My mutually agreed upon goals are motivating and appropriately challenging. \\
\hline
Diagnosing	& Q1 & My manager is able to fully assess my competence (knowledge and skill) for each one of my identified goals. \\
		& Q2 & My manager’s leadership style depends on my needs on a specific goal/task. \\
		& Q3 & My manager stays connected with what I am doing and gives me appropriate feedback. \\
		& Q4 & My manager asks how he/ she can support me as a leader.\\
\hline
Matching	& Q1 & My manager is able to assess my commitment (motivation and confidence) on each of my goals. \\
		& Q2 & My manager knows when to allow me to figure situations out and acts as a sounding board, not a provider of directives. \\
		& Q3 & My manager is comfortable being directive. \\
		& Q4 & My manager is comfortable being supportive. \\
\hline
Coaching	& Q1 & My manager is able to effectively coach me through potential challenges as I work towards my goals. \\
		& Q2 & I meet regularly one on one with my manager so I can ask for the direction and support I need. \\
		& Q3 & My manager invites feedback about how he or she could be more effective as a leader. \\
\hline
\end{tabular}
\end{table}

\subsection{Model Fitting}
As can be noticed in the boxplots in the end of the report, none of the answers to the 15 questions above illustrate any difference between the leaders who completed {\bf{LDP}} and those who didn't. This observation was consistent when a logistic regression model was fit to the data. Table 1 is a summary of the model fit. As can be seen in the table other than Goals-Q1, none of the answers have any significant association with LDP completion. The p-values are all greater than 0.05, hence, the associations are not statistically significant. 

Individual t-test results between the two groups yield similar results. For example, testing the hypothesis
$$H_0: \mu_{G1_A} = \mu_{G1_B} \hspace{1cm} vs \hspace{1cm} H_a: \mu_{G1_A} < \mu_{G1_B}$$ where $\mu_{G1_A}:$ Mean of answers of Goals, Q1 for leaders who did not complete LDP and $\mu_{G1_B}:$ Mean of answers of Goals, Q1 for leaders who completed LDP, two sample t-test yields 
$t = 0.207, \ df = 65.358, \ \text{p-value} = 0.5817$

% latex table generated in R 3.1.1 by xtable 1.7-4 package
% Wed Nov 11 16:50:21 2015
\begin{table}[H]
\centering
\begin{tabular}{rrrrr}
  \hline
  covariate	& estimate & std. error & z value & p-value \\ 
  \hline
  (Intercept) 	& 2.2582  & 4.2820 & 0.53  & 0.5979 \\ 
  Goals.Q1 	& 2.3669  & 1.1568 & 2.05  & 0.0408 \\ 
  Goals.Q2 	& -0.9224 & 1.2212 & -0.76 & 0.4501 \\ 
  Goals.Q3 	& 0.4169  & 1.2487 & 0.33  & 0.7385 \\ 
  Goals.Q4 	& -1.7714 & 0.9815 & -1.80 & 0.0711 \\ 
  Diagnosing.Q1 & -0.6215 & 0.8610 & -0.72 & 0.4704 \\ 
  Diagnosing.Q2 & -0.2532 & 0.7373 & -0.34 & 0.7313 \\ 
  Diagnosing.Q3 & -1.0259 & 0.8871 & -1.16 & 0.2475 \\ 
  Diagnosing.Q4 & -0.4572 & 0.8111 & -0.56 & 0.5730 \\ 
  Matching.Q1 	& 0.4789  & 1.0533 & 0.45  & 0.6494 \\ 
  Matching.Q2 	& -0.3249 & 0.9813 & -0.33 & 0.7406 \\ 
  Matching.Q3 	& -0.4516 & 0.7969 & -0.57 & 0.5710 \\ 
  Matching.Q4 	& 1.2323  & 1.0729 & 1.15  & 0.2507 \\ 
  Coaching.Q1 	& -0.1621 & 0.8654 & -0.19 & 0.8514 \\ 
  Coaching.Q2 	& 0.5628  & 0.6397 & 0.88  & 0.3789 \\ 
  Coaching.Q3 	& 0.4592  & 0.5343 & 0.86  & 0.3901 \\ 
  ResponsePct 	& 0.3030  & 0.7493 & 0.40  & 0.6859 \\ 
  \hline
\end{tabular}
\caption{Fitting Fitness test questions with LDP completion as the response}
\end{table}

When individual questions didn't yield significant results, average of each category was tested. For example, new covariate $Goals_{Avg}$ was created, which was the average of all the answers related to $Goals$. Similarly, $Diagnosing_{Avg}$, $Matching_{Avg}$ and $Coaching_{Avg}$. Below is the result of the model fit
% latex table generated in R 3.1.1 by xtable 1.7-4 package
% Wed Nov 11 21:44:34 2015
\begin{table}[H]
\centering
\begin{tabular}{rrrrr}
  \hline
 & estimate & std. error & z value & p-value \\ 
  \hline
(Intercept) & 3.7351 & 3.4058 & 1.10 & 0.2728 \\ 
  Goals.Avg & -0.5551 & 0.9138 & -0.61 & 0.5435 \\ 
  Diagnosing.Avg & -1.8530 & 1.1394 & -1.63 & 0.1039 \\ 
  Matching.Avg & 0.7192 & 1.2965 & 0.55 & 0.5791 \\ 
  Coaching.Avg & 0.9661 & 0.7996 & 1.21 & 0.2269 \\ 
   \hline
\end{tabular}
\caption{Fitting Average of Fitness test questions with LDP completion as the response}
\end{table}

Even in this model none of the question categories seem to yield statistically significant associations with LDP completions. 

Performance Rating of 2014 was then tested with LDP completion to detect any effect. Only three types of performance rating had big sample size to compare. 
% latex table generated in R 3.1.1 by xtable 1.7-4 package
% Wed Nov 11 21:58:40 2015
\begin{table}[H]
\centering
\begin{tabular}{lcccc}
 \hline
 & Exceeds Many & Meets Expectations & Meets Most & Total\\ 
 \hline
 LDP Yes &  18 &  32 &   6 & 56\\ 
 LDP No &  14 &  50 &  13 & 77\\ 
 \hline
 Total & 32 & 82 & 19 & 133 \\
 \hline
\end{tabular}
\end{table}
Pearson's Chi-squared test yields $\chi^2 = 3.8093, df = 2, \text{p-value} = 0.1489$. There is no statistically significant association between LDP completion and performance ratings. In this test, leaders who completed their LDP in 2015 were allocated to the group of ``LDP No'', since their performance evaluation was before their LDP completion. If however, there was reason to believe that 2015 performance evaluations would be similar to that of 2014, then the analysis would be different. Below is the table illustrating this:
% latex table generated in R 3.1.1 by xtable 1.7-4 package
% Thu Nov 12 12:36:50 2015
\begin{table}[H]
\centering
\begin{tabular}{lcccc}
\hline
& Exceeds Many & Meets Expectations & Meets Most & Total\\ 
\hline
LDP Yes &  21 &  55 &   6  & 82\\ 
LDP No &  11 &  27 &  13 & 51\\ 
\hline
Total & 32 & 82 & 19 & 133 \\
\hline
\end{tabular}
\end{table}
A Chi-square goodness of fit test does reveal statistically significant association between leaders who completed the LDP training and their receiving superior performance evaluations. The $\chi^2 = 8.5012,\ df = 2,\  \text{p-value} = 0.01426$

\section{Analysis of Your Voice survey}
Answers of 2015 survey outputs of the following categories were analyzed, to detect any impact of LDP training:
Direct Report Score, Manager Effectiveness, Quality, Trust, Growth and Development, Recognition, Business Acumen, Client Focus, Market Insight and  Communication. Wherever available, their 2014 scores were used as a baseline value for their 2015 numbers. For each category, the following analysis were conducted:
\begin{enumerate}
\item Boxplot - To visually detect any difference between people with and without LDP training
\item T-test - To detect any significant difference in scores between people with and without LDP training
\item Linear model - To estimate the effect of LDP training on the 2015 scores, controlling for their baseline 2014 scores. 
\[ y = \beta_0 + \beta_1 x + \beta_2 z\]
where, $y: $ 2015 score; $x: $ 2014 score of the same person; $z: $ If the person completed LDP or not. In each of the analysis, the null hypothesis is that LDP has not made any difference to their scores. If the coefficient $\beta_2$ is statistically significant (p-value $<$ 0.05), then we reject the null hypothesis, and conclude that LDP did have an impact. 

\end{enumerate}
Below are the results: \\
\subsection*{Analysis Results}
\begin{enumerate}
\item {\bf{Direct Report Scores}}\\
% first column
\begin{minipage}[t]{0.3\textwidth}
\begin{figure}[H]
\centering 
\includegraphics[scale=0.3, page=1]{BoxPlots-DR.pdf}
\end{figure}
\end{minipage}
%second column
\begin{minipage}[t]{0.6\textwidth}
\vspace{0.8cm}
t-test: p-value = 0.0000 \\
The model fit is:
% latex table generated in R 3.1.1 by xtable 1.7-4 package
% Wed Dec 16 14:51:50 2015
\begin{table}[H]
\centering
\begin{tabular}{rrrrr}
  \hline
 & Estimate & Std. Error & t value & p-value \\ 
  \hline
  (Intercept) & 0.7022 & 0.0192 & 36.53 & $0.0000^{**}$ \\ 
  Data2014 & 0.1268 & 0.0217 & 5.85 & $0.0000^{**}$ \\ 
  LDP No & -0.0521 & 0.0173 & -3.02 & $0.0027^{**}$ \\ 
   \hline
\end{tabular}
\end{table}
Conclusion: Both, t-test, and linear model, suggest that LDP did have an impact in yielding better direct report scores. p-value is 0.0027. 
\end{minipage}

\item {\bf{Manager Effectiveness}}\\
\begin{minipage}[t]{0.3\textwidth}
\begin{figure}[H]
\centering 
\includegraphics[scale=0.3, page=2]{BoxPlots-DR.pdf}
\end{figure}
\end{minipage}
%second column
\begin{minipage}[t]{0.6\textwidth}
\vspace{0.8cm}
t-test p-value = 0.03124 \\
The model fit is:
% latex table generated in R 3.1.1 by xtable 1.7-4 package
% Wed Dec 16 14:51:50 2015
\begin{table}[H]
\centering
\begin{tabular}{rrrrr}
  \hline
 & Estimate & Std. Error & t value & p-value \\ 
  \hline
(Intercept) & 0.7120 & 0.0201 & 35.39 & 0.0000 \\ 
  Data 2014 & 0.1446 & 0.0204 & 7.10 & 0.0000 \\ 
  LDP (No) & -0.0236 & 0.0181 & -1.30 & 0.1924 \\ 
   \hline
\end{tabular}
\end{table}
Conclusion: Although t-test yields significant different, when controlling for 2014 scores the inference changes. LDP did not seem to improve manager effectiveness.
\end{minipage}

\item {\bf{Quality}}\\
\begin{minipage}[t]{0.3\textwidth}
\begin{figure}[H]
\centering 
\includegraphics[scale=0.3, page=3]{BoxPlots-DR.pdf}
\end{figure}
\end{minipage}
%second column
\begin{minipage}[t]{0.6\textwidth}
\vspace{0.8cm}
t-test p-value = 0.1886 \\
The model fit is:
% latex table generated in R 3.1.1 by xtable 1.7-4 package
% Wed Dec 16 14:51:50 2015
\begin{table}[H]
\centering
\begin{tabular}{rrrrr}
  \hline
 & Estimate & Std. Error & t value & p-value \\ 
  \hline
(Intercept) & 0.6528 & 0.0188 & 34.79 & 0.0000 \\ 
  Data 2014 & 0.1249 & 0.0219 & 5.71 & 0.0000 \\ 
  LDP (No) & -0.0100 & 0.0171 & -0.59 & 0.5579 \\ 
   \hline
\end{tabular}
\end{table}
Conclusion: Neither the t-test, nor the linear model show any impact of LDP on quality. 
\end{minipage}

\item {\bf{Trust}}\\
\begin{minipage}[t]{0.3\textwidth}
\begin{figure}[H]
\centering 
\includegraphics[scale=0.3, page=4]{BoxPlots-DR.pdf}
\end{figure}
\end{minipage}
%second column
\begin{minipage}[t]{0.6\textwidth}
\vspace{0.8cm}
t-test p-value = 0.0066 \\
The model fit is:
% latex table generated in R 3.1.1 by xtable 1.7-4 package
% Wed Dec 16 14:51:51 2015
\begin{table}[H]
\centering
\begin{tabular}{rrrrr}
  \hline
 & Estimate & Std. Error & t value & p-value \\ 
  \hline
(Intercept) & 0.8397 & 0.0166 & 50.50 & 0.0000 \\ 
  Data 2014 & 0.0647 & 0.0151 & 4.28 & 0.0000 \\ 
  LDP (No) & -0.0271 & 0.0145 & -1.87 & 0.0624 \\ 
   \hline
\end{tabular}
\end{table}
Conclusion: The t-test is more significant than the linear model results. As per the linear model, LDP did improve trust, but only at a 0.10 significance level (not the conventional 0.05)
\end{minipage}

\item {\bf{Growth and Development}}\\
\begin{minipage}[t]{0.3\textwidth}
\begin{figure}[H]
\centering 
\includegraphics[scale=0.3, page=5]{BoxPlots-DR.pdf}
\end{figure}
\end{minipage}
%second column
\begin{minipage}[t]{0.6\textwidth}
\vspace{0.8cm}
t-test p-value = 0.0097 \\
The model fit is:
% latex table generated in R 3.1.1 by xtable 1.7-4 package
% Wed Dec 16 14:51:51 2015
\begin{table}[H]
\centering
\begin{tabular}{rrrrr}
  \hline
 & Estimate & Std. Error & t value & p-value \\ 
  \hline
(Intercept) & 0.7744 & 0.0194 & 39.82 & 0.0000 \\ 
  Data 2014 & 0.0953 & 0.0191 & 5.00 & 0.0000 \\ 
  LDP (No) & -0.0296 & 0.0174 & -1.70 & 0.0890 \\ 
   \hline
\end{tabular}
\end{table}
Conclusion: The t-test is more significant than the linear model results. As per the linear model, LDP did improve growth and development, but only at a 0.10 significance level (not the conventional 0.05)
\end{minipage}

\item {\bf{Recognition}}\\
\begin{minipage}[t]{0.3\textwidth}
\begin{figure}[H]
\centering 
\includegraphics[scale=0.3, page=6]{BoxPlots-DR.pdf}
\end{figure}
\end{minipage}
%second column
\begin{minipage}[t]{0.6\textwidth}
\vspace{0.8cm}
t-test p-value = 0.00034 \\
The model fit is:
% latex table generated in R 3.1.1 by xtable 1.7-4 package
% Wed Dec 16 14:51:51 2015
\begin{table}[H]
\centering
\begin{tabular}{rrrrr}
  \hline
 & Estimate & Std. Error & t value & p-value \\ 
  \hline
(Intercept) & 0.8209 & 0.0189 & 43.47 & 0.0000 \\ 
  Data 2014 & 0.0821 & 0.0176 & 4.67 & 0.0000 \\ 
  LDP (No) & -0.0413 & 0.0166 & -2.48 & 0.0134 \\ 
  \hline
\end{tabular}
\end{table}
Conclusion: Both, the t-test and linear model show very significant impact of LDP in improving recognition. p-value is 0.0134. 
\end{minipage}

\item {\bf{Business Acumen}}\\
\begin{minipage}[t]{0.3\textwidth}
\begin{figure}[H]
\centering 
\includegraphics[scale=0.3, page=7]{BoxPlots-DR.pdf}
\end{figure}
\end{minipage}
%second column
\begin{minipage}[t]{0.6\textwidth}
\vspace{0.8cm}
t-test p-value = 0.0000 \\
The model fit is:
% latex table generated in R 3.1.1 by xtable 1.7-4 package
% Wed Dec 16 17:22:36 2015
\begin{table}[H]
\centering
\begin{tabular}{rrrrr}
  \hline
 & Estimate & Std. Error & t value & p-value \\ 
  \hline
(Intercept) & 0.7757 & 0.0207 & 37.56 & 0.0000 \\ 
  Data 2014 & 0.1563 & 0.0190 & 8.23 & 0.0000 \\ 
  LDP (No) & -0.0557 & 0.0182 & -3.06 & 0.0023 \\ 
   \hline
\end{tabular}
\end{table}
Conclusion: Both, the t-test and linear model show very significant impact of LDP in improving business acumen. p-value is 0.0023. 
\end{minipage}

\item {\bf{Client Focus}}\\
\begin{minipage}[t]{0.3\textwidth}
\begin{figure}[H]
\centering 
\includegraphics[scale=0.3, page=8]{BoxPlots-DR.pdf}
\end{figure}
\end{minipage}
%second column
\begin{minipage}[t]{0.6\textwidth}
\vspace{0.8cm}
t-test p-value = 0.0011 \\
The model fit is:
% latex table generated in R 3.1.1 by xtable 1.7-4 package
% Wed Dec 16 14:51:52 2015
\begin{table}[H]
\centering
\begin{tabular}{rrrrr}
  \hline
 & Estimate & Std. Error & t value & p-value \\ 
  \hline
(Intercept) & 0.8509 & 0.0179 & 47.47 & 0.0000 \\ 
  Data 2014 & 0.0816 & 0.0160 & 5.11 & 0.0000 \\ 
  LDP (No) & -0.0345 & 0.0157 & -2.19 & 0.0286 \\ 
   \hline
\end{tabular}
\end{table}
Conclusion: Both, the t-test and linear model show very significant impact of LDP in improving client focus. p-value is 0.0286. 
\end{minipage}

\item {\bf{Market Insight}}\\
\begin{minipage}[t]{0.3\textwidth}
\begin{figure}[H]
\centering 
\includegraphics[scale=0.3, page=9]{BoxPlots-DR.pdf}
\end{figure}
\end{minipage}
%second column
\begin{minipage}[t]{0.6\textwidth}
\vspace{0.8cm}
t-test p-value = 0.0000 \\
The model fit is:
% latex table generated in R 3.1.1 by xtable 1.7-4 package
% Wed Dec 16 14:51:53 2015
\begin{table}[H]
\centering
\begin{tabular}{rrrrr}
  \hline
 & Estimate & Std. Error & t value & p-value \\ 
  \hline
(Intercept) & 0.7512 & 0.0208 & 36.15 & 0.0000 \\ 
  Data 2014 & 0.1723 & 0.0192 & 8.97 & 0.0000 \\ 
  LDP (No) & -0.0448 & 0.0183 & -2.45 & 0.0147 \\ 
   \hline
\end{tabular}
\end{table}
Conclusion: Both, the t-test and linear model show very significant impact of LDP in improving market insight. p-value is 0.0147. 
\end{minipage}

\item {\bf{Communication}}\\
\begin{minipage}[t]{0.3\textwidth}
\begin{figure}[H]
\centering 
\includegraphics[scale=0.3, page=10]{BoxPlots-DR.pdf}
\end{figure}
\end{minipage}
%second column
\begin{minipage}[t]{0.6\textwidth}
\vspace{0.8cm}
t-test p-value = 0.0004 \\
The model fit is:
% latex table generated in R 3.1.1 by xtable 1.7-4 package
% Wed Dec 16 14:51:53 2015
\begin{table}[H]
\centering
\begin{tabular}{rrrrr}
  \hline
 & Estimate & Std. Error & t value & p-value \\ 
  \hline
(Intercept) & 0.6574 & 0.0229 & 28.71 & 0.0000 \\ 
  Data 2014 & 0.1275 & 0.0280 & 4.56 & 0.0000 \\ 
  LDP (No) & -0.0607 & 0.0211 & -2.87 & 0.0042 \\ 
   \hline
\end{tabular}
\end{table}
Conclusion: Both, the t-test and linear model show very significant impact of LDP in improving communication. p-value is 0.0042. 
\end{minipage}
\end{enumerate}

\section{Analysis of 2015 Bonus compensation}
Below are plots of the 2015 bonus compensation broken down by whether the leaders completed LDP training or not, and also, whether they attended SLM training or not.
\begin{figure}[H]
\centering 
\includegraphics[scale=0.45, page=1]{BoxPlots-Bonus.pdf}
\includegraphics[scale=0.45, page=2]{BoxPlots-Bonus.pdf}
\end{figure}
As is evident, there is not much difference in bonus compensation when broken down by LDP completion. However, SLM seems to have strong influence. Below is a boxplot broken down by LDP and SLM simultaneously.
\begin{figure}[H]
\centering 
\includegraphics[scale=0.55, page=3]{BoxPlots-Bonus.pdf}
\end{figure}
Fitting this model, with SLM and LDP, and controlling for 2014 compensation as a baseline, we have:
% latex table generated in R 3.1.1 by xtable 1.7-4 package
% Wed Dec 16 23:42:06 2015
\begin{table}[H]
\centering
\begin{tabular}{rrrrr}
  \hline
 & Estimate & Std. Error & t value & p-value \\ 
  \hline
  (Intercept) & 47501 & 4731.3547 & 10.04 & 0.0000 \\ 
  Bonus\_2014 & 0.15 & 0.0448 & 3.35 & 0.0009 \\ 
  SLMYes & 35607 & 7439.0652 & 4.79 & 0.0000 \\ 
  LDP.CompletionNo & -7995 & 7070.1123 & -1.13 & 0.2592 \\ 
  \hline
\end{tabular}
\end{table}
The model fit confirms that although leaders who completed LDP receive an average of \$7,995 higher than those who did not complete LDP, but that difference is not statistically significant. Leaders who attended SLM on an average receive \$35,607 higher than who did not. This difference is statistically significant (p-value = 0).

In addition to analyzing the bonus amount, the ``percentage bonus allocated'' variable was also analyzed. Below is a plot of this variable:
\begin{figure}[H]
\centering 
\includegraphics[scale=0.55, page=3]{BoxPlots-Bonus_pct.pdf}
\end{figure}
The model fitting bonus percentage is below:
% latex table generated in R 3.2.2 by xtable 1.8-0 package
% Mon Dec 21 23:22:38 2015
\begin{table}[H]
\centering
\begin{tabular}{rrrrr}
  \hline
 & Estimate & Std. Error & t value & p-valie \\ 
  \hline
(Intercept) & 0.0142 & 0.0055 & 2.58 & 0.0105 \\ 
  Bonus\_2014\_pct & 0.9355 & 0.0173 & 54.13 & 0.0000 \\ 
  SLMYes & 0.0175 & 0.0076 & 2.31 & 0.0216 \\ 
  LDP.CompletionNo & 0.0108 & 0.0072 & 1.49 & 0.1378 \\ 
   \hline
\end{tabular}
\end{table}
Neither the boxplot, nor the linear model indicate any impact of LDP completion on bonus percentage. 

\newpage
\section{Executive Summary}
\begin{enumerate}
\item None of the fitness test attributes had any statistically significant improvement as a result of leaders attending LDP training. \\
\item In your voice survey, there were several areas where statistically significant improvements were observed. Below is a table, ranked in order of decreasing statistical significance, with LDP training
\begin{table}[H]
\centering
\begin{tabular}{l|c|l}
\hline
\hline
Attribute & p-value & Comment \\
\hline
Business Acumen 	& 0.0023 & Strongly Significant \\
Direct Report Score 	& 0.0027 & Strongly significant \\
Communication 		& 0.0042 & Strongly significant \\
Recognition 		& 0.0134 & Significant \\
Market Insight 		& 0.0147 & Significant \\
Client Focus 		& 0.0286 & Significant \\
Trust 			& 0.0624 & Somewhat significant \\
Growth and Development 	& 0.0890 & Somewhat significant \\
Manager Effectiveness 	& 0.1924 & Not significant \\
Quality 		& 0.5579 & Not significant \\
\hline
\hline
\end{tabular}
\end{table}

\item Bonus compensation in 2015, both total amount and percentage, was more strongly associated with leaders attending SLM and not as much by LDP completion.

\end{enumerate}


\newpage
\section{Appendix - Summary Plots}
Below are the boxplots {\footnote{The interpretation of boxplots can be found here: https://en.wikipedia.org/wiki/Box\_plot\#/media/File:Boxplot\_vs\_PDF.svg}} for Goals-Q1
\begin{figure}[H]
\centering 
\includegraphics[scale=0.45, page=1]{../BoxPlots.pdf}
\includegraphics[scale=0.45, page=2]{../BoxPlots.pdf} \\
\includegraphics[scale=0.65, page=3]{../BoxPlots.pdf}
\end{figure}

\newpage
Below are the boxplots for Goals-Q2
\begin{figure}[H]
\centering 
\includegraphics[scale=0.45, page=4]{../BoxPlots.pdf} 
\includegraphics[scale=0.45, page=5]{../BoxPlots.pdf} \\
\includegraphics[scale=0.65, page=6]{../BoxPlots.pdf} \\
\end{figure}

\newpage
Below are the boxplots for Goals-Q3
\begin{figure}[H]
\centering 
\includegraphics[scale=0.45, page=7]{../BoxPlots.pdf} 
\includegraphics[scale=0.45, page=8]{../BoxPlots.pdf} \\
\includegraphics[scale=0.65, page=9]{../BoxPlots.pdf} \\
\end{figure}

\newpage
Below are the boxplots for Goals-Q4
\begin{figure}[H]
\centering 
\includegraphics[scale=0.45, page=10]{../BoxPlots.pdf} 
\includegraphics[scale=0.45, page=11]{../BoxPlots.pdf} \\
\includegraphics[scale=0.65, page=12]{../BoxPlots.pdf} \\
\end{figure}

\newpage
Below are the boxplots for Diagnosing-Q1
\begin{figure}[H]
\centering 
\includegraphics[scale=0.45, page=13]{../BoxPlots.pdf} 
\includegraphics[scale=0.45, page=14]{../BoxPlots.pdf} \\
\includegraphics[scale=0.65, page=15]{../BoxPlots.pdf} \\
\end{figure}

\newpage
Below are the boxplots for Diagnosing-Q2
\begin{figure}[H]
\centering 
\includegraphics[scale=0.45, page=16]{../BoxPlots.pdf} 
\includegraphics[scale=0.45, page=17]{../BoxPlots.pdf} \\
\includegraphics[scale=0.65, page=18]{../BoxPlots.pdf} \\
\end{figure}

\newpage
Below are the boxplots for Diagnosing-Q3
\begin{figure}[H]
\centering 
\includegraphics[scale=0.45, page=19]{../BoxPlots.pdf} 
\includegraphics[scale=0.45, page=20]{../BoxPlots.pdf} \\
\includegraphics[scale=0.65, page=21]{../BoxPlots.pdf} \\
\end{figure}

\newpage
Below are the boxplots for Diagnosing-Q4
\begin{figure}[H]
\centering 
\includegraphics[scale=0.45, page=22]{../BoxPlots.pdf} 
\includegraphics[scale=0.45, page=23]{../BoxPlots.pdf} \\
\includegraphics[scale=0.65, page=24]{../BoxPlots.pdf} \\
\end{figure}

\newpage
Below are the boxplots for Matching-Q1
\begin{figure}[H]
\centering 
\includegraphics[scale=0.45, page=25]{../BoxPlots.pdf} 
\includegraphics[scale=0.45, page=26]{../BoxPlots.pdf} \\
\includegraphics[scale=0.65, page=27]{../BoxPlots.pdf} \\
\end{figure}

\newpage
Below are the boxplots for Matching-Q2
\begin{figure}[H]
\centering 
\includegraphics[scale=0.45, page=28]{../BoxPlots.pdf} 
\includegraphics[scale=0.45, page=29]{../BoxPlots.pdf} \\
\includegraphics[scale=0.65, page=30]{../BoxPlots.pdf} \\
\end{figure}

\newpage
Below are the boxplots for Matching-Q3
\begin{figure}[H]
\centering 
\includegraphics[scale=0.45, page=31]{../BoxPlots.pdf} 
\includegraphics[scale=0.45, page=32]{../BoxPlots.pdf} \\
\includegraphics[scale=0.65, page=33]{../BoxPlots.pdf} \\
\end{figure}

\newpage
Below are the boxplots for Matching-Q4
\begin{figure}[H]
\centering 
\includegraphics[scale=0.45, page=34]{../BoxPlots.pdf} 
\includegraphics[scale=0.45, page=35]{../BoxPlots.pdf} \\
\includegraphics[scale=0.65, page=36]{../BoxPlots.pdf} \\
\end{figure}

\newpage
Below are the boxplots for Coaching-Q1
\begin{figure}[H]
\centering 
\includegraphics[scale=0.45, page=37]{../BoxPlots.pdf} 
\includegraphics[scale=0.45, page=38]{../BoxPlots.pdf} \\
\includegraphics[scale=0.65, page=39]{../BoxPlots.pdf} \\
\end{figure}

\newpage
Below are the boxplots for Coaching-Q2
\begin{figure}[H]
\centering 
\includegraphics[scale=0.45, page=40]{../BoxPlots.pdf} 
\includegraphics[scale=0.45, page=41]{../BoxPlots.pdf} \\
\includegraphics[scale=0.65, page=42]{../BoxPlots.pdf} \\
\end{figure}

\newpage
Below are the boxplots for Coaching-Q3
\begin{figure}[H]
\centering 
\includegraphics[scale=0.45, page=43]{../BoxPlots.pdf} 
\includegraphics[scale=0.45, page=44]{../BoxPlots.pdf} \\
\includegraphics[scale=0.65, page=45]{../BoxPlots.pdf} \\
\end{figure}


\end{document}
